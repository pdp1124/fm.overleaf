\part{Trasmissione e Trasformazione del Movimento }
\chapter{Breve Introduzione}

La movimentazione delle macchine, dei macchinari e degli impianti avviene, nella stragrande maggioranza dei casi,
tramite i {\em motori}\index{motori}. La situazione di gran lunga pi\`u diffusa
vede la sorgente primaria di moto in uno o pi\`u
motori che presentano, in generale,  un albero cilindrico tramite il quale
mettono a disposizione dell'utilizzatore
la potenza meccanica necessaria sotto forma di rotazione di tale albero e di un momento torcente ad esso applicato.
Mi preme tranquillizzare subito quei lettori che iniziassero a fremere
pensando a eccezioni 
illustri ai motori rotativi, e dico loro che il periodo  sopra riportato ci
fornisce soltanto un sistema di
riferimento per inquadrare il problema, e nulla toglie alla sua generalit\`a.
Presenteremo infatti in questa parte del volume
la cinematica di alcuni dispositivi che permettono  di
mettere in relazione reciproca movimenti geometricamente
simili, ovvero geometricamente eterogenei tra loro.
Partiremo  quindi
considerando i casi pi\`u semplici di trasformazione di
movimenti rotatori in movimenti ancora rotatori, e ci spingeremo fino a
studiare il comportamento cinematico
di dispositivi che, sfruttando un movimento di ingresso rotatorio
(spesso a velocit\`a costante), producono in uscita leggi di movimento o traiettorie
molto particolari.
Tali leggi di moto e tali particolari traiettorie saranno ottenute tramite
{\em sistemi articolati}\index{sistemi!articolati}
ed {\em eccentrici}\index{eccentrici} che, in gergo, 
prendono il nome di {\em camme}\index{camme}.

\noindent Pi\`u nello specifico
si parler\`a di
{\em trasmissioni}\index{trasmissione!del moto}
nei casi frequentissimi di generazione di un moto rotatorio da uno simile.
Si annoverano in questa famiglia di dispositivi meccanici
le {\em ruote dentate}\index{ruote!dentate}, delle quali un esemplare
\`e mostrato in figura \ref{fig:f2a}, e gli {\em ingranaggi}\index{ingranaggi}
che esse costituiscono,
le {\em cinghie di trasmissione}\index{cinghie di trasmissione} con le
relative {\em pulegge}, le
{\em catene}, i {\em giunti} omocinetici e non ecc. ma anche, di nuovo,
quadrilateri e altri sistemi articolati.
Il che mostra che la trasmissione del moto e la generazione di particolari
movimenti costituiscono, in fondo,
un'unica problematica della Meccanica Applicata alle
Macchine, anche se i due temi, quello delle trasmissioni e quello
dei meccanismi che generano movimenti peculiari,
posseggono specificit\`a cos\`i
profonde da esser tratti, nei libri, quasi sempre in capitoli disgiunti,
quando non occupano trattazioni specialistiche distinte.
Siccome questo volume \`e dedicato alla Cinematica, per quanto
riguarda la trasmissione del moto rotatorio studieremo
soltanto le ruote dentate, sorvolando sulle cinghie di trasmissione che,
bench\'e rappresentino un dispositivo meccanico molto frequentemente
impiegato e manifestino interessanti problematiche dinamiche,
dal punto di vista cinematico non offrono spunti di particolare
spicco.
\begin{wrapfigure}{r}{0.5\textwidth}
     \begin{center}
     \includegraphics[width=0.42\textwidth]{part2/ruote/FIG/f2a.jpeg}
     \end{center}
        \caption{\em Ruota dentata.}
     \label{fig:f2a}
\end{wrapfigure}
\noindent Quando parliamo di ``generazione di particolari movimenti''
intendiamo, in generale, la possibilit\`a di
produrre moti le cui leggi temporali dello spostamento o delle
sue derivate abbiano determinate caratteristiche, oppure di generare
opportune traiettorie a partire da moti rotatori.
I meccanismi  atti a questi particolari scopi verranno
quindi analizzati in ambito cinematico e tali studi, stante lo scopo di questi
appunti e lo spazio limitato, non avranno alcuna pretesa di completezza n\'e di
profondit\`a.
Introdurremo cos\`i i {\em sistemi articolati}\index{sistemi!articolati}, in particolare i quadrilateri, che possono trasformare 
un moto rotatorio con velocit\`a angolare costante in moti
con velocit\`a angolare variabile, oppure condurre qualche loro
punto, normalmente appartenente alla biella, che chiameremo pertanto {\em punto
di biella}\index{biella!punto di}, a percorrere traiettorie
speciali, le quali possono, ad esempio, approssimare
spostamenti rettilinei: traiettorie
queste molto pregiate nell'ambito delle macchine automatiche 
per impiego manifatturiero. 
Vi sono poi i {\em manovellismi}\index{manovellismo} (casi degeneri dei
quadrilateri) che trasformano il moto rettilineo in  moto
rotatorio e viceversa. Essi imperversano da pi\`u di cent'anni nel
settore dei motori a combustione interna, e tuttora tengono banco essendo
presenti (in genere a quaterne) nei motori delle nostre attuali automobili.
\noindent Esiste, infine, la sofisticata possibilit\`a di
provvedere un {\em cedente}\index{cedente} di precise
{\em leggi di moto}\index{leggi di moto}.
Spesso tale cedente consiste in un'asta guidata da due collari,
azionata tramite dispositivi meccanici
eccentrici, opportunamente profilati: le {\em camme}.


\noindent Ciascuna delle famiglie di dispositivi appena
citati merita, e di fatto possiede, trattazioni
specialistiche monotematiche e il loro approfondimento in un corso di Meccanica
Applicata di base risulterebbe eccessivamente tecnico, perci\`o fuori luogo.
La nostra intenzione \`e quella di mostrare, in questa parte del presente tomo,
alcuni (pochi) membri appartenenti alle tipologie citate di meccanismi,
passandoli in rassegna e tentando di fornire,
ove possibile, qualche spunto per la loro analisi cinematica e per 
la loro sintesi.

\chapter{Trasmissione del Moto mediante Ruote Dentate}\index{ruote!dentate}\label{cap_ruote_ev}

\section{Le Ruote di Frizione}
\noindent Credo che a ben pochi lettori sfugga, almeno da un punto di vista
meramente visivo, cosa sia una {\em ruota dentata}.

\begin{wrapfigure}{r}{0.48\textwidth}
      \begin{center}
      \includegraphics[width=0.43\textwidth]{part2/ruote/FIG/f22.pdf}
     \end{center}
\begin{picture}(0,0)(-63,1)
\scriptsize{
\put(21,235){$\bm F$}
\put(-2,190){$r_1$}
\put(4,108){$r_2$}
\put(-55,159){$\omega_1$}
\put(-55,40){$\omega_2$}
\put(21,40){$\bm F$}
\put(65,89){$v_1=v_2$}
}
\end{picture}
        \caption{\em Ruote di frizione.}
     \label{fig:f22}
\end{wrapfigure}

\noindent Le ruote dentate sono comparse alla maggior parte di noi in contesti
eterogenei che vanno dai giochi per bambini agli elettrodomestici, ad alcuni
strumenti da cartoleria, senza parlare dell'emblema della Repubblica Italiana.
Qualche privilegiato ha potuto vedere le ruote dentate impiegate nella
meccanica delle automobili e dei motoveicoli. Tutti possono agevolmente 
immaginare (qualora non ne avessimo contezza diretta) il relativo ingranamento di una ruota dentata con una sua simile.  

\noindent Lasciamo per il momento da parte la caratteristica fondamentale di queste
ruote, che si manifesta tramite una serie di sporgenze opportune, chiamate denti.
Fissiamo invece 
le idee su due cilindri, che presentino raggi adeguati ai nostri
scopi, pensando di calettarli sui
rispettivi alberi, paralleli e liberi di ruotare attorno ai propri assi.
Possiamo fare in modo che le superfici cilindriche siano tra loro
a contatto e che tale contatto sia mantenuto da una data forza $\bm F$.
La figura \ref{fig:f22} rappresenta i due cilindri coi rispettivi alberi.
Ipotizzando che il moto relativo di una ruota rispetto all'altra sia di puro
rotolamento, possiamo affermare che le velocit\`a angolari dei due alberi
dovranno essere tali da produrre velocit\`a periferiche uguali
tra loro sulle due ruote.
Cio\`e dovr\`a valere $v_1=\omega_1 r_1 = v_2 = \omega_2 r_2$, 
relazione che permette di ricavare subito il {\em rapporto di trasmissione}\index{rapporto di trasmissione} della coppia di 
{\em ruote di frizione}\index{ruote!di frizione},
\begin{equation}
\tau={\omega_1\over{\omega_2}}=
{{r_2}\over{r_1}}\,.
\label{eq:tau}
\end{equation}
\noindent I due alberi ruotano pertanto con velocit\`a angolari differenti tra
loro e in strettissima relazione coi raggi delle due ruote che trasmettono
il moto. Nonostante le ruote di frizione siano realmente esistite nel
mondo della meccanica\footnote
{
Un curioso esempio di applicazione: il FERMIAC era un calcolatore analogico basato su ruote di frizione.
}, e si trovino ancora in applicazioni dove le
potenze in gioco sono sia molto grandi sia molto piccole, credo che il
loro pi\`u grave inconveniente potenziale, cio\`e
la possibilit\`a che il loro moto relativo di puro rotolamento venga
contaminato dallo strisciamento, sia evidente anche ad occhi non
particolarmente esperti. Non riteniamo opportuno chiarire
in questa sede il  meccanismo dell'attrito (in prevalenza statico)
necessario a trasmettere correttamente il moto in determinate condizioni,
n\'e illustreremo come possono essere realizzate, nella pratica, le coppie
di ruote di frizione, che tuttavia si fabbricano ancora, come
abbiamo appena affermato, anche
nel contesto di rilevanti potenze da trasmettere.
In questo luogo, ci basta osservare che
la trasmissione corretta del moto, tramite questi dispositivi,
\`e condizionata dal valore della forza $\bm F$ mediante la quale
le ruote sono costrette a mantenere il contatto tra loro.
In cascata, il valore efficace di tale forza \`e subordinato
al coefficiente di attrito tra le due superfici, a sua volta soggetto
a variazioni non leggere del suo valore,
a causa dell'usura e delle condizioni, probabilmente mutevoli, di
lubrificazione delle ruote stesse.
La garanzia del rapporto di trasmissione teorico, derivante dal rapporto tra i
diametri delle ruote, \`e perci\`o negata dai probabili scorrimenti
tra le due primitive nel punto di contatto.


\section{Percorso Intuitivo verso le Ruote Dentate} \label{percorso-ruote}

\noindent Riprendendo il filo dal precedente paragrafo,
siamo naturalmente tentati di impedire lo slittamento della trasmissione
ponendo su una delle ruote
alcune sporgenze e ricavando sulla seconda ruota degli incavi, tali da accogliere
le suddette sporgenze.
Inizieremo a breve un cammino che possiamo definire {\em induttivo}, a valle
del quale riteniamo che la soluzione,
adottata nella pratica come {\em standard} per la creazione delle
ruote dentate, dovrebbe essere ricevuta senza perplessit\`a. 
Premettiamo per\`o alcune considerazioni che riguardano il taglio che abbiamo
desiderato dare all'esposizione del presente argomento.
Le ruote dentate rappresentano nella meccanica gli organi maggiormente
diffusi nell'ambito della trasmissione e della trasformazione del moto, in
generale rotatorio. Si comprende perci\`o facilmente la loro
centralit\`a
in quasi tutte le applicazioni industriali. Crediamo di non esagerare dicendo
che esse rappresentano l'oggetto simbolo della meccanica e del funzionamento
delle macchine.
Pertanto, anche se nella Meccanica Applicata alle Macchine esse godono di una
posizione privilegiata, il loro studio si incontra anche in altre discipline
dell'ingegneria: nel Disegno di Macchine, nella Costruzione di Macchine e nella
Tecnologia Meccanica. Naturalmente, ci\`o accade anche per altri argomenti
di particolare interesse nella nostra 
materia. Tuttavia, ci \`e sembrato di notare 
che le ruote dentate vengano esposte anche altrove
con molta attenzione persino ai loro aspetti cinematici
e geometrici che ne determinano un
buono o cattivo funzionamento: cosa strana, questa, perch\'e l'aspetto
cinematico legato all'ingranamento \`e un argomento piuttosto intricato ma,
forse per questo, si ritiene che non sia mai chiarito a sufficienza.
Cercheremo, come sempre in queste note,
di rimanere circoscritti ai soli aspetti cinematico-geometrici
di questi dispositivi, divagando solamente quando qualche altra loro
caratteristica
(resistenza, facilit\`a di realizzazione e possibilit\`a di impiego) 
sar\`a determinante nel condizionarne la forma. Per tale motivo, questo volume,
che tratta esclusivamente la cinematica, non include
una porzione massiccia di particolarit\`a e di connotazioni delle
ruote dentate, alcune anche molto comuni, altre maggiormente 
tecniche, fondamentali per il loro
effettivo impiego. Un esempio che crediamo significativo:
non tratteremo le ruote con dentatura elicoidale, di grande importanza
nella pratica ma cinematicamente non dissimili dalle loro sorelle
a denti diritti.
Assicuriamo coloro i quali volessero
investigare le ruote dentate nei loro
aspetti pi\`u profondi e nella loro variet\`a di tipologie e applicazioni
che testi molto pi\`u completi, autorevoli e dettagliati
del nostro non mancano. A nostro parere, ad esempio,
questo argomento \`e esposto in modo rigoroso ed esaustivo in
\cite{pellicano}; indichiamo anche un manuale, a livello di scuola superiore,
che riporta un numero notevole di esempi utili per 
il progettista: \cite{punzi}. La ``Bibbia'' in questo campo, ad oggi, rimane \cite{henriot}.

\noindent Come abbiamo gi\`a accennato, riteniamo che
da un punto di vista didattico
sia efficace seguire una sorta di sentiero, percorrendo il quale
saremo costretti ad ammettere
che la possibilit\`a di proseguire sia condizionata dall'accettazione
di alcuni ``precetti'', pi\`u o meno tassativi, dettati da particolari
e stringenti circostanze.
Lungo tale percorso incontreremo pertanto
precetti forti e precetti deboli, e queste due qualit\`a saranno indicate tra
parentesi. Un precetto forte non potr\`a ammettere eccezioni, mentre uno
debole potr\`a essere in qualche caso disatteso, e quando lo sar\`a,
cercheremo di chiarire il perch\'e.
Inizieremo il nostro cammino dalle
ruote di frizione introdotte poco sopra, proponendo
un tentativo per la loro modifica,
in modo da assicurarle dal pericolo di
scorrere l'una sull'altra.
L'esperimento (mentale) che si pu\`o condurre \`e quello
di dotare una delle due ruote di alcune sporgenze che, inserendosi
in opportuni incavi ricavati sull'altra ruota, riescano a trasmettere
la componente tangenziale della forza, che le due ruote si scambiano,
senza basarsi sul fenomeno dell'attrito. Potremmo 
pensare alla ruota che deve essere incavata come fatta di plastilina
o di un qualunque materiale plastico e tenero, che possa poi indurire.
In questa materia plastica la ruota recante le asperit\`a potr\`a
pertanto operare un'azione
di incisione. Questo procedimento di taglio deve essere
condotto pensando che le due circonferenze delle ruote di frizione, 
che chiameremo {\em circonferenze primitive}\index{circonferenza!primitiva},
rotolino l'una sull'altra senza strisciare. Si intuisce che,
i due profili, quello della ruota creatrice
unitamente a quello tagliato nella plastilina,
una volta indurito, saranno
in grado di trasmettere il moto allo stesso modo delle due primitive
rotolanti, cio\`e con rapporto di trasmissione
costante.

\begin{figure}[hbt]
\begin{center}
\includegraphics[width=0.8\textwidth]{part2/ruote/FIG/ruote/belbon.pdf}
\end{center}
\begin{picture}(0,0)(130,0)
\scriptsize{
\put(229,101){$\alpha=180^{\circ}$}
\put(214,34){$\beta=270^{\circ}$}
\put(350,63){ruota 1 tagliata}
\put(250,50){ruota 2 creatrice}
}
\end{picture}
\vskip -7mm
      \caption{\em Dentatura ingenua, 
ottenuta mediante l'applicazione di sporgenze triangolari alla ruota di
sinistra. $\alpha$ \`e l'arco di dentatura della ruota azzurra,
$\beta$ la rotazione della ruota rosa.}
 \label{fig:belbon}
\end{figure}

\noindent La figura \ref{fig:belbon} rappresenta una ruota creatrice,
di colore rosa, recante delle
sporgenze a triangolo isoscele all'esterno della propria circonferenza primitiva, 
esterne cio\`e alla circonferenza della ruota di frizione.
Tale ruota, di colore rosa,``intaglia'', tramite le sue sporgenze,
i denti sulla ruota azzurra.
Il primo precetto che emerge (forte) esige
un lieto fine per la dentatura della ruota tagliata (azzurra).
In altre parole, risulta fondamentale che,
una volta completato il taglio sui $360^{\circ}$ di tale ruota,
essa presenti
un numero intero di denti. Riteniamo inutile commentare questa
indicazione. Aggiungiamo solamente che, se il numero
di denti sulla ruota azzurra risultasse frazionario,
la sua dentatura verrebbe
distrutta da ulteriori passaggi della ruota creatrice.
Allo scopo di prevenire il citato inconveniente e di ottemperare
a questo precetto, la progettazione delle 
ruote dentate si basa principalmente sul loro numero di denti e
non sui loro diametri,
rendendo cos\`i di fatto impossibili rapporti di trasmissione rappresentati
da frazioni stravaganti o, peggio, da numeri irrazionali. Per fortuna
il mondo della tecnica non richiede tali prestazioni!
Scelta quindi una coppia di numeri di denti che meglio si adatta al
rapporto di trasmissione che vogliamo realizzare, giungeremo poi
alla reale dimensione delle ruote introducendo un numero dimensionale
che chiamiamo {\em modulo}\index{modulo}, $m$, misurato in millimetri,
legato al diametro della ruota nel modo seguente: $D= z\, m$, con $z$
numero di denti della ruota e $D$  diametro primitivo (quello
della corrispondente ruota di frizione). Va da s\'e che il
{\em passo}\index{passo}, cio\`e
la misura dell'arco di circonferenza primitiva che separa i due fianchi
attigui e omologhi, sinistri o destri, di due denti vale $p=\pi m$.

\noindent Un secondo precetto (debole) \`e la richiesta, per le asperit\`a del
creatore, di simmetria speculare rispetto a un raggio.
In effetti, le nostre sporgenze sono triangoli isosceli e ci\`o permette di
ottenere condizioni geometriche pari nei due sensi di marcia. Questa
\`e una restrizione che viene, in particolarissimi e rari casi, disattesa.
Ci accorgiamo per\`o, a questo punto, che la ruota azzurra ora
tagliata non potr\`a facilmente generare altri {\em partner}.
La via pi\`u breve per renderci conto di ci\`o consiste nel tentativo
di tagliare una gemella della ruota azzurra
(magari con ugual numero di denti)
utilizzando lei stessa come
creatore.
Purtroppo, trovandosi in questo caso le sporgenze della ruota creatrice
(la azzurra di figura
\ref{fig:belbon}) al di sotto della primitiva,
essa non sar\`a in grado di incidere alcunch\'e sul disco grezzo, lasciandoci
perplessi circa la nostra scelta.
Tuttavia ci accorgiamo che tagliare la {\em partner} della ruota azzurra di
figura \ref{fig:belbon}
tramite lei stessa non \`e il solo metodo che
abbiamo a disposizione.
\begin{figure}[hbt]
\begin{center}
\includegraphics[width=0.72\textwidth]{part2/ruote/FIG/ruote/belbon_interno.pdf}
\end{center}
\begin{picture}(0,0)(0,0)
\scriptsize{
\put(96,95){$\alpha=180^{\circ}$}
\put(175,174){$\beta=90^{\circ}$}
\put(106,62){ruota 1 tagliata}
\put(300,50){ruota 2 creatrice}
}
\end{picture}
\vskip -7mm
      \caption{\em 
Tentativo di dentatura complementare con dentiera a curvatura negativa. La dentatura risulta impossibile.
}
 \label{fig:belbon_interno}
\end{figure}
La ruota rosa (creatrice) di figura \ref{fig:belbon}
possiede un utensile ``duale'' che consiste nella
stessa ruota con curvatura opposta.
Questa inversione, necessaria, della curvatura del creatore non pu\`o
essere chiarita a dovere in questo luogo. Rassicuriamo il lettore che, nel
paragrafo dove si tratta l'ottenimento dei profili coniugati di assortimento,
ci\`o risulter\`a perfettamente illustrato. Qui basta dire che questo
utensile ``duale'', rappresentato in rosa in figura \ref{fig:belbon_interno},
sarebbe perfettamente in grado,
come si intuisce dal suo aspetto, di creare la ruota 
rosa di figura \ref{fig:belbon}.
Ma anche in questo caso, rappresentato appunto
in figura \ref{fig:belbon_interno},
si nota che la ruota creatrice non trova materiale da incidere sullo
stesso disco grezzo dalla quale abbiamo ottenuto la ruota azzurra.
In tale figura viene riportato, con colore tenue, il sovra-metallo necessario
per potere tagliare questa benedetta {\em partner}. Ma \`e chiaro che l'obbligo
di impiegare dischi grezzi
differenti prelude alla necessit\`a di tollerare l'esistenza di
due famiglie di ruote, i membri di una delle quali, probabilmente,
ingraneranno coi membri dell'altra.
Si giunge cos\`i a un terzo precetto (forte) che \`e quello
di potere creare ruote, magari di due generi diversi che chiamiamo
a) e b), in
cui tutte le ruote appartenenti alla famiglia a) possano ingranare con quelle
appartenenti alla famiglia b)\footnote{
Questo requisito minimo per quanto riguarda l'assortibilit\`a delle ruote
dentate viene quasi sempre abbondantemente rispettato, anzi, nelle dentature
ad evolvente, come vedremo, le due famiglie si riducono a una.
}.
L'idea che ci viene in mente per ottemperare a tale precetto, 
quello cio\`e di far s\`i che esistano due famiglie di ruote,
che chiameremo di {\em assortimento}\index{ruote!di assortimento}
 e che siano simmetriche dal punto di vista 
del sovra-metallo che il disco grezzo deve presentare all'esterno
della circonferenza primitiva, consta nel creare sulla dentatrice
un profilo
che stia a cavallo della sua stessa primitiva.
\begin{figure}[hbt]
\begin{center}
\includegraphics[width=0.85\textwidth]{part2/ruote/FIG/ruote/squalo.pdf}
\end{center}
\begin{picture}(0,0)(0,0)
\scriptsize{
\put(290,115){$\alpha=360^{\circ}$}
\put(220,133){$\beta=540^{\circ}$}
\put(85,145){famiglia a)}
\put(85,80){ruota 1 tagliata}
\put(230,50){ruota 2 creatrice}
}
\end{picture}
\vskip -7mm
      \caption{
\em Dentatura ottenuta da sporgenze a ``dente di squalo''. I denti creatori
misurano, sulla primitiva, quanto gli spazi vuoti.
I numeri di denti valgono $z_1=30$ e $z_2=60$.
}
 \label{fig:squalo}
\end{figure}
Il secondo tentativo,
rappresentato in figura \ref{fig:squalo}, mostra una dentatrice a ``denti di 
pescecane'' che taglia la ruota azzurra.
La dentatura \`e in questo caso presentata
completa, sottolineando cos\`i il rispetto del precetto che
impone numeri
di denti interi sulle due ruote, $z_1$ e $z_2$. Anche se confessiamo
 che le stesse figure
\ref{fig:belbon} e \ref{fig:belbon_interno} presenterebbero, sia
sulla ruota creatrice sia su quella tagliata,
un numero intero di denti
qualora tali ruote fossero rappresentate complete:
 non abbiamo avuto il coraggio di scrivere
una procedura che ammettesse le frazioni di dente.
\begin{figure}[hbt]
\begin{center}
\includegraphics[width=\textwidth]{part2/ruote/FIG/ruote/squalo_interno.pdf}
\end{center}
\begin{picture}(0,0)(130,0)
\scriptsize{
\put(385,128){$\alpha=360^{\circ}$}
\put(200,130){$\beta=180^{\circ}$}
\put(346,155){famiglia b)}
\put(346,110){ruota 1 tagliata}
\put(230,47){ruota 2 creatrice}
}
\end{picture}
\vskip -7mm
      \caption{\em Dentatura ottenuta dall'utensile della figura \ref{fig:squalo}
con curvatura opposta.
I numeri di denti valgono $z_1=30$ e $z_2=60$.
}
 \label{fig:squalo_interno}
\end{figure}
In figura \ref{fig:squalo_interno} si accenna alla dentatura che
si otterrebbe per la famiglia b) di ruote complementari alla ruota 
azzurra della famiglia a) di figura \ref{fig:squalo}. Ovviamente, la ruota
appartenente a questa seconda famiglia b),
recante numero di denti pari alla ruota
 dentatrice (rosa)
di figura \ref{fig:squalo}, coinciderebbe con la dentatrice stessa e di
questa famiglia risulterebbe la maggiore\footnote{
Naturalmente, per la famiglia di ruote b) si potrebbero ottenere
ruote pi\`u grandi
della dentatrice di figura \ref{fig:squalo_interno} utilizzando come
utensile una qualsiasi ruota
appartenente alla famiglia a). Non presentiamo in queste note le dentature
interne, ma quello ora esposto rappresenta l'{\em escamotage} che consente
di eseguire dentature interne derivanti dalla dentiera creatrice {\em standard},
che \`e una cremagliera, quindi sicuramente di diametro maggiore rispetto
alla dentatura interna da eseguire.
}.
Ripetiamo che non siamo ancora in grado di valutare  se tutte le ruote della
famiglia a) siano o meno le {\em partner} cinematiche ideali delle ruote b),
ma il nostro fiuto ci convince che  siamo sulla buona strada.
Aggiungiamo, a questi procedimenti ingenui di taglio delle ruote dentate, due
precisazioni.
La prima: il taglio
della maggior parte delle ruote dentate avviene, da un punto di vista
concettuale, proprio nel modo in cui noi le abbiamo tagliate
nell'esperimento mentale, con qualche opportuno accorgimento pratico.
Siccome il materiale da tagliare, che \`e nella gran parte dei casi
acciaio al carbonio, non \`e ``morbido'', il processo di taglio deve avvenire
per asportazione di truciolo. Nei nostri due casi si dovrebbe operare
spostando la ruota creatrice, rosa, fuori dal piano,
per poi accostarla a quella da tagliare; a questo punto
riportarla nel piano, asportando in tal modo il materiale in eccesso
sulla ruota azzurra. Questo procedimento va ripetuto molte volte e a 
piccoli passi, cos\`i da simulare, durante il taglio, l'effettivo
processo di inviluppo continuo che avremmo col rotolamento, l'una
sull'altra, delle due primitive (le circonferenze delle ruote di frizione
riportate in colore rosso e blu). Da qui deriva la seconda precisazione.
Il movimento alternato dell'utensile, che \`e costretto,
uscendo dal piano, ad uscire di scena,
ci consente di avere delle fasi oziose dove siamo
liberi di imporre, alla ruota creatrice, spostamenti non basati sul
rotolamento delle primitive, ma che risultano utili per i nostri fini di taglio.
Infatti, normalmente, una volta guadagnato sulla
ruota creatrice un angolo $\Delta \beta$
corrispondente a $360 /z_2$, si torna indietro con questa 
ruota di tale quantit\`a. Operando in tale maniera, non siamo
costretti ad avere un utensile creatore con un numero di denti elevato,
anzi in genere basta un numero di denti modesto per creare ruote in
un vasto intervallo di numeri di denti. Anche noi ci siamo comportati
in questo modo per disegnare numericamente le nostre figure,
assicurandoci cos\`i di non rendere il codice
che le genera troppo pesante. Inoltre, il semplice movimento di creazione
per inviluppo, che consiste nel rotolamento delle due primitive, pu\`o
essere sostituito da qualsiasi movimento che rispetti tale moto relativo:
nelle nostre figure si \`e scelto di tenere ferma la ruota azzurra e di
movimentare la ruota creatrice di conseguenza. Gli angoli rappresentati
nelle figure \ref{fig:belbon}, \ref{fig:belbon_interno}, \ref{fig:squalo} e
\ref{fig:squalo_interno} si riferiscono esclusivamente al movimento
del creatore, in particolare alla sua rotazione $\beta$ e all'angolo
della porzione di ruota tagliata $\alpha$.
 
\noindent  Non \`e necessaria un'esperienza da meccanico consumato
per constatare  che i denti tagliati sulla ruota 1) di figura \ref{fig:squalo}
sono estremamente scavati alla base.
Notiamo anche che i contatti tra ruota tagliata e creatore e presumibilmente 
tra ruote a) e ruote b) si manifestano sia su superfici (le ruote
dentate possiedono uno spessore) sia su segmenti singolari, come mostrato
in figura \ref{fig:squalo}.
Tali contatti singolari sono in grado, da un punto di vista teorico,
di garantire il corretto e costante rapporto di trasmissione, \index{trasmissione!omocinetica}
che \`e un altro precetto (forte), il quarto.
Detti contatti non sono per\`o idonei a trasferire forze da un elemento
all'altro;
veniamo dunque al quinto precetto (forte):
nella trasmissione del movimento devono necessariamente
essere implicati contatti di strisciamento tra superfici
aventi raggio di curvatura del loro profilo, nel punto di contatto,
 ben diverso da zero. Vedremo
a breve che questo precetto ci porter\`a a considerare utili,
al fine della costruzione delle ruote dentate, esclusivamente opportune
coppie di profili
coniugati, la cui definizione viene rimandata al prossimo paragrafo.

\noindent Altre due intuizioni potrebbero balenarci in mente:
per prima cosa, ci sembra probabile che una riduzione
dell'altezza dei ``denti di squalo'' nella dentatrice possa essere un rimedio
per la forma dei denti
troppo sotto-tagliata alla base della ruota 1); la seconda osservazione
riguarda il raggio della ruota creatrice.
Infatti, qualora utilizzassimo una dentatrice
che rimanesse inalterata
una volta cambiata di segno la
sua curvatura, non avremmo pi\`u due famiglie di
ruote, ma una soltanto, le quali potrebbero (forse) ingranare tutte
quante tra loro e questo, quello cio\`e di ridurre le famiglie di ruote da due
a una, lo diamo come sesto precetto (debole)\footnote{Chi vuole
pu\`o trovare nel capitolo \ref{ruotecy} 
la descrizione di particolari ruote a profilo
cicloidale per le quali si tollera (o si richiede)
la presenza di due famiglie di ruote.}.
Tale simmetria  dell'utensile creatore si chiama
{\em auto-complementarit\`a}\index{auto-complementarit\`a} e
ci conduce  a un creatore a
dentiera rettilinea coi fianchi dei denti pure costituiti da segmenti di retta.
Ma tutte queste intuizioni, che vengono in mente per lo pi\`u a chi le ruote
dentate gi\`a le conosce, 
andrebbero poste in ordine. Per questo siamo costretti
 a toglierci dal nostro comodo sentiero intuitivo
e addentrarci, per poi uscirne il prima possibile, nel pi\`u impervio cammino
che porta alla costruzione dei cosiddetti {\em profili coniugati di
assortimento}\index{profili!di assortimento}.

\section{Costruzione di Profili Coniugati di Assortimento} \label{prof_con}

\noindent Questo paragrafo tratta un problema di cinematica riportato
sulla maggior parte dei
 libri di Meccanica Applicata alle Macchine, ma anche su
alcuni testi di Meccanica Razionale. Prima di 
impostare la definizione dei profili coniugati e di presentare una soluzione
per il loro ottenimento, una volta date le primitive del moto, avvertiamo
il lettore che di tutto questo studio si utilizzeranno soltanto un paio di
conclusioni, legate a due casi molto particolari,
e che tali risultati saranno 
esposti, in riassunto, all'inizio del prossimo paragrafo, al quale rimandiamo
i lettori che non sentono particolare inclinazione per questo argomento.

\noindent Ma cosa sono due {\em profili coniugati}\index{profili!coniugati}?
Due curve regolari qualsiasi possono essere messe tra loro in una
infinit\`a di relazioni reciproche le quali, determinando il loro
posizionamento e il loro movimento relativo,
le rendono tra loro coniugate. 
Come vedremo a breve, ciascuna di queste infinite possibili relazioni, di
natura matematica, che determina il movimento di una curva rispetto all'altra,
deve per\`o possedere alcune caratteristiche (poche) particolari.
Siano $\sigma_1(t)\equiv [\sigma_1(t)_x,\sigma_1(t)_y]^T$ e
$\sigma_2(\tau)\equiv [\sigma_2(\tau)_x,\sigma_2(\tau)_y]^T$,
nel piano cartesiano $(x,y)$, due curve continue e prive di singolarit\`a
in due dati intervalli dei loro parametri, $t$ e $\tau$.
Si scelga ora una funzione, $\tau=f(t)$ definita negli intervalli di interesse,
che mette in corrispondenza biunivoca due intervalli finiti
di coppie di punti appartenenti alle due curve. La relazione tra i due
parametri, $f()$, lega pertanto tra loro i punti
$\sigma_1(t)$ e $\sigma_2(f(t))$ che saranno i candidati
alla coniugazione andando, dopo opportune trasformazioni, a coincidere.
Inoltre, la stessa funzione $f()$ metter\`a naturalmente in relazione anche
le inclinazioni delle tangenti alle curve stesse, che
indicheremo con $\alpha_1(t)$ e $\alpha_2(f(t))$:
anche questi due valori dovranno coincidere nel punto
di coniugazione tra i profili.
Per ora, richiediamo soltanto che $f()$ sia continua e vedremo in seguito
se a tale funzione siano da richiedere o meno altre restrizioni.
Utilizziamo la relazione tra i due parametri per eseguire
alcune {\em trasformazioni piane}\index{trasformazioni piane} di una delle due
curve. In questa sede si \`e deciso ad arbitrio di operare su $\sigma_2$, per
ottenere da tali trasformazioni (continue) la sua coniugata $\sigma_1$.
Procediamo nel modo
seguente; disegnata $\sigma_1(t)$, proponiamo la famiglia di curve
${\sigma_2(\tau)}_t$, che invilupper\`a $\sigma_1$,
mediante la seguente relazione che consta di quattro trasformazioni piane,
le quali legano tra loro le due curve
\begin{equation}
{\sigma_2(\tau)}_t = 
 {\bm T}[\sigma_1(t)]\cdot {\bm R}[\alpha_1(t)]
\cdot{\bm R}[-\alpha_2(f(t))]\cdot
 {\bm T} [-\sigma_2(f(t))]\cdot \sigma_2(\tau)\,.
\label{eq:inviluppo}
\end{equation}

\begin{figure}[hbt]
\begin{center}
\includegraphics[width=0.8\textwidth]{part2/ruote/FIG/ruote/profili_coniugati_definizione_steps.pdf}
\end{center}
\begin{picture}(0,0)(0,0)
\scriptsize{
\put(318,27){$\sigma_1(t)$}
\put(259,84){$\sigma_1(c)$}
\put(209,186){$\sigma_2(\tau)$}
\put(197,182){$a)$}
\put(176,152){$b)$}
\put(215,75){$d)$}
\put(210,95){$c)$}
\put(315,68){$e)$}
\put(192,125){$\sigma_2(f(c))$}
\put(161,68){$0,0$}
\put(159,74){\rotatebox{120}{$\longrightarrow$}}
\put(75,40){$\sigma_2'(\tau)= {\bm T} [-\sigma_2(f(c))]\cdot \sigma_2(\tau)$}
\put(20,156){$\sigma_2''(\tau)={\bm R}[-\alpha_2(f(c))]\cdot \sigma_2'(\tau)$}
\put(132,150){{$\longrightarrow$}}
\put(180,53){$\sigma_2'''(\tau)={\bm R}[\alpha_1(c)]]\cdot \sigma_2''(\tau)$}
\put(215,62){\rotatebox{130}{$\longrightarrow$}}
\put(236,147){$ \sigma_2(\tau)_c={\bm T}[\sigma_1(t)]\cdot \sigma_2'''(\tau)$}
}
\end{picture}
\vskip -7mm
      \caption{\em
Le quattro trasformazioni della formula \ref{eq:inviluppo}.
}
 \label{fig:profili_coniugati_definizione_steps}
\end{figure}
\noindent Le trasformazioni ${\bm T}$ e ${\bm R}$ sono rispettivamente
le traslazioni e le rotazioni nel piano. Esse agiscono sui vettori colonna
dei punti di $\sigma_2$. In particolare, le traslazioni ${\bm T}$ accetteranno
due parametri, dati dai punti $[x,y]^T$ della
curva che contengono come argomento, mentre le rotazioni ${\bm R}$ ne
accetteranno uno solo: tale parametro
\`e l'angolo compreso tra la tangente alla curva, cui l'argomento della
rotazione si riferisce, e una direzione arbitraria e fissa (nelle
nostre figure, quella verticale).
Allo scopo di chiarire la trasformazione operata in \ref{eq:inviluppo}
illustriamo in figura 
\ref{fig:profili_coniugati_definizione_steps} un esempio dove 
la curva $\sigma_1$, riportata in blu, \`e, nella fattispecie, una parabola di
quarto grado rovesciata con vertice nell'origine, mentre la $\sigma_2$, verde,
consiste in una parabola di secondo grado, sempre col vertice nell'origine
e asse verticale.
Infine, la funzione che lega i due parametri \`e semplicemente data
da $\tau=0.3t$. Rimarchiamo che, nonostante tutte le scelte arbitrarie
operate al fine di ottenere la figura
\ref{fig:profili_coniugati_definizione_steps} riflettano una decisa semplicit\`a
(parabole centrate, funzione lineare che lega i due parametri), ci\`o non
inficia la generalit\`a dell'esempio, come vedremo pi\`u avanti, dove lo stesso
codice che genera la figura appena citata viene impiegato per
la generazione di profili coniugati pi\`u complessi. 
Consideriamo la trasformazione \ref{eq:inviluppo} per un valore del parametro
$t=c$.  Tale parametro individua i punti $\sigma_1(c)$ e $\sigma_2(f(c))$,
in rosso nella figura \ref{fig:profili_coniugati_definizione_steps}.
Analizziamo la \ref{eq:inviluppo} partendo da
destra e mettendo in fila le operazioni necessarie per avere le due
curve coniugate nel punto di $\sigma_1(c)\equiv\sigma_2(f(c))$.
Il termine pi\`u a destra, $\sigma_2(\tau)$,
sar\`a semplicemente la curva mobile a), di colore verde, gi\`a citata poc'anzi.
La trasformazione $ {\bm T} [-\sigma_2(f(c))] $ \`e una traslazione che porta il 
punto di $\sigma_2$ destinato ad essere coniugato (col punto ``c-esimo'' di
$\sigma_1$) nell'origine degli assi. Il risultato \`e rappresentato dalla 
curva di colore violetto b) e nome $\sigma_2'(\tau)$.
L'allineamento degli spazi tangenti alle due curve, nei
punti di coniugazione (di parametri $c$ e $f(c)$), avviene in due fasi.
Applichiamo alla $\sigma_2'(\tau)$ una prima rotazione 
di segno contrario alla tangente a $\sigma_2'$ in $\tau=f(c)$, ottenendo in tal
modo $\sigma_2''(\tau)={\bm R}[-\alpha_2(f(c))]\sigma_2'(\tau)$,
parabola c), riportata in colore azzurro chiaro. Applichiamo quindi una seconda
rotazione, pari all'inclinazione della tangente a $\sigma_1$ in $c$, ottenendo
cos\`i $\sigma_2'''(\tau)={\bm R}[\alpha_1(c)]\cdot\sigma_2''(\tau)$,
che \`e la parabola in colore grigio d).
\begin{figure}[hbt]
\begin{center}
\includegraphics[width=0.8\textwidth]{part2/ruote/FIG/ruote/profili_coniugati_definizione_a.pdf}
\end{center}
\begin{picture}(0,0)(0,0)
\scriptsize{
\put(315,42){$\sigma_1$}
\put(206,189){$\sigma_2$}
\put(67,66){$\lambda_1$}
\put(251,176){$\lambda_2$}
}
\end{picture}
\vskip -7mm
      \caption{\em
Profili coniugati: inviluppo e relative polari.
}
 \label{fig:profili_coniugati_definizione_a}
\end{figure}
Finalmente, non rimane
che traslare la curva $\sigma_2'''(\tau)$, 
la quale si trova ora col suo ``c-esimo'' punto in (0,0) e con la tangente in
quel punto inclinata come la tangente a $\sigma_1(c)$,
sul punto di coniugazione con $\sigma_1$, e questo si esegue
mediante la traslazione $\sigma_2(\tau)_c={\bm T}[\sigma_1(c)]\sigma_2'''(\tau)$,
ottenendo in tal modo la parabola e).
Il risultato di queste operazioni, al variare del punto
di coniugazione, cio\`e al variare di $c$,  \`e quello mostrato in figura
\ref{fig:profili_coniugati_definizione_a}. La parabola di colore
rosso, che crea l'inviluppo di $\sigma_1$, viene rappresentata,
onde evitare un eccessivo
appesantimento dell'immagine, soltanto in una ventina di posizioni,
rendendo comunque perfettamente l'idea del processo di inviluppo.
Da un punto di vista matematico, le condizioni necessarie e sufficienti
affinch\'e la $\sigma_2$, durante il suo movimento, inviluppi la curva
$\sigma_1$ sono due, e precisamente \`e richiesto che i punti omologhi
delle due curve coincidano per tutti i valori omologhi dei due parametri (nei
due intervalli di interesse) e che, in tali punti, anche le tangenti
alle due curve siano la stessa retta\footnote
{
L'inviluppo di una curva, espressa
come $\phi(x,y,\lambda)=0$ dove $\lambda$ \`e il parametro che la muove nel
piano, si ottiene eliminando il parametro stesso
dal sistema
\begin{equation}
\begin{cases}
\phi(x,y,\lambda)=0 \\
\displaystyle \frac{\partial \phi(x,y,\lambda)} {\partial\lambda}=0\,,
\end{cases}
\nonumber
\end{equation}

\noindent il che \`e del tutto equivalente a quanto da noi richiesto.
}.

\vskip .2cm
\footnotesize
\noindent Si intuisce facilmente che entrambe queste condizioni sono assicurate
dal procedimento descritto dalla \ref{eq:inviluppo} e illustrato in figura
\ref{fig:profili_coniugati_definizione_a} ma, rischiando di passare per
pedanti, ne riportiamo comunque la dimostrazione.
Quanto alla prima condizione, cio\`e la prescrizione che le due curve
$\sigma_2$ e l'inviluppo creato dalla \ref{eq:inviluppo}, coincidano,
essa deriva con semplicit\`a da ci\`o che
abbiamo discusso circa la  struttura della \ref{eq:inviluppo} stessa.
Considerando infatti un valore del
parametro $t=c$ \hspace{.1 cm} e, di conseguenza, $\tau=f(c)$, avremo
\begin{equation}
{\sigma_2(f(c))}_c = 
 {\bm T}[\sigma_1(c)]\cdot {\bm R}[\alpha_1(c)]\cdot{\bm R}[-\alpha_2(f(c))]\cdot
 {\bm T} [-\sigma_2(f(c))]\cdot \sigma_2(f(c))\,.
\label{eq:dimost_x}
\end{equation}
\noindent Ma considerando la prima (si parte sempre da destra) delle
trasformazioni della \ref{eq:dimost_x} abbiamo
\begin{equation}
{\bm T} [-\sigma_2(f(c))]\cdot \sigma_2(f(c))=(0,0)^T.
\label{eq:dimost_x1}
\end{equation}
\noindent Pertanto, tenendo presente che le rotazioni non hanno effetto sul vettore
nullo, avremo
\begin{equation}
{\sigma_2(f(c))}_t = 
 {\bm T}[\sigma_1(c)]\cdot {\bm R}[\alpha_1(c)]\cdot{\bm R}[-\alpha_2(f(c))]\cdot
 \left(\begin{matrix} 0 \\ 0 \\ \end{matrix}\right)= 
 {\bm T}[\sigma_1(c)]\cdot 
 \left(\begin{matrix} 0 \\ 0 \\ \end{matrix}\right)= 
{\sigma_1(c)}\,. 
\label{eq:dimost_p}
\end{equation}
\noindent Poco pi\`u laboriosa \`e la dimostrazione
della coincidenza delle tangenti alle due curve nel punto di coniugazione.
Analizziamo, a tale proposito, in che modo la \ref{eq:dimost_x}
trasforma il punto infinitamente vicino a quello di parametro $c$
della curva $\sigma_2$. Tale punto risulta individuato dalla seguente espressione
differenziale
\begin{equation}
\sigma_2(f(c)+{\rm d}\tau))=\sigma_2(f(c))+
{{\rm d}{\sigma_2}\over{{\rm d}\tau}}\bigr|_{(f(c))} 
{\rm d} \tau \,, 
\label{eq:dimost_t1}
\end{equation}
\noindent dove il termine ${{\rm d}{\sigma_2}\over{{\rm d}\tau}}\bigr|_{(f(c))}
{\rm d} \tau$ rappresenta un vettore infinitesimo
diretto come la tangente alla $\sigma_2$ nel punto $\sigma_2(f(c))$.
Chiamiamo con $\bm \delta_c$ il vettore infinitesimo 
\begin{equation}
{\bm \delta_c}= \sigma_2(f(c)+{\rm d}\tau)_c- \sigma_2(f(c))_c\,,
\label{eq:vett_inf}
\end{equation}
\noindent intendendo
con i due addendi a destra del segno di uguaglianza l'applicazione della
\ref{eq:dimost_x} sia al punto $\sigma_2(f(c)+{\rm d}\tau))$, sia
al punto $\sigma_2(f(c))$. Data la linearit\`a degli operatori
presenti nella
\ref{eq:dimost_x} vale, per gli operandi, la propriet\`a distributiva
\begin{equation}
{\bm T}()\cdot{\bm R}()\cdot {\bm R}()\cdot{\bm T}()\cdot({\bm a}+{\bm b})=
{\bm T}()\cdot{\bm R}()\cdot {\bm R}()\cdot{\bm T}()\cdot{\bm a}+
{\bm T}()\cdot{\bm R}()\cdot {\bm R}()\cdot{\bm T}()\cdot{\bm b}\,.
\end{equation}

\noindent Ci\`o che rimane dall'applicazione della \ref{eq:dimost_x}
risulta quindi essere
\begin{equation}
{\bm \delta_c} =
 {\bm T}[\sigma_1(c)]\cdot {\bm R}[\alpha_1(c)]\cdot{\bm R}[-\alpha_2(f(c))]
\cdot 
 {\bm T} [-\sigma_2(f(c))]
	\cdot (
{{\rm d}{\sigma_2}\over{{\rm d}\tau}}\bigr|_{(f(c))} 
{\rm d} \tau )\,.
\label{eq:dimost_x3}
\end{equation}
\noindent Sarebbe tedioso ripercorrere per gradi le operazioni
illustrate in figura \ref{fig:profili_coniugati_definizione_steps}, pertanto ci
limitiamo qui a seguire velocemente le trasformazioni del vettore tangente infinitesimo
${{\rm d}{\sigma_2}\over{{\rm d}\tau}}\bigr|_{(f(c))}
{\rm d} \tau$. Ricordando che si parte sempre da destra, esso si muove, mediante
il primo operatore di traslazione,
nell'origine degli assi. Quindi, tramite le due rotazioni, tale vettore
si dispone come la
tangente alla curva $\sigma_1$ nel punto $t=c$. Infine,
mediante l'ultima traslazione $\bm \delta_c$, esso si ritrover\`a
proprio in tale punto
di $\sigma_1$. Risulta cos\`i provato che anche i vettori tangenti alla
$\sigma_2$ diventano, dopo il procedimento di coniugazione, altrettanti
vettori tangenti alla curva $\sigma_1$.



\vskip .2cm
\normalsize
\noindent Abbiamo gi\`a ammesso che le due dimostrazioni, test\'e sviluppate e
stampate con tipi minori, si
presentano come leggermente superflue, essendo il procedimento contenuto
nella \ref{eq:inviluppo}, ed esposto graficamente in
figura \ref{fig:profili_coniugati_definizione_steps}, intrinsecamente volto
ad ottenere la coniugazione dei profili. Tali dimostrazioni ci sono per\`o
utili per meglio circoscrivere i vincoli matematici ai quali $\sigma_1$,
$\sigma_2$ e $f()$ devono sottostare. Affinch\'e si possa manifestare
la coincidenza nei punti di coniugazione, il che equivale alla possibilit\`a
di scrivere la \ref{eq:dimost_x}, la funzione $\tau=f(t)$, che lega i
parametri delle due curve, deve essere definita per tutti i valori di $t$
nell'intervallo di
interesse. Solo questo? S\`i, soltanto questo. Ad essa non \`e chiesto n\'e
di essere monotona n\'e di sottostare
a restrizioni sulle sue derivate, anzi non \`e richiesta neppure
la sua derivabilit\`a. Deve essere chiaro per\`o che ragionando
in questo modo, cio\`e ammettendo che $f()$ possa manifestare qualsiasi
bizzarria, \`e probabile che non si otterranno profili coniugati con
caratteristiche adatte all'impiego nella
meccanica delle macchine. Per quanto riguarda la possibilit\`a
di scrivere la \ref{eq:dimost_t1}, che costituisce la base della seconda 
dimostrazione, circa la coincidenza delle tangenti ai profili, \`e richiesto
che $\sigma_2$, e di conseguenza $\sigma_1$, siano differenziabili, e nulla pi\`u.
Certo, si intuisce con facilit\`a
che aggiungendo altre restrizioni sia ai profili (immaginandoli
ad esempio curve regolari e lisce), sia
alla funzione che lega tra loro i due parametri (che si potrebbe ipotizzare
continua e monotona), si intuisce, dicevamo, che le speranze di ottenere
profili coniugati lisci, privi di singolari\`a e interferenze 
sarebbe superiore.
Ma, come vedremo, tutte queste apprensioni circa la bont\`a dei profili e della
relazione cinematica che li fa scorrere l'uno sull'altro sono fuori luogo nella
meccanica applicata, dove saremo costretti ad ammettere
che i (il?) soli profili interessanti sono anche estremamente semplici.  
Una volta nota, tramite la \ref{eq:inviluppo}, la traiettoria
di due punti qualsiasi appartenenti alla curva $\sigma_2$, chiamiamoli
$\sigma_2(\tau_1)$ e $\sigma_2(\tau_2)$, mentre essa inviluppa $\sigma_1$,
possiamo individuare, al variare del parametro $t$,
le normali a tali traiettorie. Per ciascun valore di $t$,
dall'intersezione di dette normali, otterremo
il centro istantaneo di rotazione di $\sigma_2$. Il luogo geometrico
delle tracce dei centri istantanei di rotazione di $\sigma_2$
si chiama {\em polare fissa}\index{polare!fissa} e verr\`a indicata con $\lambda_1$.
La {\em polare mobile}\index{polare!mobile}, $\lambda_2$,
si ottiene invece lasciando che, una volta individuato il
centro istantaneo di rotazione per un dato valore di
$t$, tale punto venga trascinato
dal piano mobile. \`E sottinteso che, come sempre accade coi moti relativi,
le parti possono essere 
invertite tenendo ferma $\sigma_2$ e facendo muovere $\sigma_1$. Durante tale moto
le polari verranno a scambiarsi tra loro ottenendo, in questo caso, la 
precedente polare mobile come attuale polare fissa.
Polare fissa e polare mobile sono a contatto
tra loro nel centro istantaneo di rotazione corrispondente ad un 
dato valore del parametro $t$ e il loro movimento relativo \`e di puro rotolamento.
In figura \ref{fig:profili_coniugati_definizione_a} sono rappresentate
le polari del moto di $\sigma_2$, fissa e mobile, mediante un centinaio
di punti scelti dal migliaio a nostra disposizione tramite
il codice che genera la stessa figura
\ref{fig:profili_coniugati_definizione_a} e le altre
\ref{fig:profili_coniugati_definizione_b} e
\ref{fig:profili_coniugati_definizione_b1} e che,
qualora fossero tutti disegnati, avrebbero reso continue tali curve.
La rappresentazione
di un minor numero di punti consente di rendere l'idea,
tramite le distanze reciproche tra i punti stessi,
della equivalenza delle lunghezze di tratti
omologhi delle due polari, equivalenza implicata dal rotolamento.
In generale le polari sono curve poco intuitive e possono presentare
discontinuit\`a e contenere punti impropri.

\begin{figure}[hbt]
\centering
\begin{minipage}[b]{0.45\textwidth}
\centering
\includegraphics[width=0.9\textwidth]{part2/ruote/FIG/ruote/profili_coniugati_definizione_b.pdf}
\begin{picture}(0,0)(130,0)
\scriptsize{
\put(123,33){$\sigma_1$}
\put(46,83){$\sigma_2$}
\put(-3,26){$\lambda_1$}
\put(114,68){$\lambda_2$}
}
\end{picture}
      \caption{\em
Curva $\sigma_2$ in scala ridotta e nuove polari.
}
 \label{fig:profili_coniugati_definizione_b}
\end{minipage}\hfill
\begin{minipage}[b]{0.45\textwidth}
\includegraphics[width=0.9\textwidth]{part2/ruote/FIG/ruote/profili_coniugati_definizione_b1.pdf}
\begin{picture}(0,0)(130,0)
\scriptsize{
\put(124,5){$\sigma_1$}
\put(30,23){$\sigma_2$}
\put(-19,8){\rotatebox{-90}{$\longrightarrow$}}
\put(-13,-3){polare fissa e polare mobile all'infinito}
}
\end{picture}
      \caption{\em
Profili coniugati traslanti.
      }
 \label{fig:profili_coniugati_definizione_b1}
\end{minipage}
\end{figure}


\noindent Una situazione non molto dissimile da quella di figura 
\ref{fig:profili_coniugati_definizione_a} \`e quella rappresentata in
\ref{fig:profili_coniugati_definizione_b}, dove la curva $\sigma_2$ \`e
riportata in scala minore e $\tau=0.07t$, allo scopo di mostrare quale
cambiamento macroscopico subiscono le due polari del moto a fronte di un
cambiamento, apparentemente leggero, di uno dei due profili coniugati.
In figura \ref{fig:profili_coniugati_definizione_b1} viene
invece mostrata
una curva $\sigma_2$ che inviluppa, tramite un moto esclusivamente 
traslatorio, la curva $\sigma_1$, sua simile ma in scala maggiore. Per questo
esempio \`e stata infatti scelta una funzione $f()$, che lega i due parametri,
tale da portare a coniugazione i punti delle due curve aventi, a priori,
 la stessa
tangente. \`E proprio questo il motivo che rende traslatorio il moto di
inviluppo e che manda le polari del moto all'infinito rendendole
due segmenti di rette improprie la cui direzione coincide con quella di
spostamento del profilo  $\sigma_2$. I profili coniugati che abbiamo finora analizzato sono
stati ottenuti tramite curve $\sigma_2(\tau)$ molto semplici,
in pratica delle parabole.  Altrettanto semplici sono
i legami $\tau=f(t)$ che definiscono, tramite la $\sigma_1(t)$,
le leggi degli spostamenti di $\sigma_2$.  In questo modo,
gli inviluppi ottenuti somigliano a oggetti effettivamente
utilizzabili nella meccanica delle macchine.
\begin{figure}[hbt]
\begin{center}
\includegraphics[width=0.8\textwidth]{part2/ruote/FIG/ruote/profili_coniugati_definizione_c.pdf}
\end{center}
\begin{picture}(0,0)(0,0)
\scriptsize{
\put(292,123){$\sigma_1$}
\put(198,87){$\sigma_2$}
\put(26,228){$\lambda_1$}
\put(62,92){$\lambda_1$}
\put(269,98){$\lambda_1$}
\put(103,313){$\lambda_2$}
\put(201,33){$\lambda_2$}
\put(129,50){$\lambda_2$}
}
\end{picture}
\vskip -5mm
      \caption{\em
Profili coniugati che si intersecano e polari discontinue.
}

 \label{fig:profili_coniugati_definizione_c}
\end{figure}
Come abbiamo per\`o ripetuto altre volte, la teoria dei profili coniugati
non mette restrizioni, se non
molto blande, alle tre entit\`a matematiche test\'e menzionate.
Per esempio, le curvature di $\sigma_1$ e  $\sigma_2$ possono essere
tali che, combinate alla funzione $f()$, producano inviluppi che 
si intersecano con le curve che li generano. In questi casi,
le polari del moto presentano spesso discontinuit\`a e punti
all'infinito,
come mostra la figura \ref{fig:profili_coniugati_definizione_c}.
Nel caso riportato in figura,
le curve $\sigma_1$ e $\sigma_2$ sono ancora parabole,
quindi curve semplici e regolari, e la funzione che lega i parametri delle due
curve \`e ancora $\tau=0.3t$. Ma le curvature dei due profili
risultano tra loro ``incompatibili''. Ben inteso, da un punto di
vista puramente matematico tutto fila liscio come sempre.
Nella tecnica, per\`o, tali profili non possono essere utilizzati
per trasmettere il  movimento o per altre funzioni concrete
quindi, per noi ingegneri,
rimangono curiosit\`a che appartengono alla teoria.
Nella figura
si possono anche vedere tutti gli spezzoni delle due polari per le quali
si intravedono gli asintoti che indicano le direzioni
dei loro punti impropri.
Capovolgiamo il problema. Supponiamo di conoscere le polari
del moto relativo, e investighiamo la possibilit\`a
di trovare due profili tra loro coniugati. Dettando le polari
la legge geometrica del moto relativo
del  piano mobile rispetto a quello fisso, ad ogni curva appartenente
al primo corrisponder\`a il relativo inviluppo sul secondo,
il quale, del profilo mobile, sar\`a anche il profilo coniugato.
\begin{wrapfigure}{r}{0.55\textwidth}
      \begin{center}
      \includegraphics[width=0.50\textwidth]{part2/ruote/FIG/ruote/profili_coniugati_invil_polar.pdf}
     \end{center}
\begin{picture}(0,0)(0,0)
\scriptsize{
\put(99.6,65){$\eta_0$}
\put(94.2,71.3){\rotatebox{120}{$\longrightarrow$}}
\put(60,100){$\lambda_2$}
\put(43,45){$\lambda_1$}
\put(170,145){$\sigma_1$}
\put(78,135){$\sigma_2$}
\put(-4,163){$\sigma_1(f(\eta_0))\equiv\sigma_2(g(\eta_0))$}
\put(36,162){\rotatebox{-64}{$\longrightarrow$}}
}
\end{picture}
\vskip -6.3mm
        \caption{\em
Profilo $\sigma_1$ coniugato a $\sigma_2$ ottenuto dall'inviluppo di $\sigma_2$,
trascinata dalla polare mobile $\lambda_2$ che rotola su $\lambda_1$.
}
     \label{fig:profili_coniugati_invil_polar}
\end{wrapfigure}

\noindent Notiamo infatti che le trasformazioni contenute nella
\ref{eq:inviluppo} possono essere riferite a qualsiasi altra coppia di
profili coniugati compatibili con il moto relativo
esplicitato dalla stessa formula. Tra queste coppie di profili
coniugati vi \`e, come caso particolare, quella formata alle polari,
per le quali la legge che lega i due parametri \`e semplicemente
$\tau=t$.
Indicando cos\`i le polari del moto con $\lambda_1$ e $\lambda_2$,
figura \ref{fig:profili_coniugati_invil_polar},
la relazione \ref{eq:inviluppo} \`e perfettamente equivalente a
\begin{equation}
{\sigma_2(\tau)}_\eta = 
 {\bm T}[\lambda_1(\eta)]\cdot {\bm R}[\beta_1(\eta)]\cdot{\bm R}[-\beta_2(\eta)]\cdot
 {\bm T} [-\lambda_2(\eta)]\cdot \sigma_2(\tau)\,,
\label{eq:inviluppo_polar}
\end{equation}

\noindent con ovvio significato degli angoli $\beta_1$ e $\beta_2$.
L'espressione \ref{eq:inviluppo_polar} indica la famiglia di 
curve $\sigma_2$ rappresentata in figura \ref{fig:profili_coniugati_invil_polar}
in colore violetto.
Basta, in questo caso, un solo parametro $\eta$ a determinare in quale
dei loro punti le polari sono a contatto, in quanto
esse rotolano l'una sull'altra senza strisciare.
Riteniamo superfluo dimostrare che la famiglia di curve 
${\sigma_2(\tau)}_\eta$ inviluppa il profilo $\sigma_1$, che nella figura
\ref{fig:profili_coniugati_invil_polar} non viene tracciato in modo esplicito,
 in quanto dovremmo nuovamente
ripetere la dimostrazione svolta tramite le formule da \ref{eq:dimost_x} fino a
\ref{eq:dimost_x3} e i relativi commenti.
I parametri di $\sigma_1(t)$ e di $\sigma_2(\tau)$ saranno legati al parametro
$\eta$ da due relazioni imposte dal processo di inviluppo, $t=f(\eta)$ e
$\tau=g(\eta)$, le quali naturalmente richiederanno che per qualunque
valore del parametro $\eta$ si abbia $\sigma_1(f(\eta))=\sigma_2(g(\eta))$.
Nella figura \ref{fig:profili_coniugati_invil_polar} i due profili $\sigma_1$ e $\sigma_2$
si toccano in $\sigma_1(f(\eta_0))=\sigma_2(g(\eta_0))$, e, ad arbitrio,
abbiamo imposto $\lambda_1(\eta_0)=\lambda_2(\eta_0)=(0,0)^T$, cio\`e che 
le due polari si tocchino, in quel caso,
nell'origine delle coordinate cartesiane del piano fisso.
Le trasformazioni contenute nella \ref{eq:inviluppo_polar} dovrebbero
ormai risultare
famigliari; partendo, come le altre volte,  da destra: 
imponiamo una traslazione del punto della polare $\lambda_2$
di parametro $\eta$ in $(0,0)$, segue una rotazione della stessa di un
angolo pari all'opposto dell'inclinazione della $\lambda_2$ in $\eta$,
$-\beta_2(\eta)$, ancora una rotazione pari all'inclinazione
della $\lambda_1$, sempre nel punto di
parametro $\eta$, $\beta_1(\eta)$. Infine, la traslazione 
nel punto di $\lambda_1(\eta)$ completer\`a la trasformazione;
rimandiamo il lettore alla figura
\ref{fig:profili_coniugati_definizione_steps} per maggior chiarezza.
Rimane sottinteso che, in tutti questi spostamenti e rotazioni, la polare
mobile trascina con s\'e il piano mobile e quindi $\sigma_2$.
\begin{figure}[hbt]
\begin{center}
\includegraphics[width=0.8\textwidth]{part2/ruote/FIG/ruote/profili_coniugati_invil.pdf}
\end{center}
\begin{picture}(0,0)(0,0)
\scriptsize{
\put(177,110){$\eta_0$}
\put(172,116){\rotatebox{120}{$\longrightarrow$}}
\put(329,210){$\sigma_a$}
\put(312,270){$\sigma_b$}
\put(144,233){$\sigma_2$}
\put(78,64){$\lambda_{1a}$}
\put(49,119){$\lambda_{1b}$}
\put(133,160){$\lambda_2$}
}
\end{picture}
\vskip -6mm
      \caption{\em
Profili coniugati inviluppati dalla curva $\sigma_2$ trascinata dalla
polare $\lambda_2$ che rotola sia su $\lambda_{1a}$, sia su $\lambda_{1b}$.
}
 \label{fig:profili_coniugati_invil}
\end{figure}
\noindent Un caso di grande interesse, anzi, il caso che giustifica ampiamente
il procedere, che ammettiamo essere piuttosto laborioso,
di questo paragrafo, \`e quello in cui la
polare $\lambda_2$ rotola su due diverse polari:
identifichiamole coi nomi $\lambda_{1a}$ e $\lambda_{1b}$ e coi
colori verde e blu, come rappresentato
in figura \ref{fig:profili_coniugati_invil}. Anticipiamo, tanta \`e l'importanza
che attribuiamo a questo passo, che una volta ottenuti i due inviluppi, mediante
il rotolamento di $\lambda_2$ su $\lambda_{1a}$ e $\lambda_{1b}$,
sar\`a $\lambda_{1a}$ a rotolare su $\lambda_{1b}$, permettendoci cos\`i di
osservare che i due inviluppi,
precedentemente ottenuti, sono coniugati tra loro.
Le relazioni che descrivono le famiglie delle $\sigma_2$
che inviluppano $\sigma_a$ e $\sigma_b$, seguendo \ref{eq:inviluppo_polar},
saranno date dalla seguente coppia di relazioni
\begin{equation}
{\sigma_{2a}(\tau)}_\eta = 
 {\bm T}[\lambda_{1a}(\eta)]\cdot {\bm R}[\beta_{1a}(\eta)]\cdot{\bm R}[-\beta_2(\eta)]\cdot
 {\bm T} [-\lambda_2(\eta)]\cdot \sigma_2(\tau)\,,
\label{eq:inviluppo_polara}
\end{equation}
\begin{equation}
{\sigma_{2b}(\tau)}_\eta = 
 {\bm T}[\lambda_{1b}(\eta)]\cdot {\bm R}[\beta_{1b}(\eta)]\cdot{\bm R}[-\beta_2(\eta)]\cdot
 {\bm T} [-\lambda_2(\eta)]\cdot \sigma_2(\tau)\,.
\label{eq:inviluppo_polarb}
\end{equation}
\noindent Come premesso, procederemo ora in questo modo: anzich\'e fare rotolare 
la $\lambda_2$ sulle altre due primitive, facciamo rotolare la polare
$\lambda_{1a}$ sulla $\lambda_{1b}$. Il rotolamento di $\lambda_{1a}$
trasciner\`a naturalmente le
$\sigma_{2a}(\tau)_\eta$, cio\`e le curve che stanno a primo membro nella
\ref{eq:inviluppo_polara} e che definiscono l'inviluppo di $\sigma_a$.
Desideriamo dimostrare che tali curve, trascinate
dal movimento di $\lambda_{1a}$, inviluppano anch'esse la $\sigma_{b}$,
ovvero che la $\sigma_a$ inviluppa la $\sigma_b$ durante il rotolamento di 
$\lambda_{1a}$ su $\lambda_{1b}$.
Tale inviluppo sar\`a indicato con ${{\sigma_{2ab}(\tau)}_\eta}_\mu$
e, non avendo direttamente a disposizione l'espressione 
di $\sigma_a$, utilizzeremo la famiglia di curve che ne definisce l'inviluppo,
esplicitate dalla \ref{eq:inviluppo_polara}, ottenendo
\begin{multline}
{{\sigma_{2ab}(\tau)}_\eta}_\mu = 
 {\bm T}[\lambda_{1b}(\mu)]\cdot {\bm R}[\beta_{1b}(\mu)]\cdot{\bm R}[-\beta_{1a}(\mu)]\cdot
 {\bm T} [-\lambda_{1a}(\mu)]\cdot {\sigma_{2a}(\tau)}_\eta = \\
 = 
 {\bm T}[\lambda_{1b}(\mu)]\cdot\color{red} {\bm R}[\beta_{1b}(\mu)]\cdot{\bm R}[-\beta_{1a}(\mu)]\cdot
 {\bm T} [-\lambda_{1a}(\mu)]\cdot \\
\color{red} {\bm T}[\lambda_{1a}(\eta)]\cdot {\bm R}[\beta_{1a}(\eta)]\cdot{\bm R}[-\beta_2(\eta)] \color{black}\cdot
 {\bm T} [-\lambda_2(\eta)]\cdot \sigma_2(\tau)\,.
\label{eq:inviluppo_polarb1}
\end{multline}
\noindent Ricordando che le polari rotolano l'una sull'altra senza strisciare,
selezioniamo da questa doppia famiglia di curve quella di parametro
$\mu=\eta$ e componiamo le trasformazioni, scritte in colore rosso,
contenute nella \ref{eq:inviluppo_polarb1} nel seguente modo
\begin{multline}
\color{red} {\bm R}[\beta_{1b}(\eta)]\cdot{\bm R}[-\beta_{1a}(\eta)]\cdot
 {\bm T} [-\lambda_{1a}(\eta)]\cdot 
 {\bm T}[\lambda_{1a}(\eta)]\cdot {\bm R}[\beta_{1a}(\eta)]\cdot{\bm R}[-\beta_2(\eta)]=\\
={\bm R}[\beta_{1b}(\eta)]\cdot{\bm R}[-\beta_2(\eta)]\,, 
\label{eq:id}
\end{multline}
\noindent da cui si ricava immediatamente
\begin{equation}
{{\sigma_{2ab}(\tau)}_\eta}_\eta = 
 {\bm T}[\lambda_{1b}(\eta)]\cdot {\bm R}[\beta_{1b}(\eta)]\cdot
{\bm R}[-\beta_2(\eta)] \cdot
 {\bm T} [-\lambda_2(\eta)]\cdot \sigma_2(\tau)\,.
\label{eq:inviluppo_polarb2}
\end{equation}
\noindent Le espressioni del secondo membro delle \ref{eq:inviluppo_polarb2}
e \ref{eq:inviluppo_polarb} si equivalgono, pertanto
la famiglia di curve ora ottenuta coincide con quella
che si trova facendo rotolare la $\lambda_2$ sulla $\lambda_{1b}$:
${{\sigma_{2ab}(\tau)}_\eta}_\eta\equiv {\sigma_{2b}(\tau)}_\eta$,
come si evince anche dalla figura
\ref{fig:profili_coniugati_invil_lagrange}, che
rappresenta la situazione in cui la polare $\lambda_{1a}$ rotola sulla
$\lambda_{1b}$ toccandola nel punto
$\lambda_{1a}(\eta_c)\equiv \lambda_{1b}(\eta_c)$\footnote
{
Tutte le figure che rappresentano inviluppi sono state ottenute numericamente
mediante l'applicazione diretta, nel codice, delle relative formule.  Ad esempio,
la curva $\sigma_a$ di figura \ref{fig:profili_coniugati_invil_lagrange} \`e stata
ottenuta dalla implementazione numerica della relazione
\ref{eq:inviluppo_polarb2}. L'aspetto diverso di tale figura
rispetto alla precedente
\ref{fig:profili_coniugati_invil_polar}, dove sono rappresentate tutte le curve
inviluppanti, dipende dal seguente stratagemma.
Oltre alle curve che creano l'inviluppo, disegnate col colore opportuno,
viene tracciato un secondo insieme di curve, con lo stesso procedimento, in
scala leggermente minore e di
colore bianco, cio\`e quello del foglio. In tal modo si ottiene, visivamente,
il solo inviluppo. Questo artificio potrebbe essere percepito da un occhio esperto.}.
\begin{figure}[hbt]
\begin{center}
\includegraphics[width=0.8\textwidth]{part2/ruote/FIG/ruote/profili_coniugati_invil_lagrange.pdf}
\end{center}
\begin{picture}(0,0)(0,0)
\scriptsize{
\put(172,145){$\eta_0$}
\put(166,143){\rotatebox{-120}{$\longrightarrow$}}
\put(203.5,107.4){$\eta_c$}
\put(198,114){\rotatebox{120}{$\longrightarrow$}}
\put(317,242){$\sigma_a$}
\put(303,264){$\sigma_b$}
\put(142,227){$\sigma_2$}
\put(97,36){$\lambda_{1a}$}
\put(63,124){$\lambda_{1b}$}
\put(129,160){$\lambda_2$}
}
\end{picture}
\vskip -6mm
      \caption{\em
I profili $\sigma_a$ e $\sigma_b$ risultano a loro volta coniugati quando la
polare $\lambda_{1a}$ rotola sulla $\lambda_{1b}$ toccandola in $\mu_c=\eta_c$.
}
 \label{fig:profili_coniugati_invil_lagrange}
\end{figure}
\noindent Nella stessa figura sono riportate, con colori meno marcati,
anche la $\sigma_2(\tau_i)$ e la
polare $\lambda_2$ che rimangono al  proprio posto.
\noindent Possiamo quindi affermare che, dati un insieme di polari
$\lambda_i$ e un profilo $\sigma_0$
collegato ad una di queste, diciamo $\lambda_0$,
i profili coniugati $\sigma_{i}$,
che otterremo dagli inviluppi generati da $\sigma_0$ quando $\lambda_0$
rotola sull'insieme delle
$\lambda_i$\footnote{Anche se quanto affermiamo non presenta
ricadute pratiche, possiamo lasciare,
nell'insieme di polari sulle quali rotola $\lambda_0$, anche essa stessa.
 Cos\`i operando $\sigma_0$ invilupper\`a s\'e stessa.}, 
risultano tutti coniugati tra loro a coppie rispetto al moto
generato dal rotolamento delle due polari associate a ciascuna di tali
coppie.  Questo importantissimo risultato costituisce lo scopo del
presente paragrafo.
Da esso discende la possibili\`a di creare {\em profili coniugati di 
assortimento}\index{profili!di assortimento}, questione di estrema
importanza nella fabbricazione di famiglie di ruote dentate che,
a primitive date, possano ingranare l'una con l'altra.


\noindent Prendiamo di nuovo in considerazione
la trasformazione \ref{eq:inviluppo} e la figura
\ref{fig:profili_coniugati_definizione_steps}.
Se omettessimo le ultime due delle quattro trasformazioni presenti
(si parte sempre da destra), cio\`e se ci fermassimo alla parabola c),
rappresentata in azzurro chiaro,
$\sigma''_2(\tau)$,
potremmo ridisegnare la stessa figura applicando
alla $\sigma_1$, con segno opposto, le
trasformazioni non effettuate sulla $\sigma_2$. Ovviamente 
la nuova figura risulterebbe traslata verso sinistra della 
quantit\`a ${\bm T}[-\sigma_1(t)]$ e la $\sigma_1$ sarebbe
ruotata di ${\bm R}[-\alpha_1(t)]$, ma la coniugazione
tra i due profili risulterebbe perfettamente equivalente.
Pi\`u in generale, riconsiderando la famiglia di polari $\lambda_i$
e i relativi profili coniugati $\sigma_i$ possiamo scrivere,
per un dato parametro $\eta_c$,
\begin{equation}
{\sigma_i''(\tau)}_{\eta_c} = 
 {\bm R}[-\beta_i(\eta_c)]\cdot
 {\bm T} [-\lambda_i(\eta_c)]\cdot \sigma_i(\tau)\,,
\label{eq:inviluppo_polar_euler}
\end{equation}

\begin{figure}[hbt]
\begin{center}
\includegraphics[width=0.8\textwidth]{part2/ruote/FIG/ruote/profili_coniugati_invil_euler.pdf}
\end{center}
\begin{picture}(0,0)(0,0)
\scriptsize{
\put(180,153){$\eta_0$}
\put(173,151){\rotatebox{-120}{$\longrightarrow$}}
\put(207,107.4){$\eta_c$}
\put(202,114){\rotatebox{120}{$\longrightarrow$}}
\put(318,244){$\sigma_a$}
\put(305,264){$\sigma_b$}
\put(222,223){$\sigma_2$}
\put(100,35){$\lambda_{1a}$}
\put(67,122){$\lambda_{1b}$}
\put(151,170){$\lambda_2$}
}
\end{picture}
\vskip -6mm
      \caption{\em
Le primitive del moto  rotolano l'una sull'altra in modo tale da mantenere
 fisso, rispetto all'osservatore assoluto, il loro punto di contatto.
}
 \label{fig:profili_coniugati_invil_euler}
\end{figure}

\noindent dove si riconosce una generalizzazione della
\ref{eq:inviluppo_polar} priva per\`o
delle ultime due trasformazioni a sinistra.
Il vantaggio offerto da questo punto di vista \`e quello di trattare
tutte le polari e i relativi profili coniugati mediante trasformazioni 
aventi la medesima struttura, le quali contengono
soltanto informazioni circa quella stessa polare o profilo coniugato.
Agendo in questo modo, il punto di contatto tra tutte le polari rimane fisso
rispetto a un osservatore assoluto e perde di senso parlare di
polari fisse e mobili. 
In figura \ref{fig:profili_coniugati_invil_euler} sono riportati
i profili coniugati e le rispettive polari, coincidenti
nei punti di parametro $\eta_c$ tramite il punto di vista
test\'e descritto e rappresentato dalla relazione \ref{eq:inviluppo_polar_euler}.
Il risultato vede ancora i profili tra loro coniugati (tutti e tre),
mentre il punto di contatto tra tutte le polari, come abbiamo
anticipato, rimane inalterato
rispetto a un osservatore assoluto. Le tre polari si muovono
in modo tale da evocare il funzionamento di un laminatoio.
In questa rappresentazione del moto relativo le polari cambiano nome
e diventano le {\em primitive del moto}\index{primitiva!del moto}.
\begin{figure}[hbt]
\begin{center}
\includegraphics[width=0.8\textwidth]{part2/ruote/FIG/ruote/profili_coniugati_punto.pdf}
\end{center}
\begin{picture}(0,0)(0,0)
\scriptsize{
\put(318,180){$\sigma_a$}
\put(284,250){$\sigma_b$}
\put(124,223){$P$}
\put(48,75){$\lambda_{1a}$}
\put(25,135){$\lambda_{1b}$}
\put(84,238){$\lambda_2$}
}
\end{picture}
\vskip -6mm
      \caption{\em
Profili coniugati inviluppati come traiettorie di un punto $P$ trascinato dalla
polare $\lambda_2$ che rotola sia su $\lambda_{1a}$, sia su $\lambda_{1b}$.
}
 \label{fig:profili_coniugati_punto}
\end{figure}
\noindent Nelle figure \ref{fig:profili_coniugati_invil},
 \ref{fig:profili_coniugati_invil_lagrange} e
 \ref{fig:profili_coniugati_invil_euler} abbiamo scelto  $\sigma_2$ 
che genera gli altri profili coniugati, di forma ellittica. Se tale curva,
rimpicciolendosi, degenerasse in un punto tutto l'impianto della generazione
dei profili coniugati
rimarrebbe inalterato. Le figure
\ref{fig:profili_coniugati_punto},
\ref{fig:profili_coniugati_punto_lagrange} e
\ref{fig:profili_coniugati_punto_euler}
sono quelle che in tal caso
si ottengono. Le riportiamo
prive di commento, in quanto dovremmo ripetere i medesimi
ragionamenti che abbiamo svolto poc'anzi senza alcuna aggiunta.
Possiamo quindi affermare che anche i profili generati
come traiettoria di un punto solidale a una delle polari o primitiva del moto,
durante il suo rotolamento sulle rimanenti,
sono tra loro coniugati. 
I profili coniugati di assortimento ottenuti come inviluppo
di una linea saranno utilizzati nel prossimo paragrafo per disegnare
le ruote dentate con denti profilati a evolvente di cerchio,
le quali costituiscono la quasi totalit\`a delle ruote dentate di
uso comune nelle applicazioni tecniche.
I profili provenienti invece dalla traiettoria di un punto possono
generare ruote con denti a profilo
cicloidale. Queste, unitamente ad alcune loro applicazioni,
verranno descritte brevemente, come gi\`a segnalato, nel capitolo
\ref{ruotecy}.
A conclusione di questo laborioso paragrafo possiamo affermare
che le figure \ref{fig:profili_coniugati_invil} e
\ref{fig:profili_coniugati_punto}, insieme alle loro evoluzioni cinematiche,
sono riportate su quasi
tutti i testi di Meccanica Applicata alle Macchine. Citando il solito
\cite{sesini1}, esse si trovano tra le pagine 71 e 74. Ma tali figure
risultano reperibili in un gran numero di testi
dove campeggiano con aspetti spesso molto simili tra loro. Anche le dimostrazioni
associate, le quali assicurano la possibilit\`a di
 ottenere profili coniugati di 
assortimento, si somigliano tutte e probabilmente sono maggiormente
 valide ed eleganti, confrontate con quella che 
abbiamo proposto. Di sicuro sono molto pi\`u sintetiche.

\begin{figure}[hbt]
\centering
\begin{minipage}[b]{0.45\textwidth}
\centering
\includegraphics[width=0.9\textwidth]{part2/ruote/FIG/ruote/profili_coniugati_punto_lagrange.pdf}
\begin{picture}(0,0)(170,12)
\scriptsize{
\put(88,87){$\eta_0$}
\put(82,85){\rotatebox{-120}{$\longrightarrow$}}
\put(106,54){$\eta_c$}
\put(101,59){\rotatebox{120}{$\longrightarrow$}}
\put(150,114){$\sigma_a$}
\put(145,124){$\sigma_b$}
\put(60,113){$P_0$}
\put(94,120){$P_c$}
\put(25,35){$\lambda_{1a}$}
\put(9,67){$\lambda_{1b}$}
\put(43,118){$\lambda_2$}
}
\end{picture}
      \caption{\em
I profili $\sigma_a$ e $\sigma_b$ risultano a loro volta coniugati quando la
polare $\lambda_{1a}$ rotola sulla $\lambda_{1b}$ toccandola in un
punto con parametro $\mu=\eta=c$.
}
 \label{fig:profili_coniugati_punto_lagrange}
\end{minipage}\hfill
\begin{minipage}[b]{0.45\textwidth}
\includegraphics[width=0.9\textwidth]{part2/ruote/FIG/ruote/profili_coniugati_punto_euler.pdf}
\begin{picture}(0,0)(180,10)
\scriptsize{
\put(88,92){$\eta_0$}
\put(82,90){\rotatebox{-120}{$\longrightarrow$}}
\put(105,54){$\eta_c$}
\put(100,59){\rotatebox{120}{$\longrightarrow$}}
\put(150,115){$\sigma_a$}
\put(145,126){$\sigma_b$}
\put(94,121){$P_c$}
\put(20,33){$\lambda_{1a}$}
\put(14,67){$\lambda_{1b}$}
\put(84,140){$\lambda_2$}
}
\end{picture}
      \caption{\em
Le primitive del moto  rotolano l'una sull'altra in modo tale da mantenere
 fisso, rispetto all'osservatore assoluto, il loro punto di contatto.
      }
 \label{fig:profili_coniugati_punto_euler}
\end{minipage}
\end{figure}



\section{Profili Coniugati a Evolvente}

\noindent Torniamo ora volentieri
al percorso intrapreso, dal quale qualche perplessit\`a di troppo ci
aveva costretto a una deviazione,
per andare dritti allo studio della cinematica
delle ruote dentate.
\begin{figure}[b]
\centering
\includegraphics[width=0.8\textwidth]{part2/ruote/FIG/ruote/evol1.pdf}
\begin{picture}(0,0)(150,0)
\scriptsize{
\put(42,217){$\sigma_2$}
\put(90,198){$T'$}
\put(117,195){$\sigma'_2$}
\put(36,192){$P'$}
\put(88,160){$\mu'$}
\put(23,156){$\alpha=\alpha'$}
\put(-41,148){$P\equiv T\equiv M$}
\put(148,143){$\lambda_2$}
\put(54,119){$H$}
\put(-32,110){$h$}
\put(-12,100){$\theta$}
\put(-30,90){$F$}
\put(-28,183){$F'$}
\put(104,94){$M'$}
\put(-30,63){$r_b$}
\put(147,59){$\mu$}
\put(-55,19){$r$}
\put(146,26){$\lambda'_2$}
\put(110,29){$H'$}
\put(4,13){$O$}
\put(120,-7){$\lambda_1$}
}
\end{picture}
\vskip 3mm
      \caption{\em
Impianto geometrico per la costruzione dell'evolvente (i).
}
 \label{fig:evol1}
\end{figure}
Abbiamo lasciato il sentiero quando i dubbi circa
la bont\`a delle soluzioni che ci proponeva via via
 l'intuito hanno avvolto un punto
chiave: le ruote delle due famiglie, a) e b), rappresentate nelle figure
\ref{fig:squalo} e \ref{fig:squalo_interno}, di colore azzurro, ingraneranno
tra loro correttamente? Le notizie contenute nel precedente paragrafo ci
assicurano che i profili dei denti delle ruote a),
 ottenuti dall'inviluppo di quelli della dentatrice di figura \ref{fig:squalo}
e quelli dei denti delle b), creati invece dalla dentatrice a curvatura
opposta di figura \ref{fig:squalo_interno}, sono tutti tra loro coniugati.
Del resto, essi sono ottenuti come inviluppo di un segmento di retta
(un fianco del dente della dentatrice) che viene
trascinato dalla stessa primitiva mentre rotola sia su una 
primitiva qualsiasi delle ruote a) sia su una appartenente alle  b), inviluppando cos\`i i profili dei loro denti.
Tutte le ruote a) saranno pertanto
ingranabili con le b), con l'ovvio limite, del quale abbiamo gi\`a
fatto cenno in precedenza, che
la dimensione di queste ultime non pu\`o superare quella della ruota 
dentatrice. Tale limite ci spinge a considerare, per il diametro
della ruota di colore rosa di figura \ref{fig:squalo}, valori molto grandi:
e se pensassimo a una dentatrice di diametro infinito? In questo modo, oltre
il problema di dentare le ruote b) di grandi dimensioni, si risolverebbe
felicemente, come vedremo a breve,
 anche la noiosa questione delle due famiglie. Ricordiamo che,
a suo tempo, tra i vari precetti da rispettare nella costruzione delle ruote
dentate era emerso, in forma debole, quello di avere una sola famiglia
di ruote di assortimento.
Ci si rende facilmente conto che,
trasformando la dentatrice in una 
{\em cremagliera}\index{cremagliera}, con denti a fianco rettilineo,
tale precetto viene automaticamente
rispettato. Infatti, una circonferenza di raggio infinito presenta
curvatura pari a zero e, stante il profilo rettilineo dei suoi denti,
la dentatrice con curvatura opposta risulterebbe essere
sempre la cremagliera di partenza. Tale caratteristica \`e conosciuta
come {\em auto-complementarit\`a}\index{auto-complementarit\`a}
(dell'utensile).
Ma andiamo per gradi e consideriamo la figura \ref{fig:evol1}.
Sia $\lambda_1$ la primitiva circolare di raggio $r$ della ruota che
desideriamo costruire.
Su di essa facciamo rotolare, rigorosamente in assenza di strisciamento,
la primitiva $\lambda_2$, che \`e la retta disegnata in verde,
alla quale \`e solidale la
``curva'' $\sigma_2$, il segmento marrone inclinato di $\theta$
rispetto ad una retta ortogonale a $\lambda_2$\footnote{
La primitiva $\lambda_2$, indicata qui seguendo la tradizione come una
retta, sar\`a, per ovvie motivazioni tecniche, un segmento di lunghezza opportuna,
sufficiente ad ottenere il tratto di inviluppo che former\`a il fianco
del dente. Il segmento $\sigma_2$ viene tradizionalmente trattato come una
semiretta la quale partendo dall'estremo $F$
si estende all'infinito, permettendo in questo modo di disegnare l'intera
evolvente. Anche in questo caso ci accontentiamo di un segmento, avendo gi\`a in mente
che esso costituir\`a il fianco dei denti dell'utensile creatore.}:
questo segmento, trascinato dalla primitiva $\lambda_2$ mentre
rotola sulla circonferenza $\lambda_1$, 
creer\`a l'inviluppo che stiamo cercando.
La figura stabilisce che, per un valore dell'angolo
 $\alpha =0$, la retta  $\lambda_2$
si trovi in posizione orizzontale e sia tangente alla
 circonferenza primitiva $\lambda_1$ nel
punto $P$.
 Il processo di inviluppo, dal quale si otterr\`a il profilo coniugato
a $\sigma_2$, consta nel fare appunto rotolare la $\lambda_2$ sulla $\lambda_1$.
A una rotazione generica $\alpha=\alpha'$ corrisponderanno
la retta $\lambda'_2$ e il segmento
$\sigma'_2$.
\`E interessante individuare, su tale segmento, il punto
attivo nella creazione del profilo coniugato, cio\`e il punto $T'$
in cui il segmento
risulter\`a tangente all'inviluppo che stiamo creando.
$T'$ sar\`a perci\`o determinato
dalla intersezione della retta che, passando dal centro istantaneo di rotazione
$M'$, taglia ortogonalmente il segmento $\sigma'_2$, imponendo cos\`i che $T'$
possieda velocit\`a
parallela al segmento $\sigma'_2$ stesso.
L'individuazione di tale punto porta con s\'e
conseguenze profonde. I due triangoli $ \triangle{P' T' M'}$ e
 $\triangle{M'H'O}$
sono simili, pertanto il secondo, al di l\`a della
sua posizione, rimane lo stesso per ogni valore di $\alpha$.
Ci\`o significa che la lunghezza del segmento $(H'M')$ (indicheremo d'ora in poi
la lunghezza dei segmenti tramite le lettere che li individuano poste tra
parentesi tonde) risulter\`a
costante al variare di $\alpha$ e precisamente avremo $(H'M')= r\sin(\theta)$.
La retta $\mu'$ risulta perci\`o tangente alla
circonferenza di raggio $r_b= r \cos(\theta)$ per qualsiasi valore di $\alpha$,
dove $r_b$, grandezza di importanza rilevantissima, \`e il raggio della
{\em circonferenza di base}\index{circonferenza!di base}.
Inoltre, la lunghezza $(H'T')$ sar\`a data dalla quantit\`a 
$(M'T')$, variabile con $\alpha$, sommata alla quntit\`a fissa $(H'M')$.
Per  $\alpha=0$ abbiamo $P\equiv T\equiv M$
e naturalmente  $(H'M')=(HM)$. 
Si osserva che risulta
sempre  $(M'T')=(M'P')\cos(\theta)$ quindi, dato il rotolamento di $\lambda_2$ 
su $\lambda_1$, $(M'P')=r\alpha$ e $(M'T')=r\alpha\cos(\theta)=r_b \alpha$.
Questo indica altres\`i
che la lunghezza dell'arco $H'H$,  che sottende un angolo pari ad $\alpha$,
equivale a
$(M'T')$ e che il segmento $H'T'$ della retta $\mu'$ si comporta, al variare di
$\alpha$, come un filo che si svolge dalla circonferenza di base.

\begin{figure}[hbt] 
\centering
\includegraphics[width=0.8\textwidth]{part2/ruote/FIG/ruote/evol2.pdf}
\begin{picture}(0,0)(150,0)
\scriptsize{
\put(-11,218){$\sigma_2^o$}
\put(74,218){$\lambda_2^o$}
\put(42,217){$\sigma_2$}
%\put(-86,212){$P''$}
\put(90,198){$T'$}
\put(118,195){$\sigma'_2$}
\put(27,179){$P'$}
%\put(-58,181){$\sigma''_2$}
\put(88,160){$\mu'$}
%\put(-96,165){$\lambda''_2$}
\put(-2,160){$P^o$}
%\put(-30,160){$T''$}
\put(31,165){$\alpha$}
\put(6,147){$P\equiv T\equiv M$}
\put(148,143){$\lambda_2$}
\put(148,134){$\mu^o$}
\put(-76,127){$M^o$}
\put(-3,118){$T^o\equiv H^o$}
\put(56,116){$H$}
\put(-184,107){$\alpha^o=-\tan(\theta)$}
%\put(-2.2,105.5){$\delta$}
\put(104,95){$M'$}
\put(13,92){$\gamma$}
%\put(-79,83){$H''$}
\put(-30,63){$r_b$}
\put(147,59){$\mu$}
\put(34,50){$\beta$}
%\put(-133,42){$M''$}
\put(110,29){$H'$}
\put(146,26){$\lambda'_2$}
\put(-55,19){$r$}
\put(120,-7){$\lambda_1$}
%\put(-144,10){$\mu''$}
\put(4,13){$O$}
\put(-30,90){$F$}
\put(-28,183){$F'$}
\put(-13,102){$F^o$}
%
}
\end{picture}
      \caption{\em
Impianto geometrico per la costruzione dell'evolvente (ii).
      }
 \label{fig:evol2}
\end{figure}

\noindent In questo intreccio di angoli, rette e segmenti riteniamo opportuno,
riferendoci ora alla figura \ref{fig:evol2},
individuare il valore dell'angolo $\beta$ dato un generico valore di $\alpha$.
Da semplici considerazioni trigonometriche otteniamo subito
\begin{equation}
\tan(\beta) = \frac{(H'T')}{(OH')}=\
\frac{(H'M')+(M'T')}{(OH')}=\
\frac{r\sin(\theta)+\alpha  r\cos(\theta)}{r\cos(\theta)}=\
\tan(\theta) + \alpha\,.
\label{eq:beta}
\end{equation}
\vskip 1mm
\noindent Il capo del filo che si svolge dalla circonferenza di base
si individua ponendo $(H^oT^o)=0$, ottenendo in questo modo anche
il valore dell'angolo $\alpha^o$,
tale da fare coincidere $H^o\equiv T^o$:
$r\sin(\theta)+\alpha  r\cos(\theta) = 0 \rightarrow \alpha^o= -\tan(\theta)$.
Consideriamo ora la lunghezza del filo svolto dalla posizione $H^o$ alla
posizione $H'$, la lunghezza dell'arco $H'H^o$ sar\`a
\begin{equation}
\overset{\large\frown}{(H'H^o)} = (H'T') \rightarrow (\beta + \gamma) r_b= r_b \tan(\beta)\,,
\label{eq:gamma0}
\end{equation}
\noindent ottenendo pertanto
\begin{equation}
 \gamma(\beta) = \tan(\beta)-\beta \,.
\label{eq:gamma}
\end{equation}
\noindent La quantit\`a ora ricavata, unitamente alla distanza
\begin{equation}
\rho(\beta) = (OT') =  \frac{r_b}{\cos(\beta)}\,,
\label{eq:rho}
\end{equation}
\noindent forma la coppia delle coordinate polari del  punto generico
 $T'$ dell'inviluppo
che risulta essere l'{\em evolvente}\index{evolvente}
 della circonferenza di base con inizio
nel punto $H^o$. Questa curva \`e riportata in vari manuali di Matematica,
e pu\`o essere definita proprio come luogo dei punti dell'estremo di un filo
che si svolge da una determinata circonferenza a partire da un punto iniziale.
L'angolo $\beta$ riveste i panni di
parametro delle
coordinate polari di questa curva ed \`e
legato al nostro angolo di rotolamento
$\alpha$ dalla relazione \ref{eq:beta}. Tale formula permetterebbe,
invertendo la tangente, di sostituire $\beta$ con $\alpha$ nelle
\ref{eq:gamma} e \ref{eq:rho}.
Ma questa operazione
non viene in genere eseguita, forse perch\'e, come vedremo a breve,
in quel poco (o molto) utilizzo che si fa delle coordinate polari dell'evolvente,
l'angolo $\beta$ si comporta egregiamente.

\noindent Riepiloghiamo. La costruzione di profili di assortimento per i denti
ci porta a una primitiva comune a tutti gli inviluppi
che \`e una retta, $\lambda_2$, che rotoler\`a sulle primitive delle ruote da
tagliare.  Ulteriori 
considerazioni, legate alla simmetria, ci indicano come curva inviluppante
un segmento di retta, $\sigma_2$, solidale a $\lambda_2$ e inclinato
rispetto a questa di un angolo opportuno, $\pi/2 -\theta$. L'angolo $\alpha$,
mediante il quale la $\lambda_2$ rotola sulla circonferenza di raggio
$r$, governa il processo di inviluppo. Tramite i ragionamenti sopra esposti
perveniamo alla definizione di tale inviluppo che risulta essere
un'evolvente della circonferenza di base, di raggio $r_b=r \cos(\theta)$.
Pur riconoscendo l'elevato valore culturale 
di questa scoperta, del fatto cio\`e che il profilo dei denti, ottenuti tramite
il taglio con una dentiera rettilinea a fianchi pure rettilinei, sia un'evolvente
della circonferenza
di base, non dedicheremo nel presente lavoro molto spazio a questa affascinante
curva geometrica. Come riportato in \cite{guiggiani},
dal quale confessiamo di avere preso diversi spunti: ``Il fatto che il
profilo attivo sia una evolvente di cerchio \`e, in fondo, una fortunata
coincidenza''.
Ci troviamo d'accordo con questa affermazione, riservandoci
 di valutare meglio l'epiteto. Torneremo a breve
su alcune (poche) applicazioni delle formule relative all'evolvente,
nel paragrafo che tratta la correzione delle ruote mediante lo spostamento
dei profili, in particolare quando opereremo con variazione di interasse.
\noindent Prima di passare agli inviluppi veri e propri, che definiranno
la forma dei denti, \`e bene fare ancora qualche considerazione sulla
figura \ref{fig:evol2}. Quando la retta $\lambda_2$ si trova nella
posizione $\lambda^o_2$ il punto $T^o\equiv H^o$ cade sulla circonferenza
di base. Tale posizione rappresenta, sul segmento $\sigma_2$,
l'estremo oltre il quale quest'ultimo risulta inefficace nel creare
l'inviluppo. Pertanto, una posizione dell'estremo $F$ esterna a tale punto,
come quella riportata nelle  nostre figure, \`e  disutile
 in quanto la frazione in eccesso, non partecipando all'inviluppo, avrebbe come
unico effetto quello di distruggere una porzione 
del materiale da tagliare: una situazione analoga a quella
di figura \ref{fig:profili_coniugati_definizione_c}, innocua nella teoria,
inaccettabile nella pratica. La distanza $(T^oP^o)$ ha perci\`o
un significato tecnico importante e vale
\begin{equation}
(T^oP^o)=\frac{r }{\cos(\theta)}\sin^2(\theta)\,,
\label{eq:maxt}
\end{equation}
come si deduce facilmente considerando i due triangoli simili
$\triangle{OP^oM^o}$ e $\triangle{M^oP^oT^o}$.
In figura \ref{fig:evol1} \`e stata indicata con $h$ la lunghezza della
proiezione di $FP$ su una direzione ortogonale all'epiciclo $\lambda_2$. Tale
lunghezza rappresenta, come vedremo a breve, una porzione
dell'altezza dei denti dell'utensile che stiamo cercando. 

\begin{figure}[hbt] 
\centering
\includegraphics[width=0.7\textwidth]{part2/ruote/FIG/ruote/inviluppo_evolvente.pdf}
\begin{picture}(0,0)(150,0)
\scriptsize{
\put(14,234){$\sigma_2^o$}
\put(17,75){$T^o\equiv H^o$}
\put(-66,19){$r$}
\put(84,19){$r_b$}
\put(150,81){$\lambda_1$}
%
}
\end{picture}
      \caption{\em
Inviluppo dell'evolvente e relativo arco di trocoide. Sporgenza di $\sigma_2$
uguale ad $h_{\rm max}$, $\theta=20^{\circ}$.
      }
 \label{fig:inviluppo_evolvente}
\end{figure}
\noindent Risulta perci\`o ancora pi\`u rilevante
specificare la lunghezza della proiezione di $(T^oP^o)$ sulla direzione
ortogonale a $\lambda^o_2$, che fornir\`a
la massima altezza utile, a partire da $\lambda_2$,
dei denti della nostra dentiera generatrice:
\begin{equation}
h_{\rm max}=r  \sin^2(\theta)\,.
\label{eq:maxh}
\end{equation}
\noindent La figura \ref{fig:inviluppo_evolvente} mostra un fianco di dente,
inviluppato dal segmento $\sigma_2$, proprio in questa condizione limite,
quando cio\`e $(FP)=(T^oP^o)$.
Si pu\`o notare che, affinch\'e l'evolvente sia completa, partendo dalla
circonferenza di base, il fianco della dentiera scava ulteriormente
al di sotto di questa, costruendo un raccordo alla base del fianco del dente 
stesso. La medesima figura ci mostra che, al
variare di $\alpha$, l'estremo del segmento generatore non sconfina mai
al di l\`a del prolungamento ideale di $\sigma_2^o$ verso il centro della ruota,
essendo $\sigma_2^o$ l'ultima posizione utile nell'inviluppo dell'evolvente
stessa. Le ulteriori posizioni di $\sigma_2$, riportate in colore marrone,
rappresentano l'uscita di scena del profilo creatore. Sull'estremo inferiore
di quest'ultimo abbiamo evidenziato un minuscolo cerchio, rispettando la coerenza
del suo colore e quello del segmento che lo porta. Tali cerchi definiscono graficamente
l'arco di {\em trocoide}\index{trocoide}\footnote{
Si chiamano trocoidi le curve disegnate da punti esterni all'epiciclo
e solidali con esso durante il
rotolamento di quest'ultimo sulla circonferenza che percorre. Nel nostro caso
l'epiciclo (o ipociclo: nella teoria \`e possibile far rotolare circonferenze
pi\`u grandi all'interno di quelle pi\`u piccole) 
\`e la retta $\lambda_2$ che rotola sulla primitiva $\lambda_1$.
} che, come si vede, delinea un raccordo del profilo del dente con una
possibile circonferenza interna a quella di base: la {\em circonferenza
di piede}\index{circonferenza!di piede}.

\begin{figure}[h]
\centering
\begin{minipage}[b]{0.45\textwidth}
\centering
\includegraphics[width=0.9\textwidth]{part2/ruote/FIG/ruote/inviluppo_evolvente_sottotaglio.pdf}
\begin{picture}(0,0)(130,0)
\scriptsize{
\put(49.2,137){$\sigma_2^o$}
%\put(52,19){$T^o$}
\put(-8,19){$r$}
\put(89,19){$r_b$}
\put(130,45){$\lambda_1$}
}
\end{picture}
      \caption{\em
Inviluppo con $\theta=20^{\circ}$ e $(FP)=2(T^oP^o)$.
}
 \label{fig:inviluppo_evolvente_sottotaglio}
\end{minipage}\hfill
\begin{minipage}[b]{0.45\textwidth}
\includegraphics[width=0.9\textwidth]{part2/ruote/FIG/ruote/inviluppo_evolvente_corto.pdf}
\begin{picture}(0,0)(130,0)
\scriptsize{
\put(49.2,137){$\sigma_2^o$}
%\put(53,54){$T^o$}
\put(-8,19){$r$}
\put(89,19){$r_b$}
\put(68,19){$r_e$}
\put(130,45){$\lambda_1$}
}
\end{picture}
      \caption{\em
Inviluppo con $\theta=20^{\circ}$ e $(FP)=0.66(T^oP^o)$.
      }
 \label{fig:inviluppo_evolvente_corto}
\end{minipage}
\end{figure}

\noindent Il raggio della circonferenza di piede si deduce considerando che
la proiezione del segmento $FP$ sulla direzione ortogonale a $\lambda_2$
\`e la massima profondit\`a, in direzione radiale, alla quale arriver\`a
la trocoide e vale
\begin{equation}
r_p = r - (FP)\cos\theta = r - h\,.
\label{eq:r_p}
\end{equation}
\noindent La figura \ref{fig:inviluppo_evolvente_sottotaglio} rappresenta
l'inviluppo creato da un segmento $\sigma_2$ di lunghezza eccessiva (verso
la ruota da tagliare). L'evolvente risulta completa e tocca
la circonferenza di base, ma il fianco del dente viene scavato alla base e
una piccola parte della evolvente stessa viene distrutta: 
situazione questa che viene indicata col nome di {\em sottotaglio dei
denti}\index{sottotaglio dei denti}. 
La figura \ref{fig:inviluppo_evolvente_corto} rappresenta invece il
profilo ottenuto quando il segmento $\sigma_2$ ha una lunghezza insufficiente
a inviluppare tutta l'evolvente, alla quale mancher\`a un arco in prossimit\`a
della circonferenza di base.  La coordinata
polare radiale del punto dell'evolvente $T'$, gi\`a riportata
nella formula \ref{eq:rho}, si pu\`o anche ricavare, 
con riferimento alla figura  \ref{fig:evol2}, applicando
il teorema di Pitagora al triangolo $\triangle OT'H'$. 
Al fine quindi di individuare il raggio della circonferenza da cui partir\`a
l'evolvente, sostituiamo il
punto $T'$ con l'estremo di $F$ di $\sigma_2$ prestando attenzione
al segno opportuno per la porzione di segmento $PF$.
Stanti queste premesse, l'evolvente partir\`a da una circonferenza di raggio 
\begin{equation}
r_e = \sqrt{r_b^2 +\left(\frac{r_b\sin\theta}{\cos\theta}-\frac{(FP)\cos\theta}{\sin\theta}\right)^2}\,,
\label{eq:ini_evo}
\end{equation}
\noindent e in termini di sporgenza dei denti della dentiera (utensile) $h$
\begin{equation}
r_e = \sqrt{r_b^2 +\left(\frac{r_b\sin\theta}{\cos\theta}-\frac{h}{\sin\theta}\right)^2}\,.
\label{eq:ini_evoh}
\end{equation}

\begin{figure}[hbt]
\centering
\begin{minipage}[b]{0.45\textwidth}
\centering
\includegraphics[width=0.9\textwidth]{part2/ruote/FIG/ruote/inviluppo_evolvente_sottotaglio35.pdf}
\begin{picture}(0,0)(130,0)
\scriptsize{
\put(43.5,150){$\sigma_2^o$}
%\put(52,19){$T^o$}
\put(20,19){$r$}
\put(88,19){$r_b$}
\put(128,98){$\lambda_1$}
}
\end{picture}
      \caption{\em
Inviluppo con $\theta=35^{\circ}$ e $(FP)=2(T^oP^o)$.
}
 \label{fig:inviluppo_evolvente_sottotaglio35}
\end{minipage}\hfill
\begin{minipage}[b]{0.45\textwidth}
\includegraphics[width=0.9\textwidth]{part2/ruote/FIG/ruote/inviluppo_evolvente_corto35.pdf}
\begin{picture}(0,0)(130,0)
\scriptsize{
\put(43.5,150){$\sigma_2^o$}
%\put(53,54){$T^o$}
\put(20,19){$r$}
\put(88,19){$r_b$}
\put(73,19){$r_e$}
\put(128,98){$\lambda_1$}
}
\end{picture}
      \caption{\em
Inviluppo con $\theta=35^{\circ}$ e $(FP)=0.66(T^oP^o)$.
      }
 \label{fig:inviluppo_evolvente_corto35}
\end{minipage}
\end{figure}

\noindent Giusto per dare un'idea dell'influenza che il valore dell'angolo
 di pressione $\theta$ esercita su quello che sar\`a il profilo dei denti
 riportiamo le figure \ref{fig:inviluppo_evolvente_sottotaglio35} e 
\ref{fig:inviluppo_evolvente_corto35} per le quali, anzich\'e
assumere $\theta=20^{\circ}$, che \`e il valore {\em standard} dell'angolo di pressione,
si \`e posto $\theta=35^{\circ}$.
La differenza pi\`u rilevante tra queste ultime due figure e le loro precedenti
omologhe consta in una significativa riduzione del raggio di base. Ci\`o
permette, aumentando l'angolo di pressione, di poter considerare
altezze utili della dentiera pi\`u elevate, come del resto \`e evidente
anche dalla formula \ref{eq:maxh}.
Ci si pu\`o chiedere se due profili, generati con angoli $\theta$ diversi,
siano tra loro coniugati. Formulata in questo modo la domanda e di conseguenza
la risposta necessitano di qualche precisazione. La risposta \`e negativa
qualora intendessimo far rotolare,
l'una sull'altra, le due circonferenze primitive (assegnate)
e ottenere i due profili tramite due diversi epicicli.
\`E la teoria dei profili coniugati a escluderlo:
deve necessariamente essere lo stesso epiciclo a rotolare
su tutte le primitive per poter ottenere i cosiddetti {\em profili di assortimento}\index{profili!di assortimento}.
Qualora invece ci si interroghi circa la possibilit\`a di coniugare i due
profili a evolvente tramite circonferenze primitive diverse da quelle di
partenza la
risposta \`e affermativa. Infatti le due curve saranno le evolventi di 
due circonferenze di base di raggi $r_{b_1}$ e $r_{b_2}$. Se tali circonferenze
vengono affacciate, mantenendo i loro centri a distanza $i>r_{b_1}+r_{b_2}$,
si avranno due possibili raggi primitivi e un possibile angolo di pressione
dati da
\begin{equation}
r_1=\frac{r_{b_1}i}{r_{b_1}+r_{b_2}}\,,\;\;\;
r_2=\frac{r_{b_2}i}{r_{b_1}+r_{b_2}}\; {\rm e}\;\;\;
\theta=\arccos\frac{r_{b_1}+r_{b_2}}{i}\,.
\label{eq:nuove_primitive}
\end{equation}
\noindent Lasciamo la verifica di queste 
formule alla volont\`a di qualche studente.
Anticipiamo solamente che i valori dei due raggi primitivi
presentano scarso interesse applicativo, mentre quello dell'angolo di pressione
per un interasse dato risulta molto rilevante per gli ingranaggi con ruote
corrette, come vedremo, in succinto, in un prossimo paragrafo.
Il fatto che i profili a evolvente siano, entro limiti ragionevoli,
tutti fra loro potenzialmente coniugati mediante primitive circolari
risulta di grandissima importanza. Da ci\`o deriva infatti
la facolt\`a di variare, sempre entro i limiti menzionati,
 l'interasse di progetto
nella fase di montaggio delle ruote che portano tali profili:
vantaggio notevole, che permette di assorbire eventuali errori
di fabbricazione e messa in opera. Un risultato ancora pi\`u importante
della flessibilit\`a derivante dalla \ref{eq:nuove_primitive} \`e
la possibilit\`a di tagliare
le cosiddette ruote a profilo corretto nelle quali, sacrificando
senza alcun rimorso la geometria ``normale''  di tali ruote e spostando
opportunamente l'utensile creatore, si ovvia al grave problema del sottotaglio,
chiamato spesso {\em interferenza} che, come vedremo in un prossimo
paragrafo, pone un limite inferiore al numero di denti dei pignoni; ma
soprattutto offre al progettista l'opportunit\`a di individuare una strategia
di ottimizzazione degli ingranaggi, quasi sempre basata sull'utilizzo di
interassi diversi da quelli che imporrebbe il taglio di ruote ``normali''.


\section{Proporzionamento Modulare delle Ruote}

\noindent Si \`e gi\`a accennato altrove che, in ottemperanza al primo
precetto che abbiamo ammesso, cio\`e il vincolo  che impone un
numero intero di denti per le ruote dentate, nel progetto di queste ultime
si parte dal loro numero di denti, e non dai loro diametri o dai loro raggi.
Il numero di denti $z$ di una ruota rappresenta per\`o solamente una
``situazione'' geometrica angolare. In pratica, $z$ impone il
numero di spicchi mediante il quale si desidera
dividere la torta. La dimensione effettiva della
ruota deve pertanto essere legata a un fattore di scala dimensionale.
Tale parametro si chiama {\em modulo}\index{modulo}, $m$, ed \`e
a sua volta legato al numero di denti e al raggio della circonferenza primitiva
 dalla relazione
\begin{equation}
2 r =mz\,,
\label{eq:mod_z}
\end{equation}
\noindent gi\`a riportata in precedenza, ma di importanza tale da non
farci sentire in imbarazzo ripetendola. Il modulo $m$ viene espresso,
nei paesi in cui si usa il sistema metrico decimale, in ${\rm mm}$. Abbiamo 
gi\`a fatto cenno anche al {\em passo}\index{passo} della ruota che \`e
la lunghezza dell'arco di primitiva che separa due fianchi, destri o sinistri,
attigui:
\begin{equation}
p=\pi m\,.
\label{eq:passo}
\end{equation}
\noindent Dalla \ref{eq:passo} si deduce immediatamente che la possibilit\`a
di ingranare di due ruote dentate \`e condizionata dal loro modulo:
soltanto ruote con moduli uguali tra loro potranno formare ingranaggi.
L'unificazione impone per il modulo $m$ una certa successione di valori
normalizzati, della quale riportiamo i pi\`u usuali
$m=1,\;1.25,\;1.5,\;2,\;2.5,\;3,\;4,\;5,\;6\; {\rm mm}$.
Si riduce, in questo modo, la variet\`a di scelta di questo fondamentale
parametro a fronte
di notevoli vantaggi, evidenti nella fase di progettazione,
ma ancora di pi\`u se si pensa alla fase di taglio.
A ciascun valore del modulo corrisponde infatti un certo numero di utensili,
{\em in primis} la dentiera tagliente (o l'utensile creatore)
 e a seguire le mole di rettifica,
ove necessario.
Per passare dai profili rappresentati nelle precedenti figure
agli effettivi fianchi dei denti delle ruote da
tagliare occorre estendere l'impianto teorico della primitiva $\lambda_2$
equipaggiata di un solo segmento $\sigma_2$ aggiungendo altre parti e
apportando qualche leggera modifica a tali elementi.
\begin{figure}[hbt]
\begin{center}
\includegraphics[width=0.94\textwidth]{part2/ruote/FIG/ruote/dentiera_riferimento.pdf}
\end{center}
\begin{picture}(0,0)(0,9.6)
\scriptsize{
\put(15,110){\rotatebox{90}{$2.5\, m$}}
\put(38,120.5){\rotatebox{90}{$1.25\, m$}}
\put(38,95){\rotatebox{90}{$1.25\, m$}}
\put(167,52){$\theta$}
\put(170,86){$F$}
\put(267,43){$p=\pi m$}
\put(264,123){$\lambda_2$}
\put(264,113.5){$\lambda_t$}
\put(76,135){$\sigma_2^-$}
\put(143,135){$\sigma_2^-$}
\put(210,135){$\sigma_2^-$}
\put(277,135){$\sigma_2^-$}
\put(121,135){$\sigma_2$}
\put(188,135){$\bm \sigma_2$}
\put(255,135){$\sigma_2$}
\put(322,135){$\sigma_2$}
\put(293,62){$p/2$}
\put(332,114){$c=mx$}
}
\end{picture}
\vskip -10mm
      \caption{\em
Dentiera normalizzata di riferimento.\index{dentiera!di riferimento}
}
 \label{fig:dentiera_riferimento}
\end{figure}

\noindent In figura \ref{fig:dentiera_riferimento} si possono notare
 la primitiva $\lambda_2$ e il segmento $\sigma_2$ contornati da
 altri oggetti
dei quali definiremo, in breve, natura e scopo.
Innanzitutto, notiamo che, a parte il numero
 di denti dell'utensile\footnote{
Il numero di denti $z_c$ dell'utensile creatore ({\em Maag}) \`e normalmente contenuto
tra i limiti $1\le z_c \le 8$, riservando il valore maggiore,
in generale, al taglio di ruote con numero di denti $z>60$.
}
e l'angolo $\theta$, tutte
le altre grandezze sono facilmente rapportabili al modulo $m$.
Il profilo $\sigma_2$, che nelle figure \ref{fig:evol1} e \ref{fig:evol2}
terminava in basso col punto $F$, e che veniva volutamente lasciato di
lunghezza indefinita all'altro capo, viene ora troncato in modo speculare
(non invertito)
rispetto a $\lambda_2$, sapendo gi\`a che i denti, dovendo avere altezza
finita, non beneficerebbero del prolungamento dell'evolvente oltre quella che,
con nome significativo, si chiama {\em circonferenza di testa}\index{circonferenza!di testa}.
Tale segmento viene poi ripetuto diverse volte e precisamente
il doppio del numero di denti di cui si vuole dotare l'utensile creatore.
La met\`a di tali segmenti, segnati coi nomi $\sigma_2^-$, 
\`e orientata in modo da creare un angolo
$-\theta$ con una linea perpendicolare  $\lambda_2$. Questi fianchi
dell'utensile sono necessari
per costruire, in una sola passata, entrambi i fianchi dei denti: le ruote
dentate trasmettono il moto in tutte e due le direzioni possibili e per
tali direzioni il loro comportamento deve essere equivalente\footnote{
Ricordiamo i precetti introdotti: il secondo (debole) prescrive
proprio questa simmetria. Esso viene talvolta disatteso, sia per particolari
applicazioni che richiedono asimmetria di comportamento dell'ingranaggio
rispetto alle direzioni del flusso di potenza, sia per ottenere
vantaggi circa la riduzione di rumorosit\`a della trasmissione, come 
testimoniato da alcune specifiche e recenti ricerche.
Stando alla nostra modesta esperienza,
confessiamo per\`o di non aver mai toccato con mano
 {\em ruote asimmetriche}\index{ruote!asimmetriche}.}.
Per una {\em dentiera normalizzata}\index{dentiera!normalizzata}
valgono le seguenti relazioni, una parte delle quali \`e riportata 
anche nella figura:
l'altezza dei denti \`e legata al modulo e vale $2.5\,m$, ugualmente
ripartita tra {\em addendum}\index{addendum} e {\em dedendum},\index{dedendum}
che pertanto valgono entrambi $1.25\, m$; i denti e i vani hanno la 
medesima forma e il passo tra due fianchi consecutivi e omologhi,
misurato su una
qualunque retta orizzontale che intercetti tali fianchi, vale $p=\pi m$.
Il valore dell'angolo di pressione che, in Europa, si ritiene normalizzato
vale $\theta=20^{\circ}$.
Oltre alla primitiva $\lambda_2$, sempre in figura 
\ref{fig:dentiera_riferimento}, troviamo la retta orizzontale $\lambda_t$
che chiameremo {\em primitiva di taglio}\index{primitiva!di taglio},
e che individua l'effettiva primitiva usata durante il moto di taglio:
essa, rotolando sulla primitiva della ruota da tagliare,
crea, tramite i profili $\sigma$, i profili dei denti.
Tale retta, rappresentata a distanza $c=mx$ 
da $\lambda_2$, si trover\`a a coincidere con questa ($c=0$) nel
taglio di {\em ruote normali non corrette}\index{ruote!non corrette}:
$x$ si chiama {\em fattore di correzione adimensionale}\index{correzione!adimensionale}, del quale daremo qualche ulteriore notizia in
un prossimo paragrafo.

\section{Taglio delle Ruote Dentate per Inviluppo}

\noindent Siamo ormai dotati dell'utensile opportuno per poter
eseguire di nuovo le dentature delle figure \ref{fig:squalo} e
 \ref{fig:squalo_interno}. Chiediamo al lettore di pazientare ancora
un momento, in modo da poter ritoccare il nostro utensile e renderlo somigliante
a quelli che realmente si usano nel processo industriale del taglio delle 
ruote dentate.
\noindent Chiariamo subito che la dentiera rappresentata in figura \ref{fig:dentiera_maag}
corrisponde solo teoricamente a quelle che si usano nel processo di taglio
per inviluppo delle ruote dentate e che portano il nome {\em Maag}\index{dentiera!Maag}.
\begin{figure}[t]
\begin{center}
\includegraphics[width=0.98\textwidth]{part2/ruote/FIG/ruote/dentiera_maag.pdf}
\end{center}
\begin{picture}(0,0)(0,9.6)
\scriptsize{
\put(5,99){\rotatebox{90}{$2.5\, m$}}
\put(33,116){\rotatebox{90}{$=$}}
\put(33,93){\rotatebox{90}{$=$}}
\put(181,55){$r_r=0.4m$}
\put(132,36){$\theta=20^{\circ}$}
\put(90,69){\rotatebox{-20}{$p_b=p \cos\theta$}}
\put(287,42){$p=\pi m$}
\put(277,110){$\lambda_2$}
\put(277,102){$\lambda_t$}
\put(307,59){$p/2$}
\put(339,102){$c=mx$}
}
\end{picture}
\vskip -6mm
      \caption{\em
Utensile Maag.
}
\vskip -3mm
\label{fig:dentiera_maag}
\end{figure}
Infatti, tali utensili a forma di pettine
devono possedere, oltre alle qualit\`a geometriche che interessano la nostra
disciplina, molti altri accorgimenti, come quelli che permettano il loro
afferraggio sulla macchina dentatrice
e altri che facilitino il processo tecnologico del taglio mediante asportazione
di truciolo: i taglienti affilati e
gli angoli di spoglia, irrinunciabili in tali processi\footnote{
Nella pratica \`e molto frequente mettere diversi pettini come quelli di
figura \ref{fig:dentiera_maag} adagiati sulle generatrici di un cilindro, ottenendo
cos\`i quello che in gergo si chiama {\em creatore}\index{creatore}. Il creatore taglia i denti 
tramite un movimento rotatorio e traslatorio e il moto di rotolamento tra le due
primitive avviene con continuit\`a. Per quest'ultimo motivo, l'asse
dei denti di ciascun pettine deve essere un'elica, e l'insieme di queste
schiere di denti conferiscono all'utensile l'aspetto di una vite.
} e non facilmente rappresentabili nel piano dove giacciono i profili dei denti.
La differenza pi\`u evidente tra la dentiera di riferimento di figura
\ref{fig:dentiera_riferimento} e l'utensile Maag, \`e la forma arrotondata
delle teste e del fondo dei denti. Questo accorgimento si rende necessario
per due ordini di difficolt\`a che l'uso di una dentiera
con spigoli vivi comporterebbe.
La prima di tali difficolt\`a risiede in ambito tecnologico
e si pu\`o riassumere dicendo che i taglienti spigolosi (bench\'e ammissibili)
sono estremamente pi\`u fragili rispetto a quelli che presentano spigoli
arrotondati. Il secondo vantaggio offerto dallo smusso dei denti riguarda
la possibilit\`a di costruire pignoni recanti un minor numero di denti,
cio\`e di ovviare in parte a quello che tradizionalmente si
chiama {\em problema dell'interferenza}, alla soluzione del
 quale dedicheremo, appena pi\`u avanti, uno spazio adeguato.
\begin{figure}[hbt]
\centering
\begin{minipage}[b]{0.48\textwidth}
\centering
\includegraphics[width=0.98\textwidth]{part2/ruote/FIG/ruote/17denti_4mod_20gradi_0corr_0p01rr_inv.pdf}
\begin{picture}(0,0)(130,0)
\scriptsize{
\put(7,45){\rotatebox{25}{$\longrightarrow$}}
\put(9,38){\rotatebox{35}{$\longrightarrow$}}
\put(14,30){\rotatebox{45}{$\longrightarrow$}}
\put(27,21){\rotatebox{75}{$\longrightarrow$}}
\put(0,13){punti della trocoide}
}
\end{picture}
      \caption{\em
Profilo dei denti generati dalla dentiera di riferimento: $z=17$, $\theta=20^{\circ}$, $r_r=0$.
}
 \label{fig:17denti_4mod_20gradi_0corr_0p01rr_inv}
\end{minipage}\hfill
\begin{minipage}[b]{0.48\textwidth}
\includegraphics[width=0.98\textwidth]{part2/ruote/FIG/ruote/17denti_4mod_20gradi_0corr_0p4rr_inv.pdf}
\begin{picture}(0,0)(130,0)
\scriptsize{
}
\end{picture}
      \caption{\em
Profilo dei denti generati da dentiera {\em Maag}: $z=17$, $\theta=20^{\circ}$, $r_r=0.4m$.
      }
 \label{fig:17denti_4mod_20gradi_0corr_0p4rr_inv}
\end{minipage}
\vskip -3mm
\end{figure}
Crediamo sia istruttivo fare un confronto tra gli inviluppi
prodotti dalla dentiera di riferimento
e dall'utensile {\em Maag} sullo stesso pignone con $z=17$, che, come
vedremo, presenta il minimo numero di denti possibili, al di sotto del quale
si manifesta il sottotaglio alla base dei denti. 
Come mostrato in figura 
\ref{fig:17denti_4mod_20gradi_0corr_0p01rr_inv}, la quale riporta
l'inviluppo ottenuto da una dentatrice con spigoli vivi,
lo sconfinamento verso l'asse del dente dei
 punti del raccordo di piede (l'arco di trocoide) rende la base del 
dente stesso sottotagliata. Questo comporta una sua diminuzione della
resistenza a flessione, comportandosi il dente, dal punto di vista
strutturale, come una mensola
sollecitata da un carico collocato a distanza variabile dall'incastro.
 Del resto il sottotaglio era prevedibile in quanto,
 stando al proporzionamento modulare della dentiera di riferimento,
abbiamo per l'altezza dei suoi denti $h=1.25 m$.  Sostituendo questo valore nella
\ref{eq:maxh} otteniamo
\begin{equation}
z_{\rm min}=\frac{2.5}{\sin^2(\theta)}\,,
\label{eq:zminrif}
\end{equation}
\noindent e per $\theta= 20^{\circ}$ si  ottiene
$z_{{\rm min}_{20^{\circ}}}=22$\index{numero minimo di denti}. Gi\`a in questo luogo, desideriamo
mettere in evidenza, alla buona, cio\`e mediante il calcolo
diretto della \ref{eq:zminrif} per
$\theta=25^{\circ}$, la sensibilit\`a del risultato di
 tale formula alla variazione dell'angolo di 
pressione: abbiamo, ad esempio, $z_{{\rm min}_{25^{\circ}}}=14$. La figura
\ref{fig:17denti_4mod_20gradi_0corr_0p4rr_inv} riporta invece il
profilo dei denti di un pignone con $z=17$ e $\theta = 20^{\circ}$ ottenuto per\`o
tramite utensile {\em Maag}, a spigoli smussati, con raggio di raccordo delle creste
dei denti $r_r=0.4m$.
In quest'ultimo caso, si nota che le teste dei denti
dell'utensile sono molto vicine al limite del sottotaglio\footnote{
Quanto qui asserito ($z_{{\rm min}_{20^{\circ}}}=17$) si ricava facilmente
osservando che nella formula \ref{eq:zminrif}, nel caso di utensile con denti
 smussati, il numeratore passerebbe da $2.5$ a $2$.
},
pertanto si ammette in generale,
e nelle condizioni ora menzionate, che il numero minimo  di denti per
un pignone (senza correzione) \`e $z_{{\rm min}_{20^{\circ}}}=17$.
Nelle due figure ora citate riportiamo anche un arco della circonferenza di
testa o di troncatura della ruota da tagliare, circonferenza che delimita
l'estremit\`a esterna dei denti.
Notiamo che in questo caso
la presenza di uno smusso alla base dei denti della dentiera non
incide sulla forma della testa dei denti della ruota:
 tale raccordo \`e un mero accorgimento costruttivo,
\begin{figure}[t]
\centering
\begin{minipage}[b]{0.48\textwidth}
\centering
\includegraphics[width=0.98\textwidth]{part2/ruote/FIG/ruote/17denti_4mod_20gradi_0corr_0p4rr.pdf}
\vskip 5mm
\begin{picture}(0,0)(130,0)
\scriptsize{
}
\end{picture}
      \caption{\em
Pignone con $z=17$, $\theta=20^{\circ}$: assenza di sottotaglio.
}
 \label{fig:17denti}
\end{minipage}\hfill
\begin{minipage}[b]{0.48\textwidth}
\includegraphics[width=0.98\textwidth]{part2/ruote/FIG/ruote/60denti_4mod_20gradi_0corr_0p4rr.pdf}
\begin{picture}(0,0)(130,0)
\scriptsize{
}
\end{picture}
      \caption{\em
Ruota dentata con $z=60$, $\theta=20^{\circ}$: fianchi dei denti somiglianti a quelli di una cremagliera.
      }
 \label{fig:60denti}
\end{minipage}
\vskip -3mm
\end{figure}
\noindent In figura \ref{fig:17denti} riportiamo finalmente ci\`o che si ottiene
tagliando una ruota con la dentiera {\em Maag} quando le due primitive, quella
della dentiera e quella della ruota, coincidono. Il pignone raffigurato ha
numero di denti $z=17$, angolo di pressione $\theta=20^{\circ}$
e, come si pu\`o notare visivamente,
 ci troviamo al limite del sottotaglio, anche se il numero di denti \`e
inferiore a 22 che, ripetiamo, rappresenta il minimo teorico per un
utensile con spigoli vivi.
La figura \ref{fig:60denti} mostra invece una ruota con $z=60$ e angolo
di pressione sempre $\theta=20^{\circ}$. In questo caso si nota che,
trovandoci molto lontani dal numero minimo di denti, la circostanza
del sottotaglio \`e
ben lungi dal presentarsi e i denti appaiono molto robusti alla base.
Si pu\`o osservare inoltre che i fianchi
dei denti, pur essendo (ovviamente) degli archi di evolvente,
assomigliano molto a quelli rettilinei della cremagliera.
Crediamo sia istruttivo rappresentare l'ingranamento di queste
due ruote, che naturalmente
le riconduce nella stessa scala (figura \ref{fig:1760}).

\noindent Siamo finalmente giunti alla fine del nostro percorso che, rispettando
i vincoli ``naturali'' che a mano a mano si sono palesati, ha portato a una
soluzione che appare quasi unica. Rimane ancora una verifica da effettuare,
che sottintende un precetto forte (l'ultimo): l'ingranaggio di due ruote
dentate deve essere in grado di trasmettere il moto con continuit\`a. Questo
significa che le coppie di denti a contatto devono sempre essere in numero
 maggiore
dell'unit\`a.
\begin{figure}[hbt]
\begin{center}
\includegraphics[width=1.0\textwidth]{part2/ruote/FIG/ruote/1760.pdf}
\begin{picture}(0,0)(200,0)
\scriptsize{
\put(180,197){$A$}
\color{white}
\put(280,160){$z_1=17$}
\put(50,160){$z_2=60$}
\put(331,139){$O_1$}
\put(151,145){$P$}
\put(134,97){$B$}
\put(122,83){$p_b$}
\put(126,73){$\mu$}
\put(100,46){$r_{b_2}$}
\color{black}
\put(146,46){$r_{p_2}$}
\put(198,31){$r_{p_1}$}
\put(216,31){$r_{b_1}$}
\put(158,23){$r_{t_2}$}
\put(181,23){$r_{t_1}$}
}
\end{picture}
\end{center}
\vskip -5mm
\caption{ \em
Ingranaggio con $z_1=17$, $z_2=60$, $\theta=20^{\circ}$.
} 
\vskip -3mm
\label{fig:1760}
\end{figure}
\noindent Con riferimento alla figura \ref{fig:1760}, ricordando che 
i punti di contatto (coniugati) tra due fianchi di evolvente giacciono
sulla perpendicolare alle evolventi stesse in tali punti, possiamo affermare
che i contatti tra i denti avvengono solo sulla retta $\mu$, tangente
alle due circonferenze di base di raggi $r_{b_1}$ e $r_{b_2}$, e chiamata
{\em retta delle pressioni}\index{retta delle pressioni}.
Pi\`u in dettaglio, le posizioni dei contatti saranno interne alle due 
circonferenze di testa delle ruote, le quali delimitano il tratto $AB$ che,
nella figura \ref{fig:1760}, \`e rappresentato in rosso. Affinch\'e
la trasmissione del moto avvenga con continuit\`a, sul segmento $AB$ devono
essere presenti uno o pi\`u punti di contatto. Sulla retta delle pressioni
i profili dei denti si succedono col passo che si misura sulle circonferenze
di base, dalle quali ``nascono'' le evolventi, destre o sinistre,
 e che naturalmente coincide con il passo della  successione dei fianchi, 
destri o sinistri, dei denti dell'utensile, misurato in direzione normale ai
fianchi stessi. Tale misura, indicata in figura \ref{fig:dentiera_maag}, vale
$p_b=\pi m \cos \theta$. Anche questa lunghezza \`e riportata in 
figura \ref{fig:1760}, giusto per
un confronto visivo, mediante un segmento verde.
Riepilogando, la trasmissione del moto sar\`a continua se il 
{\em fattore di ricoprimento}\index{fattore di ricoprimento} $f_c$, dato
dal rapporto tra la lunghezza del segmento $AB$ e il passo di base, \`e
maggiore dell'unit\`a: $f_c=(AB)/p_b>1$.
 La lunghezza del segmento $PB$ (come del resto quella del segmento
$PA$) si determina applicando il teorema del coseno al triangolo $\triangle{O_1BP}$ (oppure $\triangle{O_2AP}$). Per $(PB)$ si ha
\begin{equation}
(PB)^2  +2r_{p_1}\sin( \theta)(PB)+r_{p_1}^2 - r_{t_1}^2=0\,.
\label{eq:segpb2}
\end{equation}
\noindent Pertanto, indicando le lunghezze $(PA)$ e $(PB)$ semplicemente
con $\eta$ e togliendo i pedici che identificano una delle due ruote, possiamo 
scrivere
\begin{equation}
\eta= -r_p\sin( \theta)\pm \sqrt{r_p^2\sin^2(\theta)-r_p^2 + r_t^2}\,.
\label{eq:segpb3}
\end{equation}
\noindent Introducendo le grandezze legate al proporzionamento normale 
e facendo il rapporto tra $\eta$ e il passo base $p_b$ abbiamo
\begin{equation}
\frac{\eta}{p_b}= \frac{-(z/2)\sin( \theta)\pm \sqrt{(z/2)^2\sin^2(\theta)+ z+1}}{\pi \cos (\theta)}\,.
\label{eq:segpb3_1}
\end{equation}
\noindent La formula \ref{eq:segpb3_1} fornisce,
con  $z=17$, un fattore di ricoprimento,
nel tratto di competenza $PB$, $\underline f_c=\eta/p_b=0.75$. La qual cosa  significa
che, nel caso peggiore (ammissibile per non avere sottotaglio),
cio\`e quando l'ingranaggio \`e costituito da due pignoni da diciassette denti,
essi possono ingranare tra loro con continuit\`a essendo, in questo caso,
$f_c=1.5$.

\section{Ruote Corrette mediante Spostamento del Profilo I} \label{ruote-corr1}

Le proporzioni della dentiera di riferimento di figura
\ref{fig:dentiera_riferimento} risultano sicuramente ragionevoli ma, 
rappresentando un compromesso esteso a tutte le ruote, qualsiasi
sia il loro numero di denti e quello della ruota accoppiata,
si intuisce facilmente che esse potrebbero non rappresentare
la scelta ottimale.
\begin{figure}[b]
\begin{center}
\includegraphics[width=1.0\textwidth]{part2/ruote/FIG/ruote/1160.pdf}
\begin{picture}(0,0)(200,0)
\scriptsize{
\put(180,197){$A$}
\color{white}
\put(280,155){$z_1=11$}
\put(50,155){$z_2=60$}
\put(285,138){$O_1$}
\put(151,145){$P$}
\put(123,82){$p_b$}
\put(124,71){$\mu$}
\put(99,46){$r_{b_2}$}
\color{black}
\put(135.5,103.5){$B$}
\put(146,46){$r_{p_2}$}
\put(213,44){$r_{p_1}$}
\put(224,50){$r_{b_1}$}
\put(158,23){$r_{t_2}$}
\put(190,23){$r_{t_1}$}
}
\end{picture}
\end{center}
\vskip -5mm
\caption{\em
Ingranaggio con $z_1=11$, $z_2=60$, $\theta=20^{\circ}$ senza correzioni.
} 
\vskip -3mm
\label{fig:1160}
\end{figure}
\noindent In figura \ref{fig:1160} \`e riportato l'ingranaggio di un pignone
con $z_1=11$ denti e $z_2=60$ denti. Si nota subito che i denti del pignone
si presentano scavati alla base. Ci\`o era ampiamente prevedibile in quanto
la \ref{eq:zminrif}, pur considerando l'arrotondamento dei taglienti
dell'utensile creatore, non permette di scendere sotto il numero 
$z_{\rm min}=17$, senza che il sottotaglio si manifesti. La scarnitura del
dente alla base \`e una circostanza molto grave? Certamente la resistenza
a flessione del dente stesso ne risente. E dal punto di vista della nostra
disciplina? Apparentemente, in quest'ambito, le cose vanno meglio: la cinematica
dell'ingranaggio \`e comunque corretta. Ma l'impegno dell'evolvente di cerchio
fino alla sua radice fa sorgere un'altra preoccupazione, dovuta al fatto che
in quel punto, com'\`e noto, il raggio di curvatura del profilo \`e pari a zero,
e da ci\`o possono emergere altre problematiche di carattere strutturale.
Per\`o, la grande produttivit\`a delle dentatrici di ruote a evolvente,
la relativa semplicit\`a di costruzione degli utensili necessari,
la possibilit\`a di affilare tali utensili senza compromettere la loro
forma ``operativa'' ci convincono, in un ragionamento un po' fantastico,
che il mondo della meccanica avrebbe accettato tranquillamente anche le
ruote sottotagliate qualora, a tale problema, non ci fosse stato rimedio
 e anzi, in talune applicazioni, cui faremo cenno alla
fine del prossimo paragrafo, tali ruote si accettano appunto.
Ma il rimedio esiste, ed \`e uno di quei ``toccasana'' che non comporta
spese, al contrario, oltre a determinare la scomparsa del sottotaglio,
le {\em correzioni tramite
lo spostamento del profilo}\index{correzione!profilo} portano una serie
di vantaggi tali da essere impiegate anche quando il sottotaglio non si
manifesta affatto.
In aggiunta, mentre da un punto di vista didattico le dentature
corrette presentano per lo studente un ulteriore (l'ultimo) piccolo
grattacapo, nella pratica esse si progettano e si eseguono con estrema
disinvoltura: se si dovesse progettare l'ingranaggio di figura \ref{fig:1760}
e il pignone di $z_1=17$ denti fosse la ruota motrice
ben difficilmente esso non verrebbe corretto.
\begin{figure}[hbt]
\begin{center}
\includegraphics[width=1.0\textwidth]{part2/ruote/FIG/ruote/1160c_i0.pdf}
\begin{picture}(0,0)(200,0)
\scriptsize{
\color{white}
\put(170,172){$A$}
\put(280,155){$z_1=11$}
\put(50,155){$z_2=60$}
\put(285,138){$O_1$}
\put(148,144){$P$}
\put(123,82){$p_b$}
\put(124,71){$\mu$}
\put(97,46){$r_{b_2}$}
\put(138,86){$B$}
\color{black}
\put(146,46){$r_{p_2}$}
\put(213,38){$r_{p_1}$}
\put(225,49){$r_{b_1}$}
\put(157,23){$r_{t_2}$}
\put(190,23){$r_{t_1}$}
}
\end{picture}
\end{center}
\vskip -5mm
\caption{\em
Ingranaggio con $z_1=11$, $z_2=60$, $\theta=20^{\circ}$, con correzioni $x_1=0.5$ e $x_2=-0.5$.
} 
\vskip -3mm
\label{fig:1160c_i0}
\end{figure}
Ricordiamo che nelle due figure \ref{fig:dentiera_riferimento} e
\ref{fig:dentiera_maag} abbiamo riportato una retta tratteggiata indicata
con $\lambda_t$, chiamata {\em primitiva di taglio}\index{primitiva!di taglio}, distante dalla primitiva della
dentiera di una quantit\`a $c=m x$. Quando le ruote vengono tagliate facendo
rotolare $\lambda_t$ sulla primitiva teorica delle ruote stesse con $c \ne 0$
esse si dicono {\em corrette}\index{ruote!corrette}. Riferendoci all'utensile {\em Maag} (ma
la questione rimane identica anche per utensili creatori a vite), si tratta
semplicemente di spostare la dentiera di una quantit\`a $c$ verso l'esterno
della ruota da
tagliare se $c >0$, oppure verso l'interno qualora la correzione fosse negativa.
In figura \ref{fig:1160c_i0} rappresentiamo l'ingranaggio
$z_1=11$ e $z_2=60$, le cui ruote sono state realizzate spostando la
primitiva di taglio dei due 
utensili delle quantit\`a adimensionali $x_1=0.5$ e $x_2=-0.5$. Anche i raggi
delle due circonferenze di testa dovranno subire
un incremento $\delta r_{1_t}=0.5 m$
e un decremento  $\delta r_{2_t}=-0.5 m$. 
Notiamo con facilit\`a il cambiamento di forma della base dei denti del 
pignone, che non palesano pi\`u alcun sottotaglio, a fronte della riduzione
modesta e
probabilmente accettabile dello spessore alla base di quelli dell'altra ruota.
Come riportato in figura \ref{fig:1160c_i0},
l'angolo di pressione vale ancora $\theta=20^{\circ}$, mantenendo 
il valore di quello dell'utensile, e l'interasse vale $i=(z_1+z_2)m/2$.
Queste ultime affermazioni sono meno ovvie
rispetto a ci\`o che a prima vista si potrebbe ritenere. In particolare
la seconda (dalla quale per\`o dipende la prima), e cio\`e che le ruote
corrette possano formare ingranaggi senza variazione di interasse rispetto
a quello teorico.
Infatti, quando la somma $c_1+c_2 \ne 0$, l'interasse di funzionamento,
cio\`e la distanza che deve separare i centri delle ruote corrette affinch\'e
ingranino correttamente, non corrisponde al valore di tale somma aggiunta
all'interasse teorico, come dimostreremo a breve. 
Intanto osserviamo ancora che l'ingranaggio di figura \ref{fig:1160c_i0} presenta
fattore di ricoprimento $f_c > 1$, fatto che andrebbe calcolato tramite
la formula \ref{eq:segpb2} e le successive, opportunamente modificate, anche se,
in questa sede, ci accontentiamo di ci\`o che si manifesta visivamente. Appare
anche chiaro che il tratto $PA$, che si chiama {\em accesso},\index{accesso}
risulta alquanto pi\`u corto di $PB$, che si chiama {\em recesso}\index{recesso}.
I due nomi
presuppongono che il pignone funga da motore, altrimenti tali nomi devono essere
invertiti. Senza entrare nel merito delle considerazioni dinamiche che
supportano l'affermazione che formuliamo, affermiamo che \`e sempre conveniente
avere un tratto di accesso ridotto rispetto a quello di recesso.
L'ingranaggio di figura \ref{fig:1160c_i0} \`e pertanto formato da due ruote
corrette mediante lo spostamento dei loro profili in due sensi opposti,
e il filo logico sottinteso alle correzioni
potrebbe essere cos\`i riassunto: si corregge il pignone, in senso positivo, 
fino alla eliminazione del sottotaglio; lo stesso valore della correzione,
cambiato di segno, si applica all'altra ruota, sperando di non spostare su
quest'ultima il sottotaglio e mantenendo invariato l'interasse teorico.
Anche se questa soluzione appare elegante, di solito non viene scelta.
Come chiarito anche in \cite{righettini}, il ragionamento 
sviluppato appena sopra sembra presupporre che il proporzionamento normale
rifletta condizioni di funzionamento ottimali e pertanto sia opportuno
derogare alle sue leggi il meno possibile. Gli spostamenti dei profili si
dimostrano invece vantaggiosi a prescindere dal problema del sottotaglio.
\`E tutt'altro che infrequente progettare le ruote dentate sacrificando
l'interasse teorico a favore dell'ottenimento
di altre caratteristiche, oppure imponendo un 
interasse funzionale a geometrie particolari. Quest'ultimo
caso si presenta spesso
per gli assi paralleli dei cambi delle automobili sui quali si devono
calettare quattro o pi\`u coppie di ruote dentate, sempre in presa,
aventi somme
dei denti leggermente diverse tra loro e caratteristiche geometriche tali
(si tratta quasi sempre di ruote elicoidali)
da non poter garantire un interasse comune senza essere corrette.
 Non entreremo nell'ambito dei criteri
di scelta delle correzioni che si apportano per ottimizzare, sotto svariati
aspetti, il funzionamento degli ingranaggi. Diciamo solamente che il caso
di figura \ref{fig:1160c_i0} potrebbe essere approcciato da un progettista 
correggendo solo il pignone e lasciando la ruota da sessanta denti con geometria
normale. Di seguito analizzeremo appunto questo caso, dove la correzione sar\`a
apportata solamente sul pignone, con la conseguenza di dover 
determinare un nuovo angolo di
pressione e un nuovo interasse, diversi da quelli normali.


\section{Ruote Corrette mediante Spostamento del Profilo II}\index{correzione!con variazione interasse}

\noindent Ricordiamo che, durante il taglio delle ruote corrette,
nonostante lo spostamento della primitiva della dentiera, tale primitiva
dovr\`a rotolare sulla primitiva teorica della ruota da tagliare, inviluppando,
in questo modo, evolventi di circonferenze di base ancora di raggio
$r_{1_b}= z_1 m /2 \cos \theta$ e $r_{2_b}= z_1 m /2 \cos \theta$. Di
conseguenza, qualsiasi siano le correzioni dei profili, il
{\em passo base}\index{passo!base} $p_b$, gi\`a riportato in figura
\ref{fig:dentiera_maag}, non cambia.
La figura \ref{fig:1160c_ix} riporta l'ingranaggio di due ruote
dentate, sempre con $z_1=11$ denti e $z_2=60$ denti. Come si legge nella didascalia,
la correzione positiva \`e stata applicata solamente al pignone, con $x=0.5$,
mentre la seconda ruota non \`e corretta. Prima di rispondere alla legittima
domanda: come sono stati determinati l'interasse effettivo di funzionamento
e il nuovo angolo di pressione? Riteniamo opportuno ribadire che
la teoria, e in particolare le premesse alle formule \ref{eq:nuove_primitive},
ci assicurano che i nostri profili a evolvente sono ancora coniugati.
In quest'ambito, quello cio\`e delle correzioni con variazione di
 interasse, si hanno
due casi emblematici. 
\begin{figure}[t]
\begin{center}
\includegraphics[width=1.0\textwidth]{part2/ruote/FIG/ruote/1160c_ix.pdf}
\begin{picture}(0,0)(200,0)
\scriptsize{
\color{white}
\put(270,40){$z_1=11$}
\put(40,120){$z_2=60$}
\put(270,18){$O_1$}
\put(39,28){$r_{b_2}$}
\put(96,78){$r_{{p_2}_f}$}
\put(182,185){$l_1$}
\put(98,140){$l_2$}
\put(167,178){$\mu$}
\put(258,187){$\zeta$}
\put(262,175){$\gamma(\theta_0)$}
\put(264,165){$\gamma(\theta_f)$}
\put(217,83){$\zeta$}
\put(220,54){$\gamma(\theta_0)$}
\put(227.5,27){$\gamma(\theta_f)$}
\put(291,106){$\delta_1$}
\color{black}
\put(232,185){$r_{b_1}$}
\put(120,210){$r_{t_2}$}
\put(97,167){$p$}
\put(162,219){$r_{t_1}$}
\put(147,149){$p$}
\put(354.5,189){$r_{{p_1}_f}$}
\put(329.5,199.5){$r_{{p_1}_0}$}
}
\end{picture}
\end{center}
\vskip -5mm
\caption{\em
Ingranaggio con $z_1=11$, $z_2=60$, $\theta_0=20^{\circ}$, $\theta_f=22^{\circ}$, $i_0=142$, $i_f=143.9$, con le correzioni $x_1=0.5$ e $x_2=0$.
} 
\vskip -3mm
\label{fig:1160c_ix}
\end{figure}
Esporremo per primo quello che porta 
all'individuazione dell'interasse
di funzionamento $i_f$ e del corrispondente angolo di pressione $\theta_f$,
date le due correzioni arbitrarie delle ruote. 
Il secondo caso fa invece riferimento
all'individuazione di quelle correzioni necessarie a produrre un interasse
di funzionamento dato.
Riprendiamo dalla figura \ref{fig:1160c_ix} e cerchiamo di individuare
un criterio, date le correzioni delle due ruote, per potere stabilire
l'effettivo interasse di funzionamento e l'effettivo angolo di pressione.
Da un punto di vista cinematico, l'ingranaggio della figura appena citata
funzionerebbe egregiamente anche con l'interasse dato dalla somma dei
raggi delle primitive teoriche pi\`u la somma delle correzioni:
$i_g=(z_1 +z_2) m /2 +(x_1+x_2)m$. Se montassimo in questo modo le ruote,
noteremmo per\`o un certo gioco che, in generale, non pu\`o essere tollerato
nelle trasmissioni di potenza, onde evitare che i denti ``sbattano''
\index{sbattimento} a causa delle eventuali (frequentissime nelle varie
applicazioni) inversioni della direzione del flusso di
potenza. Ecco quindi il criterio cercato: le ruote devono essere montate senza gioco\footnote{
Un minimo gioco tra le ruote deve sempre sussistere, per evitare ai cuscinetti
di supporto di funzionare in regime di carico gravoso e non previsto,
ma soprattutto per consentire la lubrificazione dei denti. La 
quantificazione di tale gioco \`e una questione delicata che dipende da
un numero rilevante di fattori: condizione di impiego della coppia di ruote,
precisione e finitura 
delle ruote stesse, tipo di lubrificazione, e molti altri. Rimandiamo
gli studenti desiderosi di approfondire alla letteratura specializzata, 
come ad esempio \cite{henriot}.
}.
Riferendoci ancora alla figura \ref{fig:1160c_ix}, quanto appena stabilito
si traduce nella seguente relazione $p=l_1 + l_2$. Questa relazione afferma
che il passo sulle primitive di funzionamento $p$, il quale ovviamente,
affinch\'e le ruote possano ingranare, deve essere lo stesso
su entrambe, \`e dato dalla
somma dello spessore dei denti sulle primitive stesse.
\`E questo uno dei casi in cui scopriamo l'utilit\`a della rappresentazione
analitica dell'evolvente e, in particolare, della \ref{eq:gamma}.
Cominciamo col calcolare la lunghezza dell'arco della primitiva teorica, o
di riferimento,
che attraversa un dente. A questo proposito notiamo che, quando la dentiera viene
spostata della quantit\`a $c=mx$, il vano sulla primitiva di taglio, che sar\`a
pari allo spessore del dente sulla primitiva di riferimento della ruota, vale
\begin{equation}
l_0= m \left( \tfrac{\pi}{2} + 2 x \tan(\theta_0) \right)\,,
\label{eq:l0}
\end{equation}
\noindent dove l'addendo di destra non \`e altro che la base di un triangolo
isoscele di altezza pari a $c=mx$ e angolo di apertura uguale al 
doppio dell'angolo dei profili della dentiera. La \ref{eq:l0} va specializzata
per le due ruote, in quanto le correzioni $x_1$ e $x_2$ saranno,
in generale, 
diverse. La conoscenza di $l_{0_1}$ e $l_{0_2}$ ci fornisce con facilit\`a
il valore degli angoli sottesi da tali archi:
\begin{equation}
\delta_1 = l_{0_1}/r_{{p_1}_0}\;\;\; {\rm e} \;\;\;\;
\delta_2 = l_{0_2}/r_{{p_2}_0}\,.
\label{eq:delta}
\end{equation}
\noindent Le lunghezze degli archi $l_1$ e $l_2$ si otterranno considerando
gli effettivi valori del raggio primitivo di funzionamento delle due ruote 
e l'aumento degli angoli $\delta$ dovuti all'angolo $\zeta$, come riportato
in figura \ref{fig:1160c_ix}.
Quanto a $\zeta$, sempre dalla figura, abbiamo
\begin{equation}
\zeta= \gamma(\theta_0)-\gamma(\theta_f)\,,
\label{eq:zeta}
\end{equation}
\noindent e tale valore\footnote{
Per la coordinata polare dell'evolvente abbiamo mantenuto la nomenclatura
esposta nella formula \ref{eq:gamma}, $\gamma()$, bench\'e
il nome pi\`u comunemente usato nella letteratura sia {\em involute
function}\index{involute function}. Pertanto, con un'abbreviazione a nostro
parere infelice, abbiamo
\begin{equation}
 {\rm inv}(\theta) = \tan(\theta)-\theta \,,
\label{eq:inv}
\end{equation}
come si trova, ad esempio, in \cite{henriot}.
} sar\`a lo stesso per le due ruote. Finalmente ricaviamo
\begin{equation}
l_1= r_{{p_1}_f} (\delta_1-2\zeta)\;\;\; {\rm e} \;\;\; l_2= r_{{p_2}_f} (\delta_2-2\zeta)\,,
\label{eq:l1l2}
\end{equation}
\noindent quindi, per il passo $p=l_1+l_2$, misurato sulle primitive di 
funzionamento possiamo scrivere
\begin{multline}
p= l_1+l_2= 
2 r_{{p_1}_f}[ \frac{1}{z_1} (\pi/2 + 2 x_1 \tan(\theta_0)) + \zeta] +\\
+2 r_{{p_2}_f}[ \frac{1}{z_2} (\pi/2 + 2 x_2 \tan(\theta_0)) + \zeta]\,.
\end{multline} 
\noindent A valle di alcune semplici sostituzioni, tenendo presente 
che $r_{{p_1}_f}= z_1 p/\pi$ e $r_{{p_2}_f}= z_2 p/\pi$,
esplicitando $\zeta$ tramite la \ref{eq:zeta}, otteniamo
\begin{equation}
\gamma(\theta_f)=2 \frac{x_1+x_2}{z_1+z_2}\tan(\theta_0) +\gamma(\theta_0)\,.
\label{eq:thetaf}
\end{equation}
\noindent La funzione $\gamma(\theta)$, col nome citato nella \ref{eq:inv}, \`e
riportata, tabulata, in diversi trattati: citiamo \cite{henriot} dove,
dalla pagina 72 alla pagina 79, $\theta$ spazia da $10^{\circ}$ a $50^{\circ}$. Tale funzione \`e
regolare e monotona crescente, pertanto, senza scomodare metodi di ricerca
degli zeri recanti nomi altisonanti, il comune metodo di bisezione
(che abbiamo usato per ottenere le nostre figure) porta,
con estrema rapidit\`a, a soluzioni apprezzabili della \ref{eq:thetaf}. Una
volta individuato il valore di $\theta_f$, che sar\`a l'effettivo angolo di
pressione, ricordando che in ogni caso
il raggio della circonferenza di base non cambia,
come abbiamo gi\`a altrove sottolineato, avremo
\begin{equation}
r_{{p_1}_f} \cos(\theta_f)=r_{{p_1}_0} \cos(\theta_0)\;\;\; {\rm e} \;\;\;
r_{{p_2}_f} \cos(\theta_f)=r_{{p_2}_0} \cos(\theta_0)\,,
\end{equation}
\noindent dalle quali si ottiene
\begin{equation}
i_f= r_{{p_1}_f} + r_{{p_2}_f}=  (r_{{p_2}_0}+ r_{{p_1}_0}) \frac{ \cos(\theta_0)}{\cos(\theta_f)}= m(z_1+z_2)\frac{ \cos(\theta_0)}{2 \cos(\theta_f)}\,.
\label{eq:if}
\end{equation}
\begin{wrapfigure}{r}{0.54\textwidth}
      \begin{center}
      \includegraphics[width=0.52\textwidth]{part2/ruote/FIG/ruote/interasse.pdf}
     \end{center}
\begin{picture}(0,0)(-63,1)
\scriptsize{
        \put(-46,119){$\ni$}
        \put(78,19){$1-i_f/i_0$}
}
\end{picture}
\vskip -4.1mm
        \caption{\em Rapporto tra interasse effettivo $i_f$ e interasse
intuitivo $i_0+m(x_1+x_2)$.}
     \label{fig:interasse}
\end{wrapfigure}
\noindent Ci sembrano opportune le seguenti due considerazioni. La prima:
$\cos(\theta_f)$, e perci\`o anche l'interasse di funzionamento $i_f$, dipende
soltanto dalla somma delle correzioni quindi, come gi\`a ammesso nel
precedente paragrafo, se $x_1 + x_2=0$ risulta $\theta_f=\theta_0$ e $i_f=i_0$,
 cio\`e angolo di pressione e interasse rimangono inalterati.
 La seconda osservazione pu\`o essere elaborata
come risposta alla domanda: l'interasse di funzionamento sar\`a dato da
 $i_f=i_0+m(x_1+x_2)$? La risposta \`e no, e per convincerci di ci\`o
studiamo l'andamento del rapporto tra l'interasse effettivo e la somma delle
correzioni addizionata all'interasse teorico, che chiamiamo $\ni$
\begin{equation}
\ni=\frac{i_f}{i_0+m(x_1+x_2)}\,,
\label{eq:ni}
\end{equation}
\noindent in funzione
della variazione adimensionale dell'interasse $\delta_i=1-i_f/i_0$.
\noindent La figura \ref{fig:interasse} mostra quindi il grafico di $\ni$
al variare di $\delta_i$, che per $0<=\delta_i<=0.1$  
evidenzia una riduzione di $i_f$ rispetto
all'interasse che si sarebbe individuato a intuito $i_0+m(x_1+x_2)$.
Tale situazione dovr\`a, peraltro, risultare compatibile con i giochi tra testa e piede
 dei denti e richiede pertanto una specifica verifica.
\vskip 3mm
\noindent Veniamo ora al secondo dei due casi emblematici
per le correzioni con variazione di interasse, che \`e quello 
in cui le correzioni incognite devono
risultare compatibili con un interasse dato. Stabilito quindi il valore di $i_f$,
dalla \ref{eq:if} ricaviamo $\theta_f$, come gi\`a accennato con la 
formula \ref{eq:nuove_primitive}.
Dalla \ref{eq:thetaf} ricaviamo poi,
con facilit\`a, la somma delle correzioni $x_1+x_2$ ammissibile con 
l'interasse di funzionamento assegnato. 
Il tema della ripartizione delle
correzioni sulle due ruote \`e complesso e rientra nell'ambito dello
studio dei criteri di scelta delle correzioni stesse. Ribadiamo che
decidere di dare al pignone la minima correzione necessaria a scongiurare
il sottotaglio dei denti e quindi regolarci di conseguenza circa la correzione
della seconda ruota, magari con una correzione di segno opposto, cos\`i
da lasciare inalterato l'interasse di progetto,
 \`e molto riduttivo e anacronistico. 
Una guida alla scelta delle correzioni tramite percorsi volti
 all'ottimizzazione degli ingranaggi (percorsi e criteri
 a volte in contesa
tra loro) \`e riportata in \cite{righettini}. Riteniamo per\`o fuori
dallo scopo di queste note approfondire ulteriormente questo tema che,
come abbiamo anticipato, \`e materia decisamente specialistica.
\begin{figure}[hbt]
      \begin{center}
      \includegraphics[width=0.65\textwidth]{part2/ruote/FIG/ruote/forma_dente.pdf}
     \end{center}
\begin{picture}(0,0)(-63,1)
\scriptsize{
\color{white}
        \put(30,53){$x=-0.3$}
        \put(37,65){$x=0$}
        \put(31,79){$x=0.3$}
        \put(31,93){$x=0.6$}
        \put(25,110){$x=1.086$}
}
\end{picture}
\vskip -5mm
        \caption{\em
Variazione della forma dei denti di una ruota con $z=17$ e correzioni $x=-0.3$, $x=0$, $x=0.3$, $x=0.6$ e $x=1.086$.
}
     \label{fig:forma_dente}
\end{figure}

\noindent Riepiloghiamo
l'argomento della correzione del profilo dei denti sottolineando tre punti
che ci sembrano importanti. 
Il primo riguarda la forma del dente, sensibilmente modificata
dallo spostamento del profilo,
come riportato in figura \ref{fig:forma_dente}.
Al crescere della correzione si passa infatti dalla condizione di
 sottotaglio a quella
che vede l'irrobustimento della 
base dei denti, fino all'appuntimento dei denti stessi quando si 
adottano valori di correzione eccessivi.
Nella figura \`e appunto riportato, per una ruota con $z=17$, il
caso di appuntimento del dente, quando le due evolventi si incrociano esattamente
sulla circonferenza di testa. La situazione dei denti a punta, ma anche con
spessore di testa troppo piccolo, \`e da evitarsi per ragioni di resistenza
meccanica.

\noindent Il secondo punto saliente riguarda
 il fattore di ricoprimento $f_c$. Esso, in sostanza, dipende
dalla somma delle correzioni $x_1 + x_2$, e 
diminuisce al crescere di questa somma, mentre non \`e molto sensibile ai valori
delle singole correzioni.
Con $x_1+x_2=0$, il fattore di ricoprimento \`e compreso tra $1.4<f_c<2$, che
ovviamente \`e anche l'intervallo dove spazia tale fattore per le ruote
non corrette.
Adottando correzioni la cui somma \`e negativa si possono
ottenere elevati valori del fattore di ricoprimento, a beneficio della
silenziosit\`a dell'ingranaggio.
Quando invece $x_1 + x_2 > 0$ il fattore di ricoprimento diminuisce e questo
pu\`o arrivare a compromettere
 la continuit\`a della trasmissione. Pertanto, nel caso di correzioni a
somma algebrica maggiore di zero, cio\`e con aumento dell'interasse,
bisogna sempre verificare il valore di $f_c$.

\noindent Il terzo punto riguarda l'allontanamento del segmento dei
 contatti $AB$ dalle ruote con correzione positiva. Ridurre la porzione
di accesso sulla linea dei contatti porta un netto
miglioramento del rendimento degli ingranaggi. Per tale motivo ai pignoni
motori, che sono una buona parte di quelli che si costruiscono, si applicano
volentieri correzioni positive che, eliminando il sottotaglio e irrobustendo
la base del dente a favore della sua resistenza strutturale, ne migliorano anche
il rendimento diminuendo il tratto di accesso, come mostra il confronto
tra i tratti $PA$ e $PB$ della figura \ref{fig:1160c_i0}.
Le cose vanno diversamente per i moltiplicatori di velocit\`a, dove la ruota
motrice non necessiterebbe, di per s\'e, correzioni positive con lo scopo
di eliminare il sosttotaglio. In tali casi, i diagrammi che ci guidano
nel ripartire la somma delle correzioni $x_1+x_2$ sulle due ruote possono
allocare una correzione positiva maggiore alla ruota motrice, cio\`e,
in questo caso, quella con maggiore numero di denti, come mostrato
in \cite{righettini}, pag. 52, e, in alcuni casi, persino 
tollerare un leggero
sottotaglio sulla ruota condotta, quella cio\`e con numero di denti minore.

\section{Note Conclusive sulle Ruote Dentate}

A conclusione di questo lungo capitolo, proponiamo un'analisi
ad elevatissima
 velocit\`a del percorso che, altrove, abbiamo seguito con la giusta
cadenza in modo da poter essere sviluppato senza incertezze. Tale traccia
ci ha condotto, quasi attraverso alcuni passaggi obbligati, alle ruote
con denti profilati a evolvente che dominano
interamente il panorama degli ingranaggi. Ad esempio, anche se non
ne abbiamo fatto cenno esplicito in questo lavoro, le ruote con denti ad asse
elicoidale, cio\`e le {\em ruote elicoidali}\index{ruote!elicoidali},
 pur possedendo caratteristiche specifiche, che un tecnico
deve conoscere perfettamente per poterle inserire correttamente 
nel progetto di una macchina, non
presentano, dal punto di vista della loro cinematica, nulla di diverso
dalle ruote che abbiamo studiato. Anzi, la loro fabbricazione si ottiene
semplicemente inclinando di un certo angolo l'utensile {\em Maag} di cui abbiamo
parlato diffusamente (oppure l'asse della ruota da tagliare).
Quindi profilo a evolvente ovunque, sulle ruote
coniche, che possono trasmettere il moto tra alberi concorrenti, sulle dentature
interne, e cos\`i via. Tale profilo si ottiene ``automaticamente'', come abbiamo visto,
dal processo di inviluppo tramite il taglio con utensili molto semplici.
In pratica, l'utilizzo dell'utensile pi\`u semplice da realizzare e da
affilare, cio\`e della dentiera riportata in figura
\ref{fig:dentiera_maag},
 ci fornisce direttamente e senza sforzo le ruote dentate a evolvente,
assortibili, facilmente adattabili a modeste variazioni di interasse,
facili da correggere per il raggiungimento di determinati scopi, come
l'eliminazione del sottotaglio dei denti.

\noindent E questa \`e
la conclusione alla quale ci premeva arrivare. Il successo di una 
tecnologia \`e spesso legato alla sua semplicit\`a e
alla sua attuabilit\`a.
Per questo tipo di  ruote non ha gran senso neppure lo studio numerico
del loro profilo come invece risulta d'obbligo,
da quando i calcolatori sono entrati a far parte degli strumenti
del progettista,  nella profilatura 
delle camme. Infatti, come abbiamo ormai troppe volte ripetuto,
l'ottenimento del loro sofisticato profilo
segue un processo
indiretto rappresentato dall'inviluppo creato dall'utensile dentiera,
processo che abbiamo usato anche noi per le nostre figure.
Anni fa, soprattutto
in fase prototipale, alcune
ruote dentate si lavoravano mediante l'impiego di frese a disco
che, scavando nel cilindro grezzo della futura ruota dentata
il vuoto tra un dente e il successivo,
modellavano i fianchi dei denti stessi.
Pu\`o darsi che qualche meccanico affezionato
alla {\em fresatura di forma}\index{fresatura di forma}
dei denti si possa ancora trovare presso
le officine che eseguono riparazioni e restauro di oggetti particolari.
La forma delle frese impiegate in questo procedimento
dipende naturalmente dal modulo della dentatura che si desidera ottenere,
ma anche, come \`e ovvio, dal numero dei denti della ruota.
Anzi, a ben vedere, sarebbe necessario impiegare 
una fresa per ogni ben preciso numero di denti.
Nella pratica per\`o bastano una decina di
utensili (di un certo modulo) per avere risultati soddisfacenti con
$z$ che spazia in un largo intervallo. Tagliando in
questo modo  il profilo dei denti
si perde per\`o completamente il controllo circa la 
capacit\`a della dentatura ottenuta di ospitare correttamente i denti
dell'altra ruota. \`E qui infatti che nasceva il problema dell'interferenza
propriamente detta,
la quale invece si manifesta come sottotaglio
dei denti quando questi si ottengono da un inviluppo. Rispetto alla
fresatura di forma,
il processo di taglio con creatore o dentiera {\em Maag} \`e
talmente pi\`u comodo, pi\`u preciso, pi\`u flessibile (si pensi
alle correzioni) da avere dato, dopo la sua comparsa,
una propulsione violenta a questo settore e allo sviluppo
della meccanica dello scorso secolo. Ma
cosa ci riserva il futuro? Si potrebbe pensare di fabbricare le ruote
dentate tagliandone direttamente il profilo con macchine
a controllo numerico, come potrebbe fare un'elettroerosione a filo?
Oppure, si potrebbe concepire di ottenerle tramite deposizione controllata
di materiale?  \`E chiaro che se si considerano queste opportunit\`a,
l'ottenimento dell'evolvente non \`e automatico e dovremmo
basarci su opportuni codici di calcolo che ci forniscano la geometria del
dente stesso. Su questo campo da gioco, poco conosciuto,
sospettiamo per\`o che il profilo a evolvente
avrebbe pi\`u di un concorrente.

\endinput

\chapter{Cenni ai Sistemi Articolati}

\section{Introduzione}

\noindent Due o pi\`u corpi rigidi possono essere collegati tra loro mediante
{\em coppie rotoidali}\index{coppia!rotoidale}\footnote{In questi appunti non si descrivono le coppie rigide 
elementari; quale esempio di tali scorrimenti di superfici di corpi rigidi l'una
sull'altra citiamo la coppia rotoidale che collega le due lame delle comuni forbici
mediante un perno, precisando che le superfici combacianti sono quella del cilindro
del perno e le due superfici cilindriche dei fori che lo ospitano. Rimandiamo  il
lettore che desideri approfondire l'argomento 
a \cite{sesini1}, pagg. 22-24, o anche \cite{martin}, pagg. 9-10, o ancora
\cite{sesini3} che tratta numerosi casi di notevole interesse.}.
\noindent Le catene cinematiche che si ottengono in questo modo prendono
il nome di {\em sistemi articolati}\index{sistemi!articolati} e tale denominazione
viene mantenuta anche nei casi in cui una o pi\`u coppie rotoidali sia sostituita
da una sua degenerazione che, portando l'asse della coppia all'infinito, diventa una
{\em coppia prismatica}\index{coppia!prismatica}.
Questa vasta famiglia di meccanismi trova estese e notevolissime applicazioni:
ogni motore a combustione interna contiene almeno un sistema articolato
(il manovellismo ordinario) e svariate
sono le applicazioni dei {\em quadrilateri articolati}\index{quadrilateri!articolati} 
nelle macchine automatiche, nelle presse, nelle macchine per cucire; l'elenco
completo delle applicazioni risulterebbe molto lungo.
Esistono sistemi articolati con coppie rotoidali i cui assi sono orientati
in direzioni sghembe tra loro oppure concorrenti in un  punto 
e qualcuno di questi {\em sistemi articolati spaziali}
ha notevoli applicazioni, come il {\em giunto di Cardano}. 
La stragrande maggioranza dei sistemi articolati presenta invece coppie rotoidali
con assi tra loro paralleli. I movimenti relativi tra i vari membri si 
realizzeranno perci\`o solo in piani paralleli tra loro e ortogonali
a tali assi e questo ci permette
di identificarli come {\em sistemi articolati piani}\index{sistemi!articolati piani},
gli unici di cui accenniamo lo studio.

\noindent Dato un sistema articolato piano, senza perdita di generalit\`a, possiamo
considerare i moti piani dei suoi membri, che in realt\`a hanno sempre luogo
in piani paralleli tra loro, ma distinti, come contenuti in un solo piano. Le tracce degli assi delle coppie rotoidali che intersecano questo piano costituiscono perci\`o
un insieme di $n$ punti, tanti quanti sono gli snodi, e tali punti sono tra loro
collegati tramite $m$ corpi rigidi. Ogni punto nel piano gode di due gradi di
libert\`a di movimento, mentre i collegamenti rigidi, che mantengono invariata
la distanza tra questi punti, assorbono un grado di libert\`a ciascuno.
Ne consegue che, quando non si presentano situazioni anomale nelle quali una o 
pi\`u parti del meccanismo risulti sovra-vincolata e magari altre parti labili,
il numero di gradi di libert\`a residuo $g$ per un sistema articolato piano 
risulta essere $g=2n - m$.

\begin{figure}[hbt]
\centering
\begin{minipage}[b]{0.30\textwidth}
\centering
\includegraphics[width=0.8\textwidth]{part2/quadri/FIG/reticolare_triangolo.pdf}
\begin{picture}(0,0)(130,0)
\scriptsize{
\put(84,85){$n=3$}
\put(84,75){$m=3$}
\put(84,65){$g=3$}
}
\end{picture}
      \caption{\em Sistema articolato con tre coppie rotoidali e tre membri.}
 \label{fig:f_triangolo}
\end{minipage}\hfill
\begin{minipage}[b]{0.30\textwidth}
\centering
\includegraphics[width=0.8\textwidth]{part2/quadri/FIG/pentalatero.pdf}
\begin{picture}(0,0)(129,0)
        \scriptsize{
\put(82,85){$n=4$, $m=5$}
\put(82,75){$g=3$}
}
\end{picture}
        \caption{\em Sistema articolato con quattro coppie rotoidali e cinque membri.}
     \label{fig:f_pentalatero}
\end{minipage}\hfill
\begin{minipage}[b]{0.30\textwidth}
\centering
\includegraphics[width=0.8\textwidth]{part2/quadri/FIG/quadrilatero.pdf}
\begin{picture}(0,0)(129,0)
        \scriptsize{
\put(82,85){$n=4$, $m=5$}
\put(82,75){$g=4$}
}
\end{picture}
        \caption{\em Sistema articolato con quattro coppie rotoidali e quattro membri.}
     \label{fig:f_quadrilatero}
\end{minipage}
\end{figure}

\noindent In figura \ref{fig:f_triangolo} \`e riportato il pi\`u semplice sistema
articolato non banale. I gradi di libert\`a residua per tale sistema sono
$g=3\times 2- 3=3$. Tre gradi di libert\`a, nel piano, sono quelli posseduti da
qualsiasi corpo rigido svincolato, e il nostro sistema comportandosi come
un corpo indeformabile  viene classificato
e studiato nella  scienza delle costruzioni come sistema reticolare. Anche il sistema
articolato riportato in figura \ref{fig:f_pentalatero} presenta un numero di gradi
di libert\`a pari a tre. Anch'esso si comporta come un corpo rigido e fa quindi 
parte delle {\em strutture reticolari}\index{strutture reticolari}.
Il sistema riportato in figura \ref{fig:f_quadrilatero} presenta invece  $g=4$,
quindi conserva un grado di deformabilit\`a interna e per questo fa parte dei
meccanismi articolati.


\section{Quadrilateri Articolati: Velocit\`a e Accelerazioni}

\noindent Il {\em quadrilatero articolato piano}\index{quadrilateri!articolati piani}
sar\`a il solo oggetto di questa breve trattazione, in questo capitolo.
La condizione pi\`u comune \`e quella che vede
uno dei suoi quattro membri fungere da telaio: in figura
\ref{fig:f_quad_telaio} il membro $AD$ \`e il telaio. 
Il membro $BC$, sul quale non si trovano coppie rotoidali vincolate
a terra, prende il nome di {\em biella}\index{biella}. Gli altri due membri, $AB$ e $DC$, 
si chiamano {\em manovelle}\index{manovella} oppure
{\em bilancieri}\index{bilanciere} a seconda che la circostanza di potere
compiere l'intero angolo giro sia o meno verificata.

\noindent Fissato il lato che funger\`a da telaio e quindi la biella, la condizione degli
altri due membri viene determinata dalla seguente
{\em regola di Grashof}\index{Grashof, regola di}
che riportiamo pari pari e senza dimostrazione da \cite{sesini1}, pag. 99:
{\em il quadrilatero articolato piano pu\`o essere a doppia manovella o
a manovella-bilanciere,
soltanto se la somma del pi\`u piccolo e del pi\`u grande dei suoi lati non \`e
maggiore della somma degli altri due. In tal caso si ha una doppia manovella,
se l'asta fissa \`e la pi\`u corta, una manovella e un bilanciere se l'asta
fissa \`e una delle contigue alla pi\`u corta (essendo la manovella il lato pi\`u corto).
In tutti gli altri casi il quadrilatero \`e a doppio bilanciere}.

\begin{figure}[hbt]
\centering
\begin{minipage}[b]{0.48\textwidth}
\centering
     \includegraphics[width=0.88\textwidth]{part2/quadri/FIG/quadrilatero_telaio.pdf}
\begin{picture}(0,0)(170,25)
        \scriptsize{
        \put(12,169){$B$}
        \put(155,111){$C$}
        \put(132,73){$D$}
        \put(12,60){$A$}
}
\end{picture}
        \caption{\em Quadrilatero con telaio vincolato a terra.}
     \label{fig:f_quad_telaio}
\end{minipage}\hfill
\begin{minipage}[b]{0.48\textwidth}
\centering
    \includegraphics[width=0.88\textwidth]{part2/quadri/FIG/quadrilatero_schematico.pdf}
\begin{picture}(0,0)(170,25)
        \scriptsize{
        \put(12,169){$B$}
        \put(94,153){$\omega_{\scriptscriptstyle{BC}}$}
        \put(115,145){${\dot{\omega}}_{\scriptscriptstyle{BC}}$}
        \put(46,112){$\omega_{\scriptscriptstyle{AB}}$}
        \put(46,100){${\dot{\omega}}_{\scriptscriptstyle{AB}}$}
        \put(155,111){$C$}
        \put(148,90){$\omega_{\scriptscriptstyle{AD}}$}
        \put(142,78){${\dot{\omega}}_{\scriptscriptstyle{AD}}$}
        \put(128,72){$D$}
        \put(12,60){$A$}
}
\end{picture}
        \caption{\em Rappresentazione schematica del quadrilatero.}
     \label{fig:f_quad_schema}
\end{minipage}
\end{figure}

\noindent Ad esempio nel caso di figura \ref{fig:f_quad_telaio} abbiamo $|\overrightarrow{DC}|+|\overrightarrow{BC}|<|\overrightarrow{AB}|+|\overrightarrow{AD}|$,
quindi avremo tre possibilit\`a: due manovelle scegliendo $DC$ come telaio,
una manovella e un bilanciere quando il telaio \`e $AD$ oppure $BC$, 
infine due bilancieri con telaio $AB$. Nel caso riportato in figura
\ref{fig:f_quad_telaio}
 il telaio risulta essere
un lato contiguo al lato pi\`u corto: il membro $CD$ funger\`a, in questo
caso, da manovella
mentre il lato $AB$ sar\`a un bilanciere.



\noindent La cinematica del quadrilatero articolato si pu\`o studiare mediante le nozioni,
esposte in un capitolo precedente, circa i moti relativi. In particolare, ricordando
l'esempio di figura \ref{fig:f114} e la relativa soluzione, possiamo anche in questo caso
esprimere il moto di un membro rispetto a un riferimento fisso e a uno
mobile e imporre che le quantit\`a cinematiche coinvolte, velocit\`a e accelerazione,
corrispondano tra loro.

\noindent Riferiamoci alla rappresentazione schematica di figura
\ref{fig:f_quad_schema}, dove oltre al quadrilatero sono rappresentate velocit\`a
angolare e accelerazioni angolari della manovella $DC$ che si supporranno note.
Se ci proponessimo di trovare le analoghe grandezze cinematiche per il bilanciere $AB$,
potremmo scrivere la seguente relazione che esprime la velocit\`a del punto $C$,
sia in termini assoluti, sia in termini relativi, nel suo moto attorno al
punto $B$. Qui immaginiamo incardinato il sistema di riferimento relativo che
trasla, mantenendo perci\`o i suoi assi sempre nelle medesime direzioni.
Avremo
\begin{equation}
{\bm v}_{\scriptscriptstyle{{\rm ass}}}=
{\bm v}_{\scriptscriptstyle{{\rm rel}}}+
{\bm v}_{\scriptscriptstyle{{\rm tr}}}\,,
\label{eq:e_v_rivals_q}
\end{equation}
\noindent cio\`e
\begin{equation}
{{\bm v}_{\scriptscriptstyle{C}}}=
{{\bm v}_{\scriptscriptstyle{C}}}_{\scriptscriptstyle{{\rm rel}}}+
{{\bm v}_{\scriptscriptstyle{B}}}\,.
\label{eq:e_v_quad_schematico}
\end{equation}
\noindent Riferendoci alle quantit\`a indicate in figura \ref{fig:f_quad_schema},
possiamo riscrivere la \ref{eq:e_v_quad_schematico} nel modo seguente
\begin{equation}
{{\bm v}_{\scriptscriptstyle{C}}}=
\omega_{\scriptscriptstyle{BC}} |\overrightarrow{BC}| \widehat{{\perp{BC}}}
+\omega_{\scriptscriptstyle{AB}} |\overrightarrow{AB}| \widehat{{\perp{AB}}}\,,
\label{eq:e_v_quad_schematico0}
\end{equation}
\noindent dove i versori
$\widehat{{\perp{BC}}}$ e $\widehat{{\perp{AB}}}$ indicano che la direzione dei
due termini incogniti a destra del segno di uguaglianza \`e nota.
L'equazione \ref{eq:e_v_quad_schematico0} contiene pertanto due 
termini vettoriali di modulo incognito e direzione conosciuta.
Si pu\`o dunque seguire (in linea di principio) lo schema grafico riportato in figura
\ref{fig:f116}, e lo svolgimento riportato nel medesimo contesto che ci permette 
di determinare la velocit\`a angolare del bilanciere e della biella,
$\omega_{\scriptscriptstyle{AB}}$ e $\omega_{\scriptscriptstyle{BC}}$.

\noindent Consideriamo ora le accelerazioni. Dato che il sistema di riferimento relativo trasla, scrivendo la \ref{e138} il termine dell'accelerazione di Coriolis sar\`a assente:
\begin{equation}
{{\bm a}_{\scriptscriptstyle{C}}}=
{{\bm a}_{\scriptscriptstyle{C}}}_{\scriptscriptstyle{{\rm rel}}}+
{{\bm a}_{\scriptscriptstyle{B}}}\,.
\label{eq:e_a_quad_schematico}
\end{equation}
\noindent Risulta conveniente scrivere i termini
${{\bm a}_{\scriptscriptstyle{C}}}_{\scriptscriptstyle{{\rm rel}}}$ e 
${{\bm a}_{\scriptscriptstyle{B}}}$ 
come somma delle loro
componenti: normale e tangenziale
\begin{equation}
{{\bm a}_{\scriptscriptstyle{C}}}=
-\omega^2_{\scriptscriptstyle{BC}} \overrightarrow{BC} + {\dot{\omega}_{\scriptscriptstyle{BC}}}|\overrightarrow{BC}|\widehat{{\perp{BC}}}
-\omega^2_{\scriptscriptstyle{AB}} \overrightarrow{AB} + {\dot{\omega}_{\scriptscriptstyle{AB}}}|\overrightarrow{AB}|\widehat{{\perp{AB}}}\,,
\label{eq:e_a_quad_schematico2}
\end{equation}
\noindent dove, ancora una volta, i versori
$\widehat{{\perp{BC}}}$ e $\widehat{{\perp{AB}}}$ indicano che la direzione
di questi due termini incogniti \`e conosciuta.
Gli altri termini risultano essere tutti noti, quindi
ricadiamo ancora nel caso di una equazione vettoriale dove le incognite sono
i moduli del secondo e del quarto termine a destra del segno di uguaglianza della
\ref{eq:e_a_quad_schematico2} e per la soluzione ci riferiamo, una volta di pi\`u,
al procedimento di figura \ref{fig:f116}, dal quale
otterremo anche
${\dot{\omega}}_{\scriptscriptstyle{AB}}$ e ${\dot{\omega}}_{\scriptscriptstyle{BC}}$.

\section{Applicazioni dei Quadrilateri} \label{q_squilibrio}
\noindent Abbiamo gi\`a ricordato che il quadrilatero articolato si presta a
svariatissime applicazioni. Nella sua versione con doppia manovella esso
pu\`o trasmettere il moto rotatorio
tra due alberi paralleli, nella configurazione in cui
le tracce degli assi di tali alberi siano i centri delle
rispettove coppie rotoidali sul telaio. In questo caso,
se il quadrilatero presenta i lati opposti uguali tra loro, la trasmissione
del moto sar\`a {\em omocinetica}\index{trasmissione!omocinetica} con
rapporto di trasmissione unitario e il meccanismo
prende il nome di {\em parallelogramma articolato}\index{parallelogramma articolato},
famoso per aver collegato, un tempo, le ruote delle locomotive a vapore in
modo tale da renderle tutte motrici.
La trasmissione del moto tramite un quadrilatero generico non sar\`a pi\`u,
in generale, omocinetica
e si potranno avere due casi di particolare interesse: manovella-manovella, oppure
manovella-bilanciere. In entrambi i casi, il progetto (sintesi) del quadrilatero
mira a ottenere un particolare movimento rotatorio o oscillatorio 
sulla manovella o sul bilanciere in uscita, tale da rendere diverse tra loro
 le velocit\`a di
percorrenza ciclica delle due fasi che questo membro attraversa durante il moto
a velocit\`a angolare costante della manovella motrice.

\begin{figure}[hbt]
\centering
\begin{minipage}[b]{0.48\textwidth}
\centering
\includegraphics[width=0.9\textwidth]{part2/quadri/FIG/quadri/doppia_manovella.pdf}
\begin{picture}(0,0)(130,0)
\scriptsize{
\put(39,125){$\alpha_2=81^{\circ}$}
\put(8,119){$B$}
\put(88,114){$C$}
\put(26,53){$A$}
\put(55,53){$D$}
\put(85,26){$\alpha_1=333^{\circ}$}
\put(109,97){$\beta_2=34^{\circ}$}
\put(11,37){$\beta_1=214^{\circ}$}
}
\end{picture}
      \caption{\em Quadrilatero a doppia manovella.}
 \label{fig:doppia_manovella}
\end{minipage}\hfill
\begin{minipage}[b]{0.48\textwidth}
\centering
\includegraphics[width=0.9\textwidth]{part2/quadri/FIG/quadri/vel_doppia_manovella.pdf}
\vspace*{6mm}
\begin{picture}(0,0)(129,0)
\scriptsize{
\put(-5,42){${\omega_{\scalebox{.5}{DC}}}/{\omega_{\scalebox{.5}{AB}}}$}
\put(45,60){${\omega_{\scalebox{.5}{BC}}}/{\omega_{\scalebox{.5}{AB}}}$}
\put(124,2){$\alpha$}
}
\end{picture}
\vskip .5mm
      \caption{\em Velocit\`a angolare della manovella $CD$ e della biella $BC$.}
     \label{fig:vel_doppia_manovella}
\end{minipage}
\end{figure}

\noindent La figura \ref{fig:doppia_manovella} riporta un quadrilatero con doppia
manovella: $AB$ e $CD$ compiono intere rotazioni, mentre l'asta $BC$ funge da biella.
In tale figura sono riportate alcune posizioni occupate dalle cerniere $B$ e $C$ 
durante la rotazione a velocit\`a angolare costante della manovella $AB$.
Analizzando tali posizioni si nota che la spaziatura dei punti
generati dalla cerniera $B$, che \`e l'estremo della manovella motrice, \`e appunto
costante mentre la spaziatura delle tracce relative alla cerniera $C$, che \`e l'estremo
della manovella condotta, \`e variabile.
Ci\`o rende perfettamente conto delle velocit\`a angolari delle due manovelle:
come abbiamo detto, costante per quella motrice e variabile per la manovella cedente.
La figura \ref{fig:vel_doppia_manovella} riporta la velocit\`a angolare della manovella
$CD$ e, per completezza-curiosit\`a, quella della biella $BC$, rapportate alla
velocit\`a angolare della manovella motrice. 

\noindent La velocit\`a angolare variabile della manovella $DC$ pu\`o essere a
sua volta utilizzata per azionare un secondo meccanismo come, ad esempio, il 
manovellismo ordinario di una pressa, al quale sarebbe in tal modo possibile
un funzionamento con {\em ritorno rapido}\index{ritorno rapido}.
Consideriamo infatti l'ipotesi di muovere quest'ultimo manovellismo mediante la rotazione della
manovella $DC$ (che fisicamente potrebbe coincidere in parte con quella del manovellismo
stesso). Scegliendo opportunamente la posizione del punto morto di tale manovellismo,
cio\`e il punto di partenza per gli angoli di manovella, potremmo avere, per il
cursore del manovellismo mosso, una corsa di andata pi\`u lenta rispetto a quella di
ritorno o viceversa.
Con riferimento alla figura \ref{fig:doppia_manovella} scegliamo come angolo di
partenza $\beta_1$. Il cursore del manovellismo invertir\`a il suo moto in $\beta_2$,
distante $180^{\circ}$ da $\beta_1$, a valle di una rotazione della
manovella $AB$ di un angolo di ``andata'' pari a
$\alpha_a=\alpha_2-\alpha_1$. Parimenti, il movimento di ritorno si estender\`a
sull'arco percorso sempre di $AB$ pari a
$\alpha_r=360-(\alpha_2-\alpha_1)$.
I tempi di andata e di ritorno, $t_a$ e $t_r$, saranno perci\`o diversi tra loro
e proporzionali ai due angoli percorsi dalla manovella motrice durante le
rispettive fasi:
\begin{equation}
\alpha_a = 360^{\circ} {t_a \over{t_a+t_s}}\,,\;\;\;
\alpha_r = 360^{\circ} {t_s \over{t_a+t_s}}\,.
\label{eq:angoli_andata_ritorno}
\end{equation}
\noindent Il rapporto $s={{\alpha_a}\over{\alpha_r}}$ 
viene talvolta indicato come {\em squilibrio del quadrilatero}\index{quadrilatero!squilibrio}
e, come abbiamo gi\`a accennato, dipende anche dalla scelta della
 posizione iniziale
$\alpha_1$. Lo squilibrio si ritiene tanto pi\`u elevato quanto pi\`u
esso si scosta dall'unit\`a.
In pratica, accade raramente che 
il rapporto tra gli angoli di andata e ritorno, cio\`e lo squilibrio, sia maggiore di
tre o minore di un terzo.
In figura \ref{fig:doppia_manovella} sono indicate le coppie di angoli
$\alpha_1$, $\beta_1$, $\alpha_2$, $\beta_2$ che per questo quadrilatero
realizzano il {\em massimo squilibrio}:
\begin{equation}
s={{333^{\circ}-81^{\circ}}\over{360^{\circ} -333^{\circ} +81^{\circ}}} = 2.3\,.
\label{eq:esempio_squilibrio}
\end{equation}
\noindent Evidenziamo, sperando di contribuire in modo positivo alla chiarezza dell'argomento,
che esiste sempre la
possibilit\`a di scegliere l'inizio delle rotazioni della manovella $DC$, $\beta_1$,
in modo tale da rendere lo squilibrio minimo, cio\`e pari a uno. Con dubbia 
utilit\`a pratica, vedremo in tal caso compiere
alla manovella cedente un mezzo giro dopo che la motrice ha compiuto anch'essa $180^{\circ}$,
rimanendo tuttavia variabile il rapporto di trasmissione tra le due manovelle.
\begin{figure}[hbt]
\centering
\begin{minipage}[b]{0.48\textwidth}
\centering
\includegraphics[width=0.9\textwidth]{part2/quadri/FIG/quadri/manovella_bilanciere.pdf}
\begin{picture}(0,0)(130,0)
\scriptsize{
\put(110,152){$C$}
\put(35,83){$B$}
\put(46,75){$\alpha_1=51^{\circ}$}
\put(8,34){$A$}
\put(111,34){$D$}
\put(4,-6){$\alpha_2=263^{\circ}$}
}
\end{picture}
\vskip 2mm
      \caption{\em Quadrilatero a manovella e bilanciere.}
 \label{fig:manovella_bilanciere}
\end{minipage}\hfill
\begin{minipage}[b]{0.48\textwidth}
\centering
\includegraphics[width=0.9\textwidth]{part2/quadri/FIG/quadri/vel_manovella_bilanciere.pdf}
\vspace*{6mm}
\begin{picture}(0,0)(129,0)
\scriptsize{
\put(35,75){${\omega_{\scalebox{.5}{DC}}}/{\omega_{\scalebox{.5}{AB}}}$}
\put(127,2){$\alpha$}
}
\end{picture}
\vskip 2mm
      \caption{\em Velocit\`a angolare della manovella $CD$.}
     \label{fig:vel_manovella_bilanciere}
\end{minipage}
\end{figure}
\noindent In figura \ref{fig:manovella_bilanciere} \`e rappresentato un quadrilatero il cui
membro $DC$ non compie intere rotazioni ma oscilla percorrendo archi di $54.2^{\circ}$  che, anche in questo esempio, assumono i nomi di andata e ritorno.
Nella stessa figura sono evidenziate, tramite la traccia della cerniera $C$, 
le differenti velocit\`a con le quali vengono descritti tali archi.
Anche qui i tempi di percorrenza
dell'andata e del ritorno, $t_a$ e $t_b$, saranno proporzionali ai corrispondenti
angoli percorsi,
a velocit\`a costante, dalla manovella $AB$. Perci\`o, appoggiandoci 
alle precedenti notazioni,
lo squilibrio di questo quadrilatero diventa
\begin{equation}
s={{263^{\circ}-51^{\circ}}\over{360^{\circ} -263^{\circ} +51^{\circ}}} = 1.43\,.
\label{eq:squilibrio_manovella_bilanciere}
\end{equation}
\noindent In questo caso, gli angoli sottesi dal bilanciere nei quali avviene
l'inversione della sua corsa
risultano  inequivocabili e lo squilibrio massimo,
riportato nella \ref{eq:squilibrio_manovella_bilanciere},
\`e l'unico squilibrio del quale abbia senso parlare.

\noindent Interessanti applicazioni di meccanismi con tempi di andata e ritorno diversi
tra loro si possono realizzare mediante l'impiego di particolari quadrilateri nei
quali una coppia rotoidale ha asse improprio. Tali meccanismi si chiamano {\em manovellismi}\index{manovellismo} e, 
data la loro importanza e diffusione, verranno trattati nel successivo
capitolo, a loro completamente dedicato.

\noindent Il problema di ottenere particolari quadrilateri, a doppia manovella oppure a manovella e bilanciere, che realizzino un dato squilibrio tra i tempi di andata e 
ritorno pu\`o essere risolto tramite specifici metodi
grafici i quali, se coadiuvati da una buona dose di esperienza del progettista,
portano al risultato desiderato.
A tali metodi di sintesi non facciamo cenno, ritenendoli
fuori luogo in questi appunti e ottimamente collocati in libri specialistici come il
pluricitato \cite{ruggieri}, pagg. 215--220. Aggiungiamo che la sintesi dei quadrilateri
avviene molto frequentemente tramite analisi ripetute, partendo da geometrie iniziali
che solitamente il progettista porta con s\'e nel suo bagaglio di esperienza.

\section{Curve di Biella}

\noindent Fin qui, poco si \`e detto del moto della biella eccezion fatta per la 
figura \ref{fig:vel_doppia_manovella} dove \`e riportata la velocit\`a angolare
della biella $BC$ rapportata a quella della manovella $AB$.
D'altra parte lo studio del moto della biella di un quadrilatero articolato ha
applicazioni pratiche molto rilevanti.
Per un dato quadrilatero, i punti del piano su cui giace la
biella (pensato solidale con essa) percorrono, durante il movimento di questa, una doppia infinit\`a di curve.
Al variare del {\em punto di biella}\index{biella!punto di} considerato, tali
{\em curve di biella}\index{biella!curve di} assumono forme diversissime tra loro
e possono risultare appetibili quando appunto servono movimentazioni
cicliche e peculiari.
Le curve di biella presentano infatti una grande
variet\`a di archi e intrecci
e vengono spesso prese in considerazione in diversi contesti
del progetto di macchine automatiche.
In tempi passati, il progettista interessato all'impiego di una curva di biella
si affidava per la sua scelta ai cosiddetti {\em atlanti}\index{atlanti di quadrilateri}.
Tra queste opere, la pi\`u
completa \`e quella di Hrones e Nelson \cite{hrones}, che presenta, tramite
un ingegnoso ordine sistematico, pi\`u di settemila curve di biella.  
\begin{figure}[hbt]
\centering
\includegraphics[width=0.9\textwidth]{part2/quadri/FIG/quadri/curve_di_biella.pdf}
\begin{picture}(0,0)(250,0)
\scriptsize{     
\put(10,352){\rotatebox{-54}{$\longrightarrow$}}
\put(-13,354){punto di biella}
\put(107,298){\rotatebox{-54}{$\longrightarrow$}}
\put(83,301){punto di biella}
\put(125,182){$C$}
\put(22,54){$A$}
\put(-30,54){$B$}
\put(142,54){$D$}
}
\end{picture}
      \caption{\em Alcune curve di biella di un quadrilatero articolato manovella-bilanciere. I punti di biella sono evidenziati.}
 \label{fig:f_cur_biella}
\end{figure}

\noindent Oggigiorno,  non \`e difficile trovare ``pagine di calcolo'' che contengono
analizzatori di 
quadrilateri i quali, a fronte dei dati che identificano il meccanismo da
studiare e della scelta del punto di biella, forniscono immediatamente la 
relativa curva. Il progettista, guidato dall'esperienza e dalle analisi numeriche
ripetute, pu\`o cos\`i
avvicinarsi per passi alla soluzione del proprio problema di progetto.
Naturalmente esistono interi programmi commerciali di calcolo che possono analizzare
i quadrilateri articolati, i quali programmi spesso sono molto generali e pensati
per l'analisi di meccanismi anche molto complessi.
Diciamo per\`o, per esperienza, che un tecnico, qualora si accinga all'utilizzo
delle curve di biella, difficilmente rinuncia a trarre ispirazione dai casi
riportati sugli atlanti, se non altro per generare l'embrione di quello che poi
costituir\`a la soluzione finale. 
\begin{wrapfigure}{r}{0.5\textwidth}
     \begin{center}
     \includegraphics[width=0.42\textwidth]{part2/quadri/FIG/quadri/watt.pdf}
     \end{center}
\begin{picture}(0,0)(0,0)
        \scriptsize{
        \put(77,85){$B$}
        \put(58,18){$B'$}
        \put(90,45){$C$}
        \put(90,20){$C'$}
        \put(89,69){$E$}
        \put(73,19){$E'$}
        \put(155,58){$D$}
        \put(11,73){$A$}
}
\end{picture}
        \caption{\em Quadrilatero di {\em Watt}.}
     \label{fig:watt}
\end{wrapfigure}
\noindent A titolo di esempio riportiamo in figura \ref{fig:f_cur_biella}
le traiettorie di alcuni punti di biella del quadrilatero a manovella-bilanciere
$ABCD$. Anche in questo caso, la distanza
tra le tracce lasciate dal punto di biella d\`a un'idea precisa della 
velocit\`a mediante la quale tale punto percorre la propria traiettoria
(accorgimento grafico, questo,
riportato con precisione sugli atlanti tramite linee a tratteggio di lunghezza
diversificata).
\noindent Non deve sorprendere se particolari curve
di biella, che, da un punto di vista teorico, rettilinee non sono, vengano
invece sfruttate nella pratica come generatori di spostamenti rettilinei
o {\em guide rettilinee
approssimate}\index{guide rettilinee approssimate}.
La figura \ref{fig:watt} mostra una di queste guide approssimata dal
movimento del
punto centrale $E$ della biella $BC$. Il quadrilatero $ABCD$ prende il nome dal suo 
inventore e si chiama quadrilatero di {\em Watt} (talvolta parallelogramma
di {\em Watt}, anche se le applicazioni che lo vedono
in tal veste sono molto rare).
Nella nostra figura, le tracce del punto di biella $E$ non sono equidistanziate,
infatti la figura \`e stata ottenuta
mediante il movimento a velocit\`a costante del bilanciere $AB$;
inoltre, sempre nella figura citata, viene riportato un pezzo
molto esteso della curva che il punto $E$ pu\`o percorrere.
Normalmente per\`o questo meccanismo viene
impiegato sfruttando la guida rettilinea approssimata, fornita dal punto mediano della 
biella, a fronte di un azionamento
direttamente applicato allo stesso punto $E$.
Tale azionamento pu\`o essere l'oscillazione verticale del telaio di un'automobile
quando, ad esempio, il quadrilatero di Watt viene utilizzato nelle sospensioni.
Ma il movimento oscillatorio verticale della scocca di un veicolo
\`e limitato nella propria corsa dai vincoli geometrici della sospensione
stessa, pertanto va da s\'e che il punto di biella percorrer\`a
soltanto una porzione, quella quasi rettilinea, della
 sua traiettoria possibile.
\newpage
\thispagestyle{empty}
\null

\endinput 
\section{Cenni Generali sulla Sintesi dei Quadrilateri}
\noindent Desideriamo concludere questo paragrafo con un brevissimo cenno
riassuntivo alle tecniche di sintesi del quadrilatero articolato.
Per prima cosa distinguiamo da subito due strade possibili: quella {\em diretta}\index{
sintesi diretta dei quadrilateri} e quella {\em indiretta}\index{sintesi
indiretta dei quadrilateri}. La prima di queste strade conduce alla determinazione,
tramite metodi grafici o analitici, dei punti di cerniera del quadrilatero
che soddisfa le richieste di
progetto in modo univoco oppure ad infinit\`a di soluzioni dipendenti da uno o 
pi\`u parametri liberi, come potrebbero essere il fattore di scala o il posizionamento
di una o pi\`u cerniere.
Il metodo indiretto consiste invece nell'analisi numerica ripetuta di diverse
soluzioni che,
guidata dall'intuito e dall'esperienza del progettista, nonch\'e dalle agili
possibilit\`a interattive dei moderni programmi di calcolo, porta via via alla
soluzione ottimale. 
\`E possibile una classificazione sistematica dei vari problemi di sintesi
che possono essere ricondotti a tre problemi paradigmatici.

\noindent Il primo di questi problemi \`e gi\`a stato esposto, con piglio descrittivo,
poco sopra in queste note. Si tratta, con riferimento alle figure
\ref{fig:doppia_manovella} o \ref{fig:manovella_bilanciere}, di correlare le posizioni
angolari della manovella motrice con le rispettive posizioni dell'altra manovella
(oppure bilanciere). Si parla in questi casi di uso del quadrilatero come
{\em generatore di funzioni}\index{generatore di funzioni}. Crediamo sia evidente
che non si possa sperare di ottenere un meccanismo in grado di seguire
pedissequamente una data funzione arbitraria $\beta(\alpha)$. Siccome le incognite,
o gradi di libert\`a,
risultano essere le quattro lunghezze dei membri del quadrilatero \`e ragionevole
aspettarsi che potranno essere soddisfatte puntualmente soltanto quattro valori:
$\beta(\alpha_1)$, $\beta(\alpha_2)$, $\beta(\alpha_3)$ e $\beta(\alpha_4)$.
In questo caso quindi vengono fatte corrispondere cinque posizioni del quadrilatero e 
si parla di problema a dieci gradi di libert\`a.
Anzich\'e essere fatte valere tutte sugli angoli, queste quattro condizioni potrebbero,
in parte, essere  applicate alle
derivate di tali angoli: ad esempio, nella sintesi del quadrilatero di figura
\ref{fig:manovella_bilanciere} dovremmo imporre velocit\`a nulla nelle due posizioni
di {\em punto morto}\index{punto morto} del bilanciere $DC$ e avremo cos\`i assegnate
$\beta(\alpha_1)$, ${\dot\beta}(\alpha_1)=0$, $\beta(\alpha_2)$ e ${\dot\beta}(\alpha_2)=0$. 
\noindent Un altro tipico problema di sintesi pu\`o essere ricondotto alla ricerca 
di traiettorie desiderabili per un dato punto del {\em piano di biella}\index{biella!piano di}, traiettorie
cio\`e percorse da un punto di biella. Ci siamo gi\`a occupati di queste traiettorie,
in modi descrittivo, in un paragrafo precedente e potremmo, in linea con la maggior parte
degli autori che trattano questo argomento, chiamare questa funzione richiesta al
quadrilatero {\em generazione di traiettorie}\index{generazione!di traiettorie}.
Da un punto di vista molto formale si potrebbe asserire che, in questo caso,
le incognite
saranno le quattro coordinate delle cerniere del telaio, le tre lunghezze delle
aste rimanenti, le due lunghezze dei rimanenti lati del triangolo che individua (tramite
uno dei suoi vertici) il punto di biella e infine la posizione iniziale della 
manovella movente: totale 10 gradi di libert\`a. A questi gradi di libert\`a,
sempre molto formalmente, corrisponderanno cinque posizioni (due coordinate ciascuna)
che potranno essere soddisfatte, oppure quattro posizioni e una tangente alla curva
di biella in uno dei quattro punti, oppure tre punti e due tangenti.
 
\noindent Un terzo problema di sintesi riguarda la possibilit\`a di fare guidare alla
biella un oggetto da posizionare nel piano. Questa funzione si pu\`o chiamare 
{\em generazione di moti piani}\index{generazione!di moti piani}. Anche qui, 
\cite{ruggieri}, pag. 279, i gradi di libert\`a sono dieci e si potrebbero soddisfare
cinque posizioni per l'oggetto movimentato, che saturerebbero tutte le incognite.

\noindent I tre problemi di sintesi dimensionale del quadrilatero sono per\`o tutti
riconducibili l'uno all'altro come riportato in \cite{ruggieri} pag 279.
Ci sembra opportuno dare solo un brevissimo cenno alla soluzione al terzo di questi
problemi a 5 gradi di libert\`a (o dieci se si considera che ogni punto \`e individuato
da due coordinate. La generazione di particolari posizioni della biella \`e stato
formulato, in modo assai generale, da Burmester quindi tale problema ne prende il
nome.

\noindent Nella sua formulazione pi\`u generale, il {\em problema di Burmester}
consiste nell'individuare un quadrilatero che,  tramite un punto di biella, tocchi
alcune posizioni precise.
Doverebbe essere chiaro da quanto fin qui esposto che il numero massimo di tali punti
di precisione non pu\`o superare il numero cinque. 
Difficilmente per\`o ci si pu\`o spingere alla saturazione di tutti i gradi di
libert\`a disponibili, in quanto si potrebbero trovare soluzioni non praticabili.
Ci si accontenta spesso di progettare quadrilateri in cui un punto di biella 
passa per tre punti di precisione, lasciando liberi due gradi di libert\`a in modo
da individuare, nel novero delle soluzioni ottenute,
meccanismi effettivamente realizzabili.

\noindent Sebbene gli studi del Burmester siano molto interessanti e
indispensabili laddove venga
richiesto al punto di biella il passaggio da quattro o da cinque punti di
precisione,
facciamo semplicemente notare che la soluzione per tre soli punti di precisione
\`e talmente a portata di mano che sarebbe un peccato non descriverla.

\noindent Sia dunque da realizzare un quadrilatero in cui un punto di biella
passi da tre punti stabiliti. Per tali punti passer\`a una e una sola circonferenza.
Il moto della biella avr\`a quindi come centro di spostamento finito il centro
di tale circonferenza.
(da concludere)


\chapter{Manovellismi}

\section{Introduzione}

\noindent I meccanismi che si identificano col nome di 
manovellismi non sono altro, da un punto di vista cinematico, che
casi particolari di quadrilateri articolati. Essi sarebbero potuti entrare
a pieno titolo nel capitolo precedente, visto che le tecniche per
il loro studio, siano queste grafiche, analitiche o numeriche, non
differiscono da quelle che si  impiegano nell'analisi dei loro
progenitori. Il presente capitolo, dedicato esclusivamente allo studio dei
manovellismi, si giustifica valutando l'importanza industriale che 
tali meccanismi hanno
avuto e quindi anche la profondit\`a e la mole di studi che
li riguarda.
Affermiamo, senza timore di esagerare, che il
{\em manovellismo ordinario}\index{manovellismo!ordinario} o
{\em di spinta}\index{manovellismo!di spinta} \`e stato il cuore della
Rivoluzione Industriale, essendo un componente essenziale del motore a vapore.
Anche ai giorni nostri,  nel settore del trasporto su strada,
tale meccanismo entra, con numero
variabile di esemplari, praticamente in tutti i motori endotermici, tralasciando
pochissime eccezioni.

\noindent Richiamiamo la nomenclatura delle quattro cerniere
di un quadrilatero ricordando la figura \ref{fig:f_quad_schema} ovvero
prendendo visione della figura \ref{fig:quad_D_infinito}.
Ciascuna delle quattro cerniere $A$, $B$, $C$, e $D$ pu\`o essere mandata all'infinito
seguendo una precisa direzione e divenendo una di quelle
entit\`a che la geometria
definisce come {\em punti impropri} del piano.
\`E scontato che, affinch\'e una (alla volta, per il momento)
delle cerniere si trovi
infinitamente distante dalle altre tre, \`e necessario che le due aste
che vi concorrono siano a loro volta di lunghezza infinita.
\noindent Data la ovvia equivalenza cinematica dei punti $A$ e $D$ e dei punti
$B$ e $C$ si pu\`o intuire che pensare situato all'infinito l'uno o l'altro dei punti di
tali coppie produca, a livello cinematico, lo stesso effetto,
e cos\`i \`e.

\noindent Affermiamo che allo spostarsi all'infinito delle cerniere
$A$ oppure $D$ il quadrilatero d\`a origine ai {\em manovellismi ordinari}\index{manovellismo!ordinario}, naturalmente dopo avere mutato la cerniera 
``fuori uso'' con un altro tipo di vincolo, in generale costituito da un 
{\em cursore}\index{cursore} che scorre in un {\em glifo}\index{glifo}.

\noindent Nel secondo caso, quando ad andare all'infinito \`e una delle
cerniere $B$ oppure $C$, avremo i {\em manovellismi
non ordinari}.\index{manovellismo!non ordinario}
\noindent Andiamo con ordine.


\section{Il Manovellismo Ordinario}

\begin{figure}[hbt]
\centering
\begin{minipage}[b]{0.48\textwidth}
\centering
\includegraphics[width=0.95\textwidth]{part2/manovellismi/FIG/manovellismi/quad_D_infinito.pdf}
\begin{picture}(0,0)(155,-12)
\scriptsize{
\put(17,53){$A$}
\put(44,85){$B$}
\put(140,56){$C$}
\put(73,-3){$D$}
}
\end{picture}
\vskip .2mm
      \caption{\em Generico quadrilatero manovella-bilanciere.}
 \label{fig:quad_D_infinito}
\end{minipage}\hfill
\begin{minipage}[b]{0.48\textwidth}
\centering
\includegraphics[width=0.95\textwidth]{part2/manovellismi/FIG/manovellismi/quad_D_infinito_man.pdf}
\begin{picture}(0,0)(155,3)
	\scriptsize{
\put(16,53){$A$}
\put(42,85){$B$}
\put(132,60){$C$}
\put(87,58){$\sigma$}
\put(93,3){$D$}
\put(100,10){\rotatebox{-90}{$\longrightarrow$}}
\put(104,3){$\infty$}
}
\end{picture}
	\caption{\em Manovellismo ordinario centrato ottenuto spostando la cerniera $D$ a distanza ``infinita''.}
     \label{fig:quad_D_infinito_man}
\end{minipage}
\end{figure}

\noindent In figura \ref{fig:quad_D_infinito_man} \`e
riportato il quadrilatero degenere che si ottiene mandando all'infinito la
cerniera $D$, una delle due vincolate al telaio, del quadrilatero di
figura \ref{fig:quad_D_infinito}.
La direzione secondo la 
quale la cerniera $D$ diventa un punto improprio \`e, in questo caso, quella 
ortogonale al segmento $AC$: otteniamo in questo modo un {\em manovellismo
ordinario centrato}\index{manovellismo!ordinario}. Si ottiene
invece un
{\em manovellismo ordinario deviato}\index{manovellismo!ordinario deviato}
scegliendo, per la cerniera $D$, una direzione diversa\footnote{
\noindent Dal punto di vista cinematico, si ottiene un meccanismo
perfettamente equivalente

\begin{wrapfigure}{r}{0.33\textwidth}
     \begin{center}
     \includegraphics[width=0.32\textwidth]{part2/manovellismi/FIG/capsulismo.pdf}
     \end{center}
\begin{picture}(0,0)(0,0)
	\scriptsize{
\put(118,81){$C$}
\put(29,97){$B$}
\put(59,35){$D$}
\put(-3,90){$A$}
\put(-5,92){\rotatebox{-159}{$\longrightarrow$}}
\put(-3,80){$\infty$}
}
\end{picture}
\vskip -5.3mm
        \caption{\em Manovellismo ordinario centrato a glifo mobile.}
     \label{fig:capsulismo}
\end{wrapfigure}

\noindent a quello di figura \ref{fig:quad_D_infinito_man} mandando all'infinito
la cerniera $A$ anzich\'e la $D$.
\noindent Per dare un po' di vitalit\`a a questa nota, 
scegliamo in questo caso di rendere mobile il glifo e tenere ferma
la cerniera $B$.
Il dispositivo pratico che ne risulta \`e piuttosto
diverso nei due frangenti.
Nella stragrande maggioranza dei casi,
il manovellismo ordinario centrato viene progettato e realizzato
tenendo fisso il glifo.
Nell'altro caso si ottiene invece il
sistema articolato che \cite{sesini1} riporta a pag. 107, Fig. IV-4 (e che
noi riportiamo in figura \ref{fig:capsulismo}), 
dichiarando che ``tale meccanismo non trova notevoli applicazioni''; anche
all'autore
non risultano applicazioni rilevanti di questo
manovellismo, neppure ai nostri giorni. La figura riportata su queste note
appare per\`o piuttosto diversa
da quella che si trova in \cite{sesini1}. Questo perch\'e
abbiamo preferito privilegiare, in questo caso, la coerenza con la precedente
figura \ref{fig:quad_D_infinito}, mantenendo perci\`o gli stessi rapporti
dimensionali tra i membri del quadrilatero e gli stessi orientamenti
relativi,
piuttosto che garantire alla manovella $CD$ di potere compiere intere
rotazioni.}.
\noindent Il percorso della cerniera $C$ diventa il segmento $\sigma$,
che pu\`o essere una scanalatura (glifo) in cui scorre un cursore di forma
opportuna. Ad esempio, nei motori endotermici, i cilindri costituiscono i
glifi e gli stantuffi hanno la funzione di cursore. \`E per\`o chiaro che l'elemento che funge
da guida cava pu\`o essere invertito con l'elemento che funge da cursore,
trasformando quest'ultimo in un glifo, e
questa opportunit\`a si presenta per tutti i manovellismi, ordinari e non.
\noindent In ambito industriale il manovellismo si studia
e si progetta, come \`e ovvio, tramite l'ausilio dei programmi di calcolo 
per l'analisi di sistemi articolati.
\noindent Tuttavia il manovellismo ordinario centrato \`e talmente diffuso,
specialmente in ambito {\em automotive} (ogni vettura  a combustione
interna ne ospita in media quattro) e gli
studi che lo hanno riguardato all'inizio del secolo scorso sono cos\`i rilevanti
e fecondi di risultati, da imporre alla coscienza 
di chi insegna Meccanica Applicata l'esposizione, magari in breve,
di quella che si pu\`o chiamare ``trattazione analitica del manovellismo'';
e cos\`i faremo noi.
\begin{figure}[hbt]
     \begin{center}
     \includegraphics[width=0.85\textwidth]{part2/manovellismi/FIG/man_ord.pdf}
\begin{picture}(0,0)(270,-3)
        \scriptsize{
        \put(38,109){$B$}
        \put(53.4,74.3){$\alpha$}
        \put(187,57){$C$}
        \put(5,59){$A$}
\put(44,66.6){\line(1,0){50}}
        \put(50,73.6){\rotatebox{-67}{$\curvearrowleft$}}
}
\end{picture}
\vskip -1mm
        \caption{\em Manovellismo ordinario centrato.}
     \label{fig:man_ord}
	\end{center}
\vskip -3mm
\end{figure}
\noindent Cominciamo col dare uno sguardo al moto del {\em piede di biella}\index{piede di biella} $C$, senza pretese di esattezza, giusto per renderci conto in modo 
qualitativo del suo movimento alternativo. Riportiamo a tal fine
di nuovo il manovellismo di figura
\ref{fig:quad_D_infinito_man} nella nuova figura
\ref{fig:man_ord},
e le quantit\`a cinematiche del punto $C$, ottenute da un'analisi
numerica, in figura
\ref{fig:vel_manovellismo_ordinario_centrato}.
\begin{figure}[hbt]
\begin{center}
    \includegraphics[width=0.55\textwidth]{part2/manovellismi/FIG/manovellismi/vel_manovellismo.pdf}
\begin{picture}(0,0)(170,25)
        \scriptsize{
        \put(80,118){$v_{\scriptscriptstyle C}$}
        \put(120,50){$a_{ \scriptscriptstyle C}$}
        \put(165,25){$\alpha$}
}
\end{picture}
        \caption{\em Velocit\`a e accelerazione nel manovellismo ordinario centrato.}
     \label{fig:vel_manovellismo_ordinario_centrato}
\end{center}
\vskip -3mm
\end{figure}
\noindent  In figura \ref{fig:man_ord}
si possono vedere le tracce
del {\em bottone di manovella}\index{bottone di manovella} $B$
e del piede di biella $C$,
mentre in figura  \ref{fig:vel_manovellismo_ordinario_centrato} sono riportati
i grafici di velocit\`a e accelerazione del punto $C$,
assumendo per la  velocit\`a angolare della manovella $AB$, di
lunghezza unitaria,
$\omega_{\scriptscriptstyle AB}=1$.
Notiamo subito, sia esaminando le tracce del piede di biella,
sia dal grafico della velocit\`a, che il punto $C$ deve rallentare 
nei punti morti (manovella con $\alpha=0$ e $\alpha=180$) fino a
fermarsi: presupposto ovvio affinch\'e il moto di $C$ si possa invertire.

\section{Espressioni Analitiche di Velocit\`a e Accelerazione del Piede
di Biella}

\noindent Con riferimento alla figura \ref{fig:moc_schematico} possiamo
innanzitutto considerare che la distanza $\overline{AC}$ \`e pari a

\begin{equation}
\overline{AC} = r \cos (\alpha) + l \cos (\beta)\,.
\end{equation}

\begin{wrapfigure}{r}{0.5\textwidth}
     \begin{center}
     \includegraphics[width=0.48\textwidth]{part2/manovellismi/FIG/moc_schematico.pdf}
     \end{center}
\begin{picture}(0,0)(32,3)
	\scriptsize{
        \put(64,97){$B$}
        \put(68,64){$\alpha$}
        \put(54,80){$r$}
        \put(120,78){$l$}
        \put(122,64){$\beta$}
        \put(164,65){$C$}
        \put(168,24){$x$}
        \put(168,16){$\dot x$}
        \put(168,8){$\ddot x$}
        \put(34,59){$A$}
\put(54,60){\line(1,0){30}}
\put(115,60){\line(1,0){40}}
}
\end{picture}
\vskip -3.95mm
        \caption{\em Rappresentazione schematica del manovellismo ordinario centrato.}
     \label{fig:moc_schematico}
\end{wrapfigure}

\noindent \`E facile perci\`o scrivere per la coordinata $x$, che indica lo
spostamento del piede di biella ($C$) dal punto morto esterno, 
\begin{equation}
x= r+l - r \cos (\alpha) - l \cos (\beta)\,.
\label{eq:formula_manovellismo}
\end{equation}
\noindent \`E opportuno (e tradizionale) introdurre a questo punto un parametro
che caratterizza fortemente il manovellismo: si tratta del rapporto tra il raggio
di manovella e la lunghezza della biella:
\begin{equation}
\lambda= \frac{r}{l}\,.
\end{equation}
\noindent Bench\'e questo rapporto possa, in teoria, spaziare in un intervallo
amplissimo, esso \`e soggetto a restrizioni quando si desidera una manovella 
libera di compiere intere rotazioni. In tal caso $\lambda< 0.28$\footnote
{Valore di lambda per i manovellismi del motore {\em Alfa Romeo} 1600 cc della
{\em Giulia GTA Sprint}.}\footnote{Al di l\`a degli ovvi problemi che 
nascerebbero aumentando $\lambda$, problemi principalmente dovuti 
alle possibili interferenze tra organi in moto alterno, di solito gli stantuffi,
il rapporto $\lambda$ si tiene comunque piuttosto distante dal limite teorico
$\lambda=0.5$, cio\`e da quello che vedrebbe la biella di lunghezza doppia
della manovella e che consentirebbe la completa rotazione di quest'ultima.
Dalla figura \ref{fig:moc_schematico} risulta $\lambda={\sin
\beta_{\scriptscriptstyle{(\alpha= 90^{\circ})}}}$. Pertanto, nelle posizioni
$\alpha=90^{\circ}$ e $\alpha=270^{\circ}$ le forze che agiscono sullo stantuffo
solleciteranno la biella
moltiplicate per $\frac{1}{\cos \beta_{\scriptscriptstyle {\rm max}}}=\frac{1}{\sqrt{1-\lambda^2}}$; \`e evidente che valori
elevati di $\lambda$ renderebbero
la trasmissione del moto inefficiente, dando luogo a ingenti spinte laterali sul glifo
e pregiudicando dal punto di vista strutturale
gli organi meccanici coinvolti.}
risulta essere un limite ragionevole per i manovellismi ordinari centrati impiegati nei motori a combustione interna e nei compressori.
Tale limitazione non vale per manovellismi dove la manovella si comporta in
realt\`a da bilanciere, come accade nella maggior parte delle presse a
ginocchiera, dove $\lambda$ pu\`o persino superare l'unit\`a.
Osserviamo che $r \sin \alpha = l \sin\beta$, e che
$\cos \beta=\sqrt{1-\lambda^2 \sin^2\alpha}$.
Tramite queste relazioni la \ref{eq:formula_manovellismo} diventa 
\begin{equation}
x= r[1 - \cos \alpha + \frac{1}{\lambda} (1 -  \sqrt{1-\lambda^2 \sin^2\alpha})]\,.
\label{eq:formula_manovellismo_1}
\end{equation}
\noindent Derivare rispetto al tempo una prima, quindi una seconda volta
la \ref{eq:formula_manovellismo_1},
allo scopo di ottenere velocit\`a e accelerazione del piede di biella,
non presenta certo difficolt\`a. Tale calcolo per\`o somiglierebbe
a un esercizio
meramente didattico di studio ``a livello liceale'' delle operazioni di
derivazione di semplici funzioni.
Come \`e noto, infatti, le derivate di radicali portano radicali, e 
le funzioni $\dot x$ e $\ddot x$, ottenute per questa via, dovrebbero
essere graficate con grande sforzo di calcolo,
oppure per via numerica, circostanza quest'ultima che vanificherebbe lo spirito
di questo percorso.


\noindent Si preferisce pertanto partire da una formula pi\`u semplice,
che approssimi lo spostamento del piede di biella, e ricavare da
quest'ultima le quantit\`a cinematiche derivate.
La strada \`e quella di esprimere il radicando della \ref{eq:formula_manovellismo_1} tramite i primi due termini del suo sviluppo in
serie, sviluppo eseguito considerando $\lambda$
(e non gi\`a $\alpha$) come variabile indipendente:
\begin{equation}
\sqrt{1-\lambda^2 \sin^2\alpha}\simeq 1 - \frac{1}{2} \lambda^2\sin^2\alpha \,.\notag
%\label{eq:svil_sqrt}
\end{equation}
\noindent Otteniamo cos\`i l'espressione arrestata al secondo ordine per lo
spostamento del piede di biella,
che risulter\`a tanto pi\`u  approssimato quanto pi\`u piccolo sar\`a
il rapporto $\lambda$:
\begin{equation}
x= r[1 - \cos \alpha + \frac{1}{2} \lambda \sin^2\alpha]\,;
\label{eq:formula_manovellismo_approx0}
\end{equation}
\noindent essendo poi
\begin{equation}
\sin^2\alpha= \frac{1 - \cos 2\alpha}{2}\,,
\label{eq:formula_manovellismo_approx}
\end{equation}
\noindent otteniamo finalmente   
\begin{equation}
x= r[1 +\frac{\lambda}{4} - \cos \alpha - \frac{\lambda}{4} \cos 2\alpha]\,.
\label{eq:formula_manovellismo_approx_1}
\end{equation}
\noindent La velocit\`a ``approssimata al second'ordine'' si
trova derivando rispetto al tempo la \ref{eq:formula_manovellismo_approx_1},
tenendo presente che ${\rm d}\alpha/{\rm d}t=\omega$:
\begin{equation}
\dot x= r\omega[\sin \alpha +\frac{\lambda}{2} \sin 2\alpha]\,,
\label{eq:formula_manovellismo_approx_vel}
\end{equation}
\noindent mentre per l'accelerazione avremo
\begin{equation}
\ddot x= r\omega^2[\cos \alpha +\lambda \cos 2\alpha]\,.
\label{eq:formula_manovellismo_approx_acc}
\end{equation}
\noindent Le espressioni \ref{eq:formula_manovellismo_approx_vel} e \ref{eq:formula_manovellismo_approx_acc}
mostrano chiaramente un andamento delle due grandezze cinematiche
composto da due armoniche.


\noindent Tramite le seguenti due osservazioni,
proviamo a evidenziare il lato a nostro avviso
pi\`u significativo
di questo studio analitico del manovellismo ordinario.
In primo luogo notiamo che per $\lambda$ molto
piccoli (biella molto lunga in rapporto al raggio di manovella), il moto del
piede di biella risulta coincidere sostanzialmente col moto
armonico:
\begin{equation}
\begin{split}
r-x=r\cos \alpha\,,\;\;\; \dot x=\omega r \sin\alpha\,\;\; {\rm e}\;\;\;\ddot x= \omega^2 r \cos\alpha\,. 
\end{split}
\label{eq:formule_primo_o}
\end{equation}
\begin{wrapfigure}{r}{0.5\textwidth}
     \begin{center}
     \includegraphics[width=0.48\textwidth]{part2/manovellismi/FIG/manovellismi/vel_manovellismo_x.pdf}
     \end{center}
\begin{picture}(0,0)(0,0)
	\scriptsize{
	\put(66,100){${\ddot x}(x)$}
	\put(80,168.5){${\dot x}(x)$}
}
\end{picture}
\vskip -5mm
        \caption{\em Velocit\`a e accelerazione del piede di biella in funzione del suo spostamento normalizzato per $r$ ($1^o$ ordine tratteggiato).}
     \label{fig:vel_manovellismo_x}
\end{wrapfigure}

\noindent \`E interessante esprimere
velocit\`a e accelerazione del piede di biella 
in funzione della coordinata $x$
anzich\'e dell'angolo $\alpha$, figura \ref{fig:vel_manovellismo_x}.
In tal caso, posto $\omega = 1$, quadrando e sommando membro a membro le formule che esprimono
velocit\`a e spostamento approssimati al primo ordine abbiamo
\begin{equation}
(r-x)^2 + {\dot x}^2=r^2\,.
\label{eq:cerchio_vel}
\end{equation}
\noindent La \ref{eq:cerchio_vel} rappresenta l'equazione di una circonferenza
di raggio $r$ e centrata a una distanza $r$ dal punto morto esterno: tale
risulter\`a quindi  l'andamento della
velocit\`a, graficata prendendo $x/r$ come ascissa.
Questo andamento verr\`a ovviamente modificato 
da valori di 
$\lambda$ non  trascurabili, quando cio\`e il moto del piede di 
biella si scoster\`a sensibilmente dal moto armonico.
La figura \ref{fig:vel_manovellismo_x} mostra l'andamento effettivo di
$\dot x(x)$, che tiene conto anche delle successive armoniche,
per $\lambda=0.28$.

\noindent La seconda osservazione riguarda l'accelerazione.
Il grafico di $\ddot x(x)$ \`e anch'esso riportato nella figura appena
citata e ha una forma che somiglia a quella di una parabola.
Fermandoci per\`o al primo ordine,
tale grafico sarebbe una retta. Infatti dalle \ref{eq:formule_primo_o}
otteniamo subito $\cos \alpha=(r-x)/r$ e poi $\ddot x = \omega^2(r-x)$.
Ponendo quindi come al solito $\omega=1$, il grafico di $\ddot x(x)$ sar\`a
una retta a quarantacinque gradi, passante dallo zero a distanza $r$ dal
punto morto esterno. In prima approssimazione, possiamo pertanto
attribuire al piede di biella una accelerazione che varia linearmente
tra i due punti morti: considerazione questa che porta con s\'e una decisa
facilit\`a di calcolo.
Un'ultima osservazione riguarda la possibilit\`a di calcolare
i due valori di accelerazione nei punti morti tramite la formula 
\ref{eq:formula_manovellismo_approx_acc}: otteniamo
$\ddot x(0^{\circ})= \ddot x(0)=r \omega^2(1+\lambda)$  e
$\ddot x(180^{\circ})= \ddot x(2r)=r \omega^2(1-\lambda)$. Lasciamo al 
paziente e volenteroso lettore il compito di mostrare che 
questi due valori
dell'accelerazione del piede di biella nei punti morti sono esatti:
essi non
risentono cio\`e dell'approssimazione introdotta nello spostamento dalla
\ref{eq:formula_manovellismo_approx_1}.

\section{Velocit\`a e Accelerazione del Piede di Biella: Analisi
mediante i Moti Relativi}

\noindent Come nello studio del quadrilatero, \`e possibile, anche in questo caso,
esprimere le quantit\`a cinematiche dei diversi punti del manovellismo
mediante le relazioni che legano queste grandezze quando
si considerano i moti relativi che avvengono tra i vari membri.

\begin{wrapfigure}{r}{0.5\textwidth}
     \begin{center}
     \includegraphics[width=0.48\textwidth]{part2/manovellismi/FIG/moc_schematico_graf.pdf}
     \end{center}
\begin{picture}(0,0)(32,7)
	\scriptsize{
        \put(58,94){$B$}
        \put(43,85){${\omega_{\scalebox{.5}{AB}}}$}
        \put(55,51){${\dot\omega_{\scalebox{.5}{AB}}}$}
        \put(102,91){${\omega_{\scalebox{.5}{BC}}}$}
        \put(124,84){${\dot\omega_{\scalebox{.5}{BC}}}$}
        \put(136,58){${{\bm v}_{\scalebox{.5}{C}}}$}
        \put(164,65){$C$}
        \put(34,59){$A$}
}
\end{picture}
\vskip -4mm
        \caption{\em Velocit\`a e accelerazioni nei moti relativi del manovellismo.}
     \label{fig:moc_schematico_graf}
\end{wrapfigure}

\noindent Con riferimento alla figura \ref{fig:moc_schematico_graf},
posizioniamo un sistema di riferimento traslante centrato nel
{\em bottone di manovella}\index{bottone di manovella} $B$. 
Possiamo cos\`i esprimere la velocit\`a del piede di biella come somma delle
due componenti:
quella relativa
e quella di trascinamento
\begin{equation}
{{\bm v}_{\scriptscriptstyle{C}}}= 
{{\bm v}_{\scriptscriptstyle{C}}}_{\scriptscriptstyle{{\rm rel}}}+
{{\bm v}_{\scriptscriptstyle{B}}}\,,
\label{eq:man_vel_graf0}
\end{equation}
\noindent ovvero esprimendo i due addendi di destra tramite le loro relazioni con
le velocit\`a angolari di biella e manovella,
\begin{equation}
{{\bm v}_{\scriptscriptstyle{C}}}= 
{\omega_{\scriptscriptstyle{BC}}}|\overrightarrow{BC}|\widehat{{\perp{BC}}}+
{{\bm v}_{\scriptscriptstyle{B}}}\,.
\label{eq:man_vel_graf1}
\end{equation}
\noindent La \ref{eq:man_vel_graf1} \`e un'equazione vettoriale nella quale
la direzione di tutti e tre i termini \`e nota
mentre $|{{\bm v}_{\scriptscriptstyle{C}}}|$ e 
${\omega_{\scriptscriptstyle{BC}}}$ risulteranno incogniti, supponendo data
la velocit\`a angolare della manovella. Si ripropone quindi
lo schema risolutivo illustrato in figura \ref{fig:f116},
dal quale ricaveremo
il valore della velocit\`a del piede di biella e la velocit\`a angolare
della biella.
Per quanto riguarda le accelerazioni possiamo scrivere
\begin{equation}
{{\bm a}_{\scriptscriptstyle{C}}}= 
{{\bm a}_{\scriptscriptstyle{C}}}_{\scriptscriptstyle{{\rm rel}}}+ 
{{\bm a}_{\scriptscriptstyle{B}}}\,. 
\label{eq:acc_man_graf0}
\end{equation}
\noindent Come abbiamo gi\`a chiarito altrove, conviene scrivere il termine
${{\bm a}_{\scriptscriptstyle{C}}}_{\scriptscriptstyle{{\rm rel}}}$
 come somma delle sue due
componenti: quella avente direzione parallela alla biella e diretta verso $B$
(componente normale) e quella ortogonale alla biella stessa.
Anche il termine ${{\bm a}_{\scriptscriptstyle{B}}}$ si pu\`o scrivere 
mediante la somma delle sue componenti,
quella radiale e quella tangenziale, le quali
risulteranno note
supponendo di conoscere la velocit\`a e la accelerazione angolari della
manovella.  Otteniamo in questo modo
\begin{equation}
{{\bm a}_{\scriptscriptstyle{C}}}= 
-{\omega_{\scriptscriptstyle{BC}}}^2 \overrightarrow{BC}+
{\dot\omega_{\scriptscriptstyle{BC}}}|\overrightarrow{BC}|\widehat{{\perp{BC}}}
-{\omega_{\scriptscriptstyle{AB}}}^2 \overrightarrow{AB}+
{\dot\omega_{\scriptscriptstyle{AB}}}|\overrightarrow{AB}|\widehat{{\perp{AB}}}\,.
\label{eq:acc_man_graf1}
\end{equation}
\noindent Ancora una volta siamo di fronte a una equazione vettoriale
dove le incognite sono
il modulo dell'accelerazione del piede di biella
$|{{\bm a}_{\scriptscriptstyle{C}}}|$ e l'accelerazione angolare della biella
${\dot\omega_{\scriptscriptstyle{BC}}}$, essendo note le direzioni di
tutti i vettori.
Una possibile soluzione \`e quindi, anche qui, 
lo schema grafico di figura \ref{fig:f116}; per inciso, la \ref{eq:acc_man_graf1} \`e formalmente identica alla \ref{eq:e_a_quad_schematico2},
la qual cosa non ci stupisce visto che stiamo di nuovo trattando
la cinematica di un quadrilatero, sia pure degenere.
Riteniamo di dover ribadire che il ruolo dei termini noti
e di quelli incogniti possono essere scambiati a discrezione: se considerassimo
assegnate la velocit\`a e l'accelerazione del piede di biella potremmo ricavare,
dalle \ref{eq:man_vel_graf1} e \ref{eq:acc_man_graf1},
le quantit\`a {${\omega_{\scalebox{.5}{AB}}}$} e
{${\dot\omega_{\scalebox{.5}{AB}}}$} della manovella $AB$.

\section{Manovellismi Non Ordinari}

\noindent Come \`e stato accennato nel paragrafo introduttivo, se ad andare
all'infinito \`e una delle due cerniere mobili di un quadrilatero si ottengono,
sostituendo opportunamente le aste e le cerniere degeneri con
membri appropriati,
i {\em manovellismi non ordinari}\index{manovellismo!non ordinario}. Questi manovellismi, ovviamente {\em cinematicamente
equivalenti}\index{equivalenza cinematica} ai quadrilateri degeneri che li 
originano,
vengono spessissimo chiamati, con ragione,
meccanismi 
{\em a glifo rotante}\index{glifo!rotante} ovvero
{\em a glifo oscillante}\index{glifo!oscillante},
dando per scontato che un preciso membro di tali meccanismi
sia un glifo. Non ci stanchiamo
di ripetere che, a rigore, i due membri dell'accoppiamento glifo-cursore
possono sempre essere scambiati tra loro senza alcuna mutazione
della cinematica del sistema.
Prima di entrare nel vivo di questo paragrafo, desideriamo esporre una breve
precisazione, che per la verit\`a avrebbe potuto precedere anche la
trattazione dei manovellismi ordinari, ma che si rende ancora
pi\`u necessaria in questa sede: cosa
intendiamo, esattamente, con la locuzione
{\em equivalenza cinematica}\index{equivalenza
cinematica}? Ancora pi\`u esplicitamente, quando asseriamo che la cerniera
$D$ e il bilanciere $DC$ del quadrilatero
degenere di figura \ref{fig:quad_D_infinito_man} vengono sostituiti
dalla coppia glifo-cursore $\sigma$, ottenendo in tal modo il
manovellismo ordinario
equivalente, di quale equivalenza parliamo?
Nello studio dei manovellismi si parla di meccanismo equivalente
quando la biella (o un corpo che la sostituisce) sar\`a interessata dallo
stesso moto piano, mantenendo per tutti i punti di biella le stesse
traiettorie, le stesse velocit\`a e le stesse accelerazioni. Nel
caso che abbiamo studiato dei manovellismi ordinari l'equivalenza
del comportamento della biella di figura \ref{fig:quad_D_infinito_man} con
la cerniera $D$ a grande distanza o con la coppia glifo-cursore $\sigma$ 
sembra all'autore sufficientemente evidente.
Qualche attenzione in pi\`u \`e richiesta
nello studio dei manovellismi non ordinari.

\noindent Giusto per fissare qualche idea circa il comportamento di 
un quadrilatero che degenera tramite l'allontanamento di una delle
due cerniere (perfettamente equivalenti) $B$ oppure $C$, eseguiamo un tentativo
di analisi qualitativa tramite i nostri consueti (e rudimentali)
strumenti di indagine numerica.
\begin{figure}[hbt]
\centering
\begin{minipage}[b]{0.52\textwidth}
\centering
\includegraphics[width=0.9\textwidth]{part2/manovellismi/FIG/manovellismi/c_infinito.pdf}
\begin{picture}(0,0)(138,-6)
\scriptsize{
\put(11,65){$B$}
\put(110,107){$C$}
\put(110,100){\rotatebox{42.45}{$\longrightarrow$}}
\put(113,98){$\infty$}
\put(2,32){$A$}
\put(50,23){$D$}
\put(52.5,37.5){$\alpha$}
\put(78,60){$\beta$}
\put(101,71){$\gamma$}
\color{red}
\put(99.3,83.3){\rotatebox{-47.55}{\line(1,0){15}}}
\put(73,60){\rotatebox{-47.55}{\line(1,0){15}}}
\put(48,38){\rotatebox{-47.55}{\line(1,0){15}}}
}
\end{picture}
      \caption{\em Quadrilatero degenere ottenuto mandando la cerniera $C$ a distanza infinita.}
 \label{fig:c_infinito}
\end{minipage}\hfill\hfill
\begin{minipage}[b]{0.40\textwidth}
\centering
\includegraphics[width=0.9\textwidth]{part2/manovellismi/FIG/glifi_c_infinito.pdf}
\vspace*{10mm}
\begin{picture}(0,0)(139,18)
\scriptsize{
\put(13,65){$B$}
\put(110,107){$C$}
\put(110,100){\rotatebox{42.45}{$\longrightarrow$}}
\put(113,98){$\infty$}
\put(0,32){$A$}
\put(54,14){$D$}
\put(50,44){$a)$}
\put(74,65){$b)$}
\put(100,91){$c)$}
}
\end{picture}
      \caption{\em Possibili meccanismi a glifo mobile derivati dalla figura a lato.}
     \label{fig:glifi_c_infinito}
\end{minipage}
\end{figure}
\noindent Abbiamo scelto, con criterio del tutto arbitrario, di rendere
improprio l'asse della cerniera $C$. Come nel caso dei manovellismi ordinari,
la scelta della direzione secondo la quale mandare tale cerniera 
a grande distanza non
\`e indifferente. Quando la cerniera, come in questo caso, \`e mobile
 tale direzione muta durante il 
movimento del meccanismo. Notiamo per\`o che, scelta in un dato istante
la direzione del punto improprio $C$, essa rimarr\`a, durante il movimento,
costante rispetto alla direzione del segmento $BD$;
in altre parole l'angolo tra la direzione del punto improprio
e $BD$ si manterr\`a costante.
Ci\`o accade in quanto il triangolo $\triangle{BCD}$ presenta angolo costante
e nullo in $C$. Ritenendo quindi le lunghezze dei lati
$BC$ e $CD$ costanti, anche se infinite, rimarranno invariati durante
il movimento anche gli altri suoi angoli\footnote
{
La lunghezza del lato $BD$, ovviamente, non si mantiene costante. Essa
per\`o risulta infinitesimale rispetto alle lunghezze degli altri
due lati, pertanto la sua variazione provoca alterazioni infinitesimali
e trascurabili del triangolo $\triangle{BCD}$.
}.
Nel caso delle figure \ref{fig:c_infinito} e \ref{fig:glifi_c_infinito}
la direzione scelta coincide con la perpendicolare al segmento $BD$,
scelta che consente di ottenere {\em manovellismi non ordinari
centrati}\index{manovellismo!non ordinario}.
\noindent In figura \ref{fig:c_infinito} sono riportate tre curve di biella
relative a tre punti che giacciono sul membro $DC$.
Lo scopo di tale figura \`e quello 
di rendere maggiormente evidente il moto del piano di
biella nei pressi del bilanciere  ``improprio''.
Da questa analisi si pu\`o notare che nei punti di intersezione con 
il bilanciere $DC$, le curve di biella sono 
ortogonali ad esso. La figura, rimarcando questa circostanza, facilita
l'intuizione. Tali direzioni
erano peraltro prevedibili a valle di semplici ragionamenti geometrici, osservando che 
tutti i punti di biella che giacciono su $DC$ descrivono, nel loro
moto relativo a tale membro,  delle circonferenze di centro $C$,
le cui tracce sono in figura rappresentate da brevi segmenti:
$\alpha$, $\beta$ e $\gamma$.
La condizione poi che il punto $C$ si trovi a grande distanza ci
autorizza, come abbiamo appena sopra chiarito, a ritenere costante
l'angolo compreso tra $DC$ e il segmento $DB$,
pari nel nostro caso a $90^{\circ}$. 
Queste considerazioni ci consentono di sostituire la cerniera in $C$ con
gli accoppiamenti prismatici che danno origine alla figura
\ref{fig:glifi_c_infinito}.
\noindent I tre meccanismi a), b) e c) sono, da un punto
di vista cinematico, un solo meccanismo, essendo il piano di biella lo stesso
per le tre soluzioni e rendendo ridondante la distinzione della cerniera
in $B$ in tre cerniere separate.
I tre sistemi articolati   
presentano un glifo sul bilanciere $CD$ dove scorre la relativa biella.
Oltre a ripetere (lo abbiamo detto gi\`a troppe volte) che il ruolo
di chi porta la
scanalatura (glifo) e di chi invece fa da cursore possono indifferentemente
essere scambiati, osserviamo che tali
accoppiamenti prismatici possono essere dislocati in qualsiasi punto della
biella equivalente. 
\noindent In particolare, considerando il meccanismo a) di figura \ref{fig:glifi_c_infinito}, che riteniamo maggiormente significativo,
possiamo ottenere i sistemi articolati equivalenti
 di figura \ref{fig:c_infinito_equivalenti}.

\begin{figure}[hbt]
\centering
\includegraphics[width=0.9\textwidth]{part2/manovellismi/FIG/c_infinito_equivalenti.pdf}
\begin{picture}(0,0)(320,0)
\scriptsize{
\put(46,75){\tiny biella equivalente}
\put(45,75){\rotatebox{220}{$\longrightarrow$}}
\put(161,75){\tiny biella equivalente}
\put(159,75){\rotatebox{220}{$\longrightarrow$}}
\put(268,86){\tiny biella equivalente}
\put(254,85){\rotatebox{180}{$\longrightarrow$}}
\put(94,14){\tiny glifo}
\put(90,22){\rotatebox{180}{$\longrightarrow$}}
\put(146,45){\tiny glifo}
\put(156,41){\rotatebox{0}{$\longrightarrow$}}
\put(253,45){\tiny glifo}
\put(264,41){\rotatebox{0}{$\longrightarrow$}}
\put(0,50){$1)$}
\put(113,50){$2)$}
\put(222,50){$3)$}
}
\end{picture}
      \caption{\em Manovellismi non ordinari centrati equivalenti.}
 \label{fig:c_infinito_equivalenti}
\end{figure}

\noindent I tre meccanismi qui riportati presentano glifi
oscillanti attorno alla cerniera $D$, dove essi sono incernierati.
La lunghezza della ``biella'' \`e, dalla configurazione $1)$ alla
configurazione $3)$,  via via decrescente fino ad annullarsi.
Nel meccanismo $3)$ il piano di biella con tutte le sue curve
va pensato infatti solidale al cursore che scorre nel glifo.
Occorre per\`o precisare che, quando si parla di glifi oscillanti e ci si riferisce
appunto alla figura \ref{fig:c_infinito_equivalenti}, caso $3)$, ancora
pi\`u delle curve
dei punti di biella interessa il movimento del glifo stesso,
il quale \`e un bilanciere, e come tale si muove.

\section{Meccanismo a Ritorno Rapido}\index{meccanismo a ritorno rapido}\index{guida di Fairbairn}

La figura \ref{fig:c_infinito_quick} riporta nuovamente il meccanismo
di figura \ref{fig:c_infinito_equivalenti}, caso $3)$, ruotato
di un angolo opportuno, in modo da porre il suo telaio in verticale. 
\begin{figure}[hbt]
\centering
\includegraphics[width=0.9\textwidth]{part2/manovellismi/FIG/manovellismi/c_infinito_quick.pdf}
\begin{picture}(0,0)(250,0)
\scriptsize{
\put(235,245){$V$}
\put(136,125){$B$}
\put(73,158){$A$}
\put(73,62){$D$}
\put(180,50){$C$}
\put(175,50){\rotatebox{-42}{$\longrightarrow$}}
\put(173,40){$\infty$}
}
\end{picture}
      \caption{\em Meccanismo a ritorno rapido.}
 \label{fig:c_infinito_quick}
\end{figure}
\noindent A una rotazione completa della manovella
$AB$, di solito a velocit\`a $\omega$ costante, corrispondono
un'oscillazione di andata del glifo e una di ritorno che si svolgono
impiegando tempi diversi fra loro. In particolare, come abbiamo gi\`a 
visto nello studio dei quadrilateri, pagina
\pageref{eq:angoli_andata_ritorno}, tali tempi
saranno proporzionali ai rispettivi angoli di manovella.
Nella figura poc'anzi citata, ottenuta da un'analisi numerica,
non trova espressione grafica il pattino
che deve permettere al punto $B$ di scorrere
in una apposita scanalatura dell'asta $DV$, che per questo
assume generalmente il nome di glifo. Con la speranza  di porre
rimedio a tale lacuna grafica, riportiamo la prossima illustrazione,
nella quale
si pu\`o anche apprezzare l'utilizzo  concreto di questo manovellismo.
\begin{figure}[hbt]
\centering
\includegraphics[width=0.9\textwidth]{part2/manovellismi/FIG/fairbairn.pdf}
\begin{picture}(0,0)(300,10)
\scriptsize{
\put(42,376){\tiny andata lenta}
\put(40,347){\tiny ritorno rapido}
\put(258,298){$V$}
\put(251,278){$K$}
\put(175,287){\tiny slitta portautensile}
\put(135,107){$B$}
\put(50,211){$\omega=cost$}
\put(66,126){$\alpha$}
\put(70,106){$\beta$}
\put(42,164){$A$}
\put(32,28){$D$}
}
\end{picture}
      \caption{\em Meccanismo di Fairbairn.}
 \label{fig:fairbairn}
\end{figure}
\noindent La {\em guida di Fairbairn}\index{glifo!di Fairbairn}\index{Fairbairn, guida di},
rappresentata in figura \ref{fig:fairbairn},
si ottiene dal meccanismo
di figura \ref{fig:c_infinito_quick} con l'aggiunta di
una bielletta (o biscottino) di
snodo oppure di un opportuno accoppiamento prismatico (\cite{sesini1}, Fig. IV-8,
pag. 108, ma anche \cite{hartog}, pag. 168) in modo tale da permettere alla
sommit\`a
del glifo di azionare una slitta rettilinea. Tale soluzione costituiva,
fra l'altro,
il cuore di una ben nota macchina utensile che necessita di una corsa di andata
``lenta'' e di un ritorno ``rapido''.
Le tracce lasciate dalla sommit\`a del glifo, durante il suo movimento, dovrebbero dare
conto, mediante la loro spaziatura, delle differenti velocit\`a di $V$ durante
un ciclo di manovella. Le stesse spaziature, proiettate in pianta, dovrebbero
altrettanto rendere l'idea della velocit\`a del punto $K$, quindi della velocit\`a
del portautensile.
Anche qui, come nel caso del manovellismo ordinario centrato, \`e possibile
impostare le equazioni che legano l'angolo $\beta$ percorso dal glifo
all'angolo $\alpha$ di manovella;
derivando queste equazioni
rispetto ad $\alpha$ oppure rispetto al tempo si otterranno infine
le equazioni della cinematica del glifo oscillante centrato. Analogamente,
non presenterebbe certo difficolt\`a alcuna mettere
 in relazione l'angolo di oscillazione del glifo con lo spostamento
della slitta portautensile, della quale si otterrebbero finalmente
 velocit\`a e accelerazione, come mostrano
importanti autori, ad esempio \cite{hartog}, pag. 168.
Non ce la sentiamo per\`o di seguirli in tale percorso,
e ci accontentiamo dell'analisi numerica
riportata in modo qualitativo nelle figure
 \ref{fig:c_infinito_quick} e
 \ref{fig:fairbairn}, mediante le tracce del bottone di manovella
 e della estremit\`a del particolare glifo che abbiamo scelto di
proporre come esempio.

\noindent A nostro parere, dalla trattazione analitica del glifo
non emergono importanti spunti didattici,
al di l\`a dei meri passaggi trigonometrici:
i volenterosi e curiosi studenti
sapranno agilmente trovare tali calcoli nella citata bibliografia, se
interessati.
Per contro, colui che si vedr\`a assegnato il compito di progettare realmente
un meccanismo a ritorno rapido si affider\`a ai contemporanei codici numerici
di simulazione di meccanismi che  forniscono, con precisione, tutte le
grandezze cinematiche in gioco.
Inoltre sottolineiamo di nuovo che
 l'importanza di questo meccanismo risiede nella 
possibilit\`a di avere tempi di andata e ritorno sensibilmente diversi fra
loro, in modo da non sprecare tempo prezioso nella fase
oziosa del moto dell'utensile,
pi\`u che nella conoscenza puntuale della velocit\`a o della
accelerazione di quest'ultimo.
Nel nostro caso, invero un po' estremo,  abbiamo $\alpha_a=257^{\circ}$ e $\alpha_r=103^{\circ}$ che permettono
al tempo di andata, corrispondente al tempo di lavoro dell'utensile, di essere
pi\`u che doppio rispetto al tempo ``morto'' del ritorno.
Ai tempi in cui le macchine utensili venivano realizzate tramite pregiate
fusioni in ghisa, il glifo di {\em Fairbairn} era l'azionamento
principale della {\em limatrice}\index{limatrice}.
A proposito di questa applicazione, sottolineiamo che
la velocit\`a massima, raggiunta dalla slitta durante
la fase di andata (lenta), si manifesta quando il punto
$K$ \`e allineato coi punti $A$, $D$ e $B$. Essa \`e facilmente calcolabile
e vale, con $K$ in tale
posizione, $|v_{max}|=|\omega| |\overrightarrow{AB}||\overrightarrow{KD}|/|\overrightarrow{BD}|$; essa
rappresenta la velocit\`a  di taglio nominale dell'utensile.

\noindent Mi si permetta (tanto  tempo \`e trascorso quindi mi sento al sicuro da
rivendicazioni o proteste di carattere commerciale) di citare la ``gloriosa
e indistruttibile'' limatrice
Garavaglia.
Tutte le officine meccaniche rispettabili
ne avevano almeno una a disposizione. L'autore per\`o confessa che, gi\`a
trent'anni or sono, queste macchine beneficiavano di
notevolissimi periodi di riposo.

\section{Cinematica del Meccanismo a Ritorno Rapido: Analisi 
mediante i Moti Relativi}

\noindent La soluzione grafica che abbiamo esposto per l'analisi cinematica
di quadrilateri e manovellismi ordinari si pu\`o naturalmente estendere
anche al caso del meccanismo a ritorno rapido.
\noindent Riferendoci alla figura \ref{fig:soluzione_grafica_glifo}
posizioniamo, in questo caso, un sistema di riferimento relativo (al quale cio\`e
verranno riferite le grandezze relative) che ruota solidalmente con il
glifo ed \`e centrato in $D$.
Siamo in grado ora di esprimere la velocit\`a del punto $B$ (bottone di
manovella) mediante la somma delle due componenti: quella relativa
e quella di trascinamento:
\begin{equation}
{{\bm v}_{\scriptscriptstyle{B}}}_{\scriptscriptstyle{{\rm ass}}}= 
{{\bm v}_{\scriptscriptstyle{B}}}_{\scriptscriptstyle{{\rm rel}}}+
{{\bm v}_{\scriptscriptstyle{B}}}_{\scriptscriptstyle{{\rm tr}}}\,,
\label{eq:glifo_vel_graf0}
\end{equation}
\noindent ovvero, 
\begin{equation}
{{\bm v}_{\scriptscriptstyle{B}}}_{\scriptscriptstyle{{\rm ass}}}= 
{{\bm v}_{\scriptscriptstyle{B}}}_{\scriptscriptstyle{{\rm rel}}}+
{\omega_{\rm g}}|\overrightarrow{DB}|\widehat{{\perp{DB}}}\,.
\label{eq:glifo_vel_graf1}
\end{equation}

\begin{wrapfigure}{r}{0.5\textwidth}
     \begin{center}
     \includegraphics[width=0.48\textwidth]{part2/manovellismi/FIG/soluzione_grafica_glifo.pdf}
     \end{center}
\begin{picture}(0,0)(32,7)
	\scriptsize{
        \put(130,113){$B$}
        \put(41,37){$D$}
        \put(45,162){$A$}
        \put(34,71){$\scriptstyle{y'}$}
        \put(71,72){$\scriptstyle{x'}$}
        \put(62,130){${\omega_{\rm{m}}}$,}
        \put(77,130){${\dot\omega_{\rm{m}}}$}
        \put(63,112.5){${\omega_{\rm{g}}}$,}
        \put(78,112.5){${\dot\omega_{\rm{g}}}$}
        \put(132,157){${{\bm v}_{\scalebox{.5}{B}}}$,}
        \put(143,157){${{\bm a}_{\scalebox{.5}{B}}}$}
}
\end{picture}
\vskip -1.3mm
        \caption{\em Velocit\`a e accelerazioni nei moti relativi del meccanismo a ritorno rapido.}
     \label{fig:soluzione_grafica_glifo}
\end{wrapfigure}
\noindent La \ref{eq:man_vel_graf1} \`e un'equazione vettoriale nella quale
la direzione di tutti e tre i termini \`e nota
mentre
$|{{\bm v}_{\scriptscriptstyle{B}}}_{\scriptscriptstyle{{\rm rel}}}|$
e ${\omega_{\rm g}}$ sono incogniti, una volta data
la velocit\`a angolare della manovella. Anche qui riproponiamo quindi
lo schema risolutivo illustrato in figura \ref{fig:f116},
mediante il quale ricaveremo i valori delle incognite.

\noindent Per quanto riguarda le accelerazioni scriviamo
\begin{equation}
{{\bm a}_{\scriptscriptstyle{B}}}_{\scriptscriptstyle{{\rm ass}}}= 
{{\bm a}_{\scriptscriptstyle{B}}}_{\scriptscriptstyle{{\rm rel}}} 
+{{\bm a}_{\scriptscriptstyle{B}}}_{\scriptscriptstyle{{\rm tr}}} 
+{{\bm a}_{\scriptscriptstyle{B}}}_{\scriptscriptstyle{{\rm cor}}}\,,
\label{eq:acc_glifo_graf0}
\end{equation}
\noindent dove si nota la presenza del termine dell'accelerazione di {\em Coriolis}
dovuta alla rotazione del sistema di riferimento relativo.
Come al solito sostituiamo alle accelerazioni le loro componenti tangenziali
e normali ottenendo
\begin{equation}
{{\bm a}_{\scriptscriptstyle{B}}}_{\scriptscriptstyle{{\rm ass}}}= 
{{\bm a}_{\scriptscriptstyle{B}}}_{\scriptscriptstyle{{\rm rel}}} 
-{\omega_{\rm g}}^2 \overrightarrow{DB} + \dot{\omega}_{\rm g}|\overrightarrow{DB}|\widehat{{\perp{DB}}}+
2{\bm \omega}_{\rm g}\times
{{\bm v}_{\scriptscriptstyle{B}}}_{\scriptscriptstyle{{\rm rel}}}\,.
\label{eq:acc_glifo_graf1}
\end{equation}
\vskip 1mm
\noindent Osserviamo che risultano note le direzioni di tutti i termini
di questa equazione.
Contando poi i termini della \ref{eq:acc_glifo_graf1} da sinistra,
affermiamo che gli unici moduli incogniti
sono quelli dei vettori che occupano il primo e il terzo posto
dopo il degno di uguaglianza. 
Perci\`o otteniamo ancora un'equazione vettoriale
dove le incognite sono
il modulo dell'accelerazione relativa di $B$,
$|{{\bm a}_{\scriptscriptstyle{B}}}_{\scriptscriptstyle{{\rm rel}}}|$,
e il modulo dell'accelerazione angolare del glifo
${\dot\omega_{\rm g}}$.
Il riferimento per una soluzione grafica resta dunque 
il solito, cio\`e lo schema grafico di figura \ref{fig:f116}.
Ammettiamo che, passando dall'esempio di pagina \pageref{fig:f116}
all'approccio grafico alla cinematica dei quadrilateri e infine
alla stessa analisi operata sui manovellismi, ammettiamo
dicevo che la nostra esposizione si 
\`e fatta via via pi\`u succinta, sottacendo di passo in passo
alcune precisazioni presenti nei primi esempi.
Il motivo di questa progressiva reticenza sta nel pudore, forse ingiustificato,
che nasce quando si \`e costretti a ripetere molte volte contenuti non
molto dissimili da quelli espressi poco addietro.
 Siamo certi che lo studente volenteroso
sapr\`a rintracciare in questo percorso tutte le informazioni che
sono state via via tralasciate.


\section{Manovellismi non Ordinari con Due Membri Rotanti}\index{glifo!rotante}

\noindent Nel precedente paragrafo abbiamo analizzato il meccanismo a ritorno
rapido nel quale, stante l'opportunit\`a di realizzare il glifo sull'asta
$DV$, figura \ref{fig:fairbairn}, abbiamo ottenuto un glifo oscillante,
che si comporta cio\`e come un  bilanciere, e abbiamo citato una delle sue
pi\`u famose applicazioni che consiste nella movimentazione di
utensili da taglio.
\begin{wrapfigure}{r}{0.4\textwidth}
     \begin{center}
\includegraphics[width=0.38\textwidth]{part2/manovellismi/FIG/manovellismi/vel_b_infinito.pdf}
\end{center}
\begin{picture}(0,0)(0,0)
\scriptsize{
\put(55,70){${\omega_{\scalebox{.5}{DC}}}/{\omega_{\scalebox{.5}{AK}}}$}
        \put(141,27){$\alpha$}
}
\end{picture}
\vskip -3.2mm
      \caption{\em Velocit\`a del membro $DC$.}
     \label{fig:vel_b_infinito}
\end{wrapfigure}
\noindent Non \`e difficile, adottando
opportune configurazioni, ottenere per tale meccanismo un glifo
 rotante (sempre che
la volont\`a del progettista lo foggi come un glifo). L'analisi
qualitativa di tali meccanismi sar\`a oggetto di questo paragrafo.
\noindent Mandiamo
all'infinito la cerniera $B$ di un opportuno quadrilatero di partenza,
avente geometria tale da poter funzionare (si riveda la regola di {\em Grashof})
 a ``doppia-manovella''.
Abbiamo scelto la cerniera $B$ anzich\'e la $C$ semplicemente per
mostrare al lettore la completa equivalenza, in questa circostanza,
 delle
due cerniere. Nella nostra analisi, che svolgeremo come abbiamo fatto
altre volte tramite l'approccio numerico,
la manovella a velocit\`a costante \`e sempre la $AB$, che lascia
le sue tracce equispaziate; questo ci
consentir\`a di avere sulla manovella $DC$, l'elemento che di solito
\`e costituito da un glifo come in figura \ref{fig:glifo_rotante},
la velocit\`a variabile.
 
\begin{figure}[htb]
\begin{center}
\includegraphics[width=.8\textwidth]{part2/manovellismi/FIG/manovellismi/b_infinito.pdf}
\end{center}
\begin{picture}(0,0)(-35,20)
\scriptsize{
\put(92,330){$B$}
\put(100,330){\rotatebox{67}{$\longrightarrow$}}
\put(104,328){$\infty$}
\put(-2,225){$C$}
\put(21,212){$K$}
\put(88,177){$A$}
\put(142,177){$D$}
\put(154,332){$\beta_2=86^{\circ}$}
\put(135,259){$\alpha_2=68^{\circ}$}
\put(115,100){$\alpha_1=292^{\circ}$}
\put(121,38){$\beta_1=274^{\circ}$}
}
\end{picture}
\vskip -3mm
      \caption{\em Manovellismo non ordinario che permette la rotazione completa
di due membri.}
 \label{fig:b_infinito}
\end{figure}

\noindent In particolare, la figura \ref{fig:vel_b_infinito} riporta l'andamento
del rapporto tra la velocit\`a angolare del glifo, $\omega_{DC}$, e quella
della manovella (presente nella figura \ref{fig:glifo_rotante}), $\omega_{AK}$.
Se volessimo valorizzare il massimo squilibrio che si pu\`o ottenere
tra la rotazione della manovella e quella del glifo,
come abbiamo esposto nel paragrafo {\ref{q_squilibrio}},
dovremmo anche qui trovare una posizione ben
precisa ``di montaggio'' di questo meccanismo.
Quest'ultimo quindi, se opportunamente sfruttato, potr\`a presentare, a fronte di due
semi-rotazioni contigue di $180^{\circ}$ del glifo, due angoli di manovella
sensibilmente diversi tra loro.
\begin{figure}[b]
\begin{center}
\includegraphics[width=.8\textwidth]{part2/manovellismi/FIG/glifo_rotante.pdf}
\end{center}
\begin{picture}(0,0)(-35,20)
\scriptsize{
\put(12,222){$C$}
\put(34,219){$K$}
\put(88,179){$A$}
\put(163,183){$D$}
\put(180,274){$\beta$}
\put(100,230){$\alpha$}
\put(52,142){$\omega={\rm cost}$}
}
\end{picture}
\vskip -5mm
      \caption{\em Manovellismo non ordinario 
 con glifo rotante.}
 \label{fig:glifo_rotante}
\end{figure}
 Riferendoci all'esempio di figura
\ref{fig:b_infinito}, figura sulla quale si 
basa quella successiva, \ref{fig:glifo_rotante}, 
troviamo indicate le coppie di angoli
$\alpha_1$, $\beta_1$, $\alpha_2$, $\beta_2$ che realizzano in questo caso
il {\em massimo squilibrio} pari a
\begin{equation}
s={{292^{\circ}-68^{\circ}}\over{360^{\circ} -292^{\circ} +68^{\circ}}} = 1.6\,.
\label{eq:esempio_squilibrio_glifo}
\end{equation} 
\noindent A una rotazione completa della manovella $AK$, a velocit\`a costante,
corrisponderanno quindi due semi-rotazioni del glifo che si completeranno
in tempi diversi tra loro con $t_a/t_r=1.6$. Si potrebbe quindi impiegare questo
glifo come manovella di
un manovellismo ordinario centrato, per il quale le corse di andata e ritorno
si svolgerebbero in tempi il cui rapporto reciproco \`e stato or ora indicato. 
Questa soluzione \`e stata talvolta adottata (glifo di {\em Whitworth}) in sostituzione
del glifo di {\em Fairbairn}, \cite{sesini1}, pag. 109.

\section{Quadrilateri Doppiamente Degeneri e Altre Considerazioni}

Tratteremo brevemente soltanto il caso in cui, spostando le due cerniere mobili,
$B$ e $C$, in due punti impropri distinti del piano, si d\`a luogo
a un interessante meccanismo di discreta importanza nelle applicazioni pratiche:
il {\em giunto di {\em Oldham}}\index{giunto di Oldham}.
La figura \ref{fig:oldham} mostra un quadrilatero a doppia-manovella
dove il telaio $AD$ misura circa la ventesima parte delle altre aste.
Siamo costretti, si fa per dire, a svelare ora l'arcano che sta alla base
delle figure dei manovellismi riportati in questo capitolo.
\begin{figure}[htb]
\begin{center}
\vspace{-0cm}\hspace{-.10cm}\includegraphics[width=.8\textwidth]{part2/manovellismi/FIG/manovellismi/oldham.pdf}
\hbox{\vspace{-12cm}\hspace{-4cm}\includegraphics[width=0.3\textwidth]{part2/manovellismi/FIG/oldham_particolare.pdf}}
\end{center}
\begin{picture}(0,0)(-55,15)
\scriptsize{
\put(102,333){$B$}
\put(110,330){\rotatebox{90}{$\longrightarrow$}}
\put(114,333){$\infty$}
\put(232,293){$C$}
\put(234,287){\rotatebox{43.4}{$\longrightarrow$}}
\put(232,283){$\infty$}
\put(111,178){$A$}
\put(125,178){$D$}
\put(175,45){$A$}
\put(198,45){$D$}
\put(205,128){$T$}
\put(72,264){$P$}
}
\end{picture}
\vskip -5mm
      \caption{\em Quadrilatero doppiamente degenere e relativo meccanismo equivalente.
}
 \label{fig:oldham}
\end{figure}
Di fatto,
le analisi numeriche che danno origine alle figure \ref{fig:quad_D_infinito_man},
\ref{fig:c_infinito}, \ref{fig:c_infinito_quick} e \ref{fig:b_infinito} sono
state eseguite mediante lo  stesso (semplice e imperfetto) codice che genera
anche le analisi dei quadrilateri del precedente capitolo dedicato
allo studio dei quadrilateri articolati. In ambito numerico,
esiste infatti la possibilit\`a di analizzare, a volte con un certo agio,
anche quelle situazioni limite lasciate
giustamente dalla matematica nella sfera delle idee astratte, come sono in
generale le grandezze che tendono all'infinito.
Inoltre, le scienze ingegneristiche, pur abbracciando senza perplessit\`a il
concetto di distanza infinita, possono in altri casi permettersi il 
lusso  di accontentarsi di qualcosa di meno e magari pi\`u a portata di mano.
\`E il nostro caso, quando pretendiamo di trattare numericamente
le lunghezze, teoricamente infinite, delle aste $AD$ e $CD$ del quadrilatero
di figura \ref{fig:quad_D_infinito_man} sostituendo tali membri con
aste di lunghezza finita, bench\'e 
cinquanta o sessanta volte maggiore rispetto alle altre due.
Del resto, a nessun ingegnere di buon senso verrebbe in mente di obiettare
che la traiettoria del cursore, in figura \ref{fig:quad_D_infinito_man},
indicata con $\sigma$,
sia un arco di circonferenza anzich\'e un segmento di retta.
Torniamo alla figura \ref{fig:oldham} e  notiamo che la manovella $CD$
\`e omocinetica alla $AB$ (nell'approssimazione numerica adottata per le
lunghezze infinite delle tre aste $AB$, $BC$, e $CD$), cio\`e le due manovelle
hanno la stessa velocit\`a angolare.
La curva percorsa dal generico punto di biella $P$ approssimer\`a
tanto meglio una circonferenza quanto pi\`u distante si trover\`a dal telaio
(come nella figura) e anch'essa risulta percorsa con velocit\`a di modulo
costante. Risulta poi evidente che le curve di biella dei punti giacenti
(punti non rappresentati in figura per evitare confusione nel disegno)
 sulle manovelle, taglieranno queste ultime
seguendo direzioni ad esse ortogonali\footnote{\`E facile convincersi
che passando a lunghezze delle tre aste $AB$, $BC$ e $CD$ cento volte
superiori a quelle rappresentate  nella figura, i punti $A$ e $D$ verrebbero
a coincidere nel disegno in un punto $V$. Il meccanismo diventerebbe cos\`i
un triangolo $\triangle VBC$ che ruota attorno al suo vertice trascinando
in questa rotazione, oltre alle due manovelle, tutto il piano di biella.}.
\begin{figure}[hbt]
\centering
\begin{minipage}[b]{0.48\textwidth}
\centering
\includegraphics[width=0.95\textwidth]{part2/manovellismi/FIG/oldham_schema.pdf}
\begin{picture}(0,0)(190,0)
\scriptsize{
\put(36,40){$A$}
\put(48,75){$\scriptstyle{M_1}$}
\put(105,45){$\scriptstyle{M_2}$}
\put(110,110){$T$}
\put(85,57){$D$}
\put(116.5,57.5){$\omega$}
\put(35,51){$\omega$}
\put(33,54){\rotatebox{30}{$\curvearrowleft$}}
\put(110,62){\rotatebox{-60}{$\curvearrowleft$}}
}
\end{picture}
      \caption{\em Rappresentazione funzionale del giunto di Oldham.}
 \label{fig:oldham_schema}
\end{minipage}\hfill
\begin{minipage}[b]{0.48\textwidth}
\centering
\includegraphics[width=0.95\textwidth]{part2/manovellismi/FIG/oldham_commerciale.pdf}
\begin{picture}(0,0)(100,0)
\scriptsize{
\put(30,80){$\scriptstyle{M_1}$}
\put(65,54){$\scriptstyle{M_2}$}
\put(-6,28){$T$}
}
\end{picture}
	\caption{\em Giunto di Oldham commerciale che permette disassamenti tra gli alberi fino a $5\, {\rm mm}$.}
     \label{fig:oldham_commerciale}
\end{minipage}
\end{figure}
\noindent Nel riquadro in basso a destra
della figura \ref{fig:oldham} \`e riportato un meccanismo equivalente
al quadrilatero doppiamente degenere che stiamo analizzando e 
la trave $T$ 
di tale figura riproduce quindi il comportamento della
biella $BC$.
\noindent Si intravede cos\`i la possibilit\`a di trasmettere
omocineticamente il moto
tra due alberi paralleli e tra loro disassati. In figura \ref{fig:oldham_schema}
\`e riportato un {\em giunto di Oldham} che presenta la trave $T$ come
una squadra. Questa soluzione \`e suggerita dal fatto che appare come privo di
senso privilegiare particolari direzioni radiali in un dispositivo il cui
funzionamento spazia su giri completi. D'altra
parte anche la posizione delle due ``chiavi'' pu\`o essere ottimizzata in modo
da diminuire l'ingombro complessivo. La figura \ref{fig:oldham_commerciale}
mostra un piccolo giunto di {\em Oldham} commerciale, dove i glifi sono stati
ricavati sulla trave $T$ e non sulle due manovelle; la figura mostra 
i tre componenti disgiunti, le due manovelle e la trave di collegamento,
nonch\'e il giunto
assemblato che dovrebbe convincere il lettore
della sua semplicit\`a e compattezza.

\noindent Come promesso nel titolo del presente paragrafo, esponiamo
qualche ulteriore considerazione circa i manovellismi e la loro classificazione
in ordinari e non ordinari. Nel percorso che abbiamo seguito in questo lavoro
viene evidenziata la differente genesi delle due tipologie di manovellismo:
quando a divenire un punto improprio \`e una delle due cerniere fisse
si ottiene un manovellismo ordinario, viceversa
si ottiene un manovellismo non ordinario quando a diventare impropria \`e
una delle altre due cerniere.  Mentre nella sfera delle applicazioni
pratiche la scelta del ``membro telaio'' \`e tutt'altro che indifferente (a
ben pochi \`e venuto in mente di costruire un motore dove a ruotare fosse
il blocco dei cilindri),
nell'ambito della cinematica teorica possiamo individuare soltanto
i moti relativi tra
le varie parti di un quadrilatero o di un manovellismo. In altre parole,
considerando di nuovo il manovellismo ordinario di figura \ref{fig:man_ord}
e tirando in ballo la nostra ormai affezionata formica esploratrice
(ve la ricordate? Quella della giostra dei cavalli...), potremmo
affermare che tale insetto vedrebbe un manovellismo ordinario quando
sta fermo sul glifo (il cilindro), un manovellismo non ordinario 
come quello di figura \ref{fig:c_infinito_equivalenti}- $3)$ quando
sta fermo sulla manovella $AB$ e un meccanismo equivalente a quello di
figura  \ref{fig:c_infinito_equivalenti}- $1)$ quando la formica sta ferma 
sulla biella $BC$. 
Questo punto di vista circa i diversi manovellismi,
che riteniamo chiarificatore e
didattico, viene esposto in molti libri che trattano questo argomento come,
ad esempio, \cite{sesini1}, pag. 107.
%\newpage
%\thispagestyle{empty}
%\null
\endinput 

\chapter{Leggi di Moto e Profilatura delle Camme}\index{eccentrici}\index{camme}
\label{camme}

\section{Introduzione}

\noindent Nell'ambito delle macchine automatiche con elevata cadenza di funzionamento,
dedicate alla produzione su vasta scala di manufatti,
si incontrano sovente organi meccanici che eseguono movimenti ripetitivi,
composti da ben precise successioni di salite, soste e discese, nomi che tendono a essere
conservati anche quando i movimenti non sono verticali. La legge temporale di tali
movimenti \`e  solitamente
dettata dal particolare processo tecnologico che la macchina o una parte di essa
\`e chiamata a svolgere.
\begin{figure}[hbt]
\centering
\begin{minipage}[b]{0.600\textwidth}
\centering
\includegraphics[width=0.9\textwidth]{part2/camme/FIG/sagoma_traslazione.pdf}
\begin{picture}(0,0)(130,5)
\scriptsize{
\put(-22,235){$y(t)$}
\put(-25,94){$v$}
\put(-15,70){$T$}
}
\end{picture}
      \caption{\em Movimentazione di un cedente mediante sagoma di traslazione.}
 \label{fig:f_sagoma_traslazione}
\end{minipage}\hfill
\begin{minipage}[b]{0.300\textwidth}
\centering
\includegraphics[width=0.9\textwidth]{part2/camme/FIG/sagoma_avvolta.pdf}
\begin{picture}(0,0)(129,0)
\scriptsize{
\put(72,220){$y(t)$}
\put(50,51) {\color{white}$\omega$}
}
\end{picture}
	\caption{\em Sagoma di traslazione avvolta su un tamburo cilindrico.}
     \label{fig:f_sagoma_avvolta}
\end{minipage}
\end{figure}
\noindent Sono frequentissimi i casi di presse e fustelle montate in cosiddette ``stazioni''
a bordo di una macchina che eseguendo particolari {\em leggi di
moto}\index{leggi di moto} tagliano, sagomano, incollano
i vari pezzi di materiale cosicch\'e, alla fine di un ``ciclo
macchina'', otterremo il manufatto finito. 
Giusto per intenderci, \`e tipico incontrare in questo perimetro
macchine automatiche  che 
producono quantit\`a elevate di stoviglie in carta o alluminio, di semplici
calzature, di tessuti, di contenitori per la cosmetica, di bottiglie
realizzate in svariate materie plastiche vuote o riempite contestualmente
alla loro formatura,
e chi pi\`u ne ha pi\`u ne metta. La cadenza di tali macchine pu\`o raggiungere e superare 
quella di due ``colpi'' al secondo.
La figura \ref{fig:f_sagoma_traslazione} riporta una idea {\em na\"ive} di un cedente 
che, tastando con una punta (o coltello) una sagoma che trasla,  viene azionato verticalmente, quindi
costretto a compiere la legge costituita dal profilo della sagoma stessa. La {\em camma lineare}\index{camme!lineari}
cos\`i ottenuta viene normalmente ``avvolta'', come in figura
\ref{fig:f_sagoma_avvolta}, su un tamburo il quale, ruotando generalmente a velocit\`a
costante, pu\`o riprodurre automaticamente ad ogni giro
la legge di moto senza la necessit\`a di ripeterla molte volte in sequenza,
come sarebbe necessario fare sulla sagoma traslante.  
\begin{wrapfigure}{r}{0.5\textwidth}
     \begin{center}
     \includegraphics[width=0.42\textwidth]{part2/camme/FIG/generic_law/legge_completa.pdf}
     \end{center}
\begin{picture}(0,0)(0,0)
	\scriptsize{
	\put(155,111){$\alpha$}
	\put(6,118){$y(\alpha)$}
	\put(7,105){$y(t)$}
	\put(155,25){$t$}
	\put(130,18){$T$}
}
\end{picture}
        \caption{\em Diagramma delle alzate.}
     \label{fig:f_diagramma_alzate}
\end{wrapfigure}
\noindent Nella stragrande maggioranza delle macchine automatiche la generazione di movimenti opportuni rispettando tempi
precisi \`e di fondamentale importanza. Questo perch\'e i movimenti di fustelle,
presselli, spruzzatori, togli-metti foglio, e altri dispositivi, frequenti in 
questo ambito, necessitano di tempi e luoghi precisi per  soddisfare il processo
tecnologico al quale sono adibiti: processo che pu\`o essere rappresentato da una operazione di imbutitura, di
formatura, di incollaggio ecc. Unitamente a ci\`o, i movimenti di questi
dispositivi devono spesso 
rispettare
sincronizzazioni temporali tra le operazioni stesse e
movimenti di altri organi della medesima macchina.
Questi movimenti ripetitivi possono essere rappresentati in un grafico che prende
il nome di {\em diagramma temporale delle alzate}\index{diagramma delle alzate}, del quale un esempio \`e riportato in
figura \ref{fig:f_diagramma_alzate}.
Abbiamo poc'anzi accennato al fatto che normalmente 
la generazione delle alzate avviene tramite la trasformazione del moto
rotatorio di un albero, il quale ruota
a velocit\`a angolare  $\omega$ costante: ad ogni rotazione
completa dell'{\em albero a camme}\index{albero a camme},
sul quale sar\`a presente la {\em camma}\index{camme}, si completer\`a un intero ciclo del diagramma delle alzate.
Entreremo tra poco nel cuore di una 
descrizione, certamente non esaustiva, dei dispositivi a camma che operano la suddetta trasformazione;
anteponiamo qui solo qualche considerazione generale sulle leggi di movimento.
Tali considerazioni, peraltro, hanno applicazioni ben pi\`u estese rispetto al solo ambito delle camme meccaniche.
Riteniamo tuttavia che i meccanismi a camma offrano un generoso campo di
esempi che aiutano a comprendere l'argomento.

\section{Le Leggi di Moto}

\noindent Fissiamo le idee sulla movimentazione della mazza di una pressa
la quale dovr\`a eseguire gli spostamenti ciclici indicati
in figura \ref{fig:f_diagramma_alzate}
con una alzata massima del cedente pari
a $40\,{\rm mm}$.
Se si conviene di indicare con $T$ il {\em periodo} di tempo entro il 
quale si completa un {\em ciclo macchina}\index{ciclo macchina}, pari
nel nostro caso a $0.8\,{\rm s}$,
la velocit\`a angolare costante dell'albero che genera tale movimento sar\`a
$\omega = {2\pi \over{T}}=7.85\,{\rm rad}/{\rm s}$ che corrisponde a 75
giri al minuto.
Dato il legame tra l'angolo percorso dall'albero a camme e il tempo trascorso,
${\rm d}\alpha= \omega {\rm d}t$,  possiamo sostituire alla variabile temporale
della funzione $y(t)$ la variabile angolare $\alpha$, considerando che
in un ciclo macchina sar\`a $0 \leq \alpha \leq 2\pi$.  
Si chiama {\em diagramma delle alzate}\index{diagramma delle alzate}
il grafico di $y(\alpha)$ che mostra il legame geometrico tra le alzate di una
camma e la sua rotazione.
Si noti che la sovrapposizione del diagramma temporale delle alzate con il diagramma in funzione dell'angolo $\alpha$ \`e possibile
soltanto quando $\omega$ \`e costante, condizione questa, come abbiamo gi\`a affermato,
generalmente soddisfatta. La velocit\`a e l'accelerazione del cedente hanno
quindi un loro corrispondente di natura geometrica. Infatti derivando la 
funzione $y(t)$ rispetto al tempo otteniamo per la velocit\`a
\begin{equation}
{{\rm d}y(t)\over{{\rm d}t}}= \dot y= \omega {{\rm d}y(\alpha)\over{{\rm d}\alpha}}= \omega y'\,,
\end{equation}
\noindent e per l'accelerazione
\begin{equation}
{{\rm d}^2y(t)\over{{\rm d}t^2}}= \ddot y= \dot\omega 
{{\rm d}y(\alpha)\over{{\rm d}\alpha}}+
\omega^2 {{\rm d}^2y(\alpha)\over{{\rm d}\alpha^2}}=\dot\omega y' + \omega^2 y''\,,
\end{equation}
\noindent che, nel caso di velocit\`a angolare costante, si riduce a
\begin{equation}
\ddot y= \omega^2 y''\,.
\end{equation}
\noindent Essendo le {\em quantit\`a cinematiche geometriche}\index{velocit\`a!geometrica}\index{accelerazione!geometrica}, legate esclusivamente alla forma
della camma (pi\`u in generale legate al particolare meccanismo che le produce
utilizzando un moto rotatorio), e indipendenti dal regime di rotazione dell'albero,
esse risultano
estremamente comode per confrontare tra loro
{\em leggi di moto} diverse,
nel processo di sintesi progettuale della camma stessa.
In moltissimi casi, il diagramma delle alzate presenta una certa libert\`a interpretativa.
Spesso, ci\`o che sostanzialmente importa nella movimentazione dell'utensile
sono i valori delle successive alzate e le loro durate temporali, nonch\'e le
durate delle varie soste:
la legge di moto $y(t)$ che collega 
due posizioni di alzata invece non \`e quasi mai fissata in maniera stringente.
Nel caso del diagramma delle alzate di figura \ref{fig:f_diagramma_alzate}, ad esempio,
contiamo tre leggi di moto distinte sugli intervalli di tempo:
$0-0.1$, $0.2-0.3$ e $0.5-0.7$, frammiste a tre 
soste: $0.1-0.2$, $0.3-0.5$ e $0.7-0.8$. 
La mancanza di vincoli rigidi per i tratti di salita e discesa, cui abbiamo test\'e accennato, ci consente  la possibilit\`a di scegliere, a parit\`a
 sostanziale del
diagramma delle alzate, leggi di moto diverse che soddisfacendo tale diagramma 
possono sottostare anche ad alcuni altri criteri che sceglieremo e imporremo in
modo opportuno. Il pi\`u costrittivo di tali criteri \`e quello che permette
al cedente di giungere nei tratti di sosta effettivamente fermo. Indicando 
perci\`o con $t_s$ il tempo in cui si svolge la legge di salita deve 
essere
\begin{equation}
\int_0^{t_s} \ddot y(t) {\rm d}t=0\,.
\label{e_vel_zero}
\end{equation}
\noindent Oppure, considerando la variabile angolare anzich\'e il tempo e
indicando con $\alpha_s$ l'angolo di rotazione della camma
sul quale si estende quel tratto di salita, dovr\`a essere
\begin{equation}
\int_0^{\alpha_s}  y''(\alpha) {\rm d}\alpha=0\,.
\label{e_vel_zero_alpha}
\end{equation}
\noindent Una volta scelta la ``forma'' della
legge di accelerazione che soddisfa le relazioni \ref{e_vel_zero} e
\ref{e_vel_zero_alpha}, dovranno essere verificate le seguenti relazioni
\begin{equation}
\int_0^{t_s} {\rm d}t\int_0^{t_s} \ddot y(t) {\rm d}t\;\;=\;\;
\int_0^{\alpha_s} {\rm d}\alpha\int_0^{\alpha_s} y''(\alpha) {\rm d}\alpha=h\,,
\label{eq:ha}
\end{equation}
\noindent o anche
\begin{equation}
\int_0^{t_s} \dot y(t) {\rm d}t\;\;=\;\;
\int_0^{\alpha_s} y'(\alpha) {\rm d}\alpha=h\,,
\label{eq:hv}
\end{equation}
\noindent in maniera che tale legge produca effettivamente l'alzata richiesta.
Segnaliamo che,
nella parte dedicata agli approfondimenti, al capitolo \ref{cam2},
\`e presente 
una breve nota circa un'ulteriore propriet\`a notevole delle leggi di moto.
\vskip .3cm
\begin{figure}[hbt]
\centering
\begin{minipage}[b]{0.49\textwidth}
\centering
\includegraphics[width=0.9\textwidth]{part2/camme/FIG/generic_law/confronto_alzate_acc_diverse_acc.pdf}
\begin{picture}(0,0)(130,0)
\scriptsize{
\put(-20,96){${{}\ddot{y}_{3}}_{\rm max}=17.5$}
\put(-40,90){$y''$}
\put(-40,80){$\ddot y$}
\put(10,81){${{}\ddot{y}_{2}}_{\rm max}=6$}
\put(45,70){${{}\ddot{y}_{1}}_{\rm max}=4$}
\put(120,1){$\alpha$}
\put(114,1){$t,$}
}
\end{picture}
      \caption{\em Diverse leggi di accelerazione.}
 \label{fig:f_confronto_accelerazioni}
\end{minipage}\hfill
\begin{minipage}[b]{0.49\textwidth}
\centering
\includegraphics[width=0.9\textwidth]{part2/camme/FIG/generic_law/confronto_alzate_acc_diverse_spost.pdf}
\begin{picture}(0,0)(130,0)
	\scriptsize{
\put(-38.5,90){$y(\alpha)$}
\put(-38.5,80){$y(t)$}
\put(114,1){$t,$}
\put(120,1){$\alpha$}
}
\end{picture}
	\caption{\em Corrispondenti diagrammi dell'alzata.}
     \label{fig:f_confronto_alzate}
\end{minipage}
\end{figure}

\noindent In  figura \ref{fig:f_confronto_alzate} sono riportate tre diverse leggi
di spostamento che producono la stessa alzata, $h=1$, durante una rotazione
dell'albero di un radiante, oppure dopo il tempo di un secondo (si sottintende
perci\`o $\omega=1$), terminando la corsa con velocit\`a nulla.
Mentre la discrepanza
tra i diagrammi dello spostamento, $y(t)$ ovvero $y(\alpha)$,
risulta essere modesta, 
si pu\`o osservare che i tre diagrammi di accelerazione,
riportati in figura \ref{fig:f_confronto_accelerazioni}
unitamente ai loro valori massimi, sono molto diversi tra loro.
Ci si pu\`o chiedere: \`e sensato scegliere la
legge di moto 1) che, a parit\`a di alzata, presenta la minima accelerazione
massima? La risposta \`e s\`i e no. Infatti, il processo di sintesi
di una legge di moto per camme deve tenere conto di molte
circostanze, e la scelta di una legge che privilegi le basse accelerazioni pu\`o
presentarsi mediocre sotto altri aspetti. Cercheremo nel seguito
 di dare un 
quadro, peraltro parziale, del complesso di vincoli che guidano
la progettazione di una legge di moto e di una camma.
Riportiamo in figura \ref{fig:f_confronto_velocita}
il confronto tra le velocit\`a associate alle precedenti tre leggi di
moto in cui si nota che la velocit\`a pi\`u elevata viene raggiunta 
dalla legge con accelerazione massima minore
e viceversa.
\begin{figure}[hbt]
\centering
\begin{minipage}[b]{0.49\textwidth}
\centering
\includegraphics[width=0.9\textwidth]{part2/camme/FIG/generic_law/confronto_alzate_acc_diverse_acc.pdf}
\begin{picture}(0,0)(130,0)
\scriptsize{
\put(-20,96){${{}\ddot{y}_{3}}_{\rm max}=17.5$}
\put(-40,90){$y''$}
\put(-40,80){$\ddot y$}
\put(10,81){${{}\ddot{y}_{2}}_{\rm max}=6$}
\put(45,70){${{}\ddot{y}_{1}}_{\rm max}=4$}
\put(120,1){$\alpha$}
\put(114,1){$t,$}
}
\end{picture}
      \caption{\em Diverse leggi di accelerazione.}
 \label{fig:f_confronto_accelerazioni2}
\end{minipage}\hfill
\begin{minipage}[b]{0.49\textwidth}
\centering
\includegraphics[width=0.9\textwidth]{part2/camme/FIG/generic_law/confronto_alzate_acc_diverse_vel.pdf}
\begin{picture}(0,0)(130,0)
\scriptsize{
\put(-48,90){$y'(\alpha)$}
\put(-48,80){$\dot y(t)$}
\put(114,1){$t,$}
\put(120,1){$\alpha$}
\put(22,80){${{}\dot{y}_{1}}_{\rm max}=2$}
\put(62,63){${{}\dot{y}_{2}}_{\rm max}=1.42$}
\put(17,39){${{}\dot{y}_{3}}_{\rm max}=1.11$}
}
\end{picture}
	\caption{\em Corrispondenti diagrammi della velocit\`a.}
     \label{fig:f_confronto_velocita}
\end{minipage}
\end{figure}

\noindent L'importanza di evitare picchi troppo elevati di accelerazione
deriva dal diretto collegamento tra questa grandezza
e l'entit\`a delle forze d'inerzia che insorgono durante il funzionamento della macchina.
Ed \`e questo il motivo che induce il progettista a disegnare la camma partendo
proprio dalle leggi di accelerazione, tenendo cos\`i sotto controllo
tale grandezza.
La legge con {\em accelerazione costante} riportata in
figura \ref{fig:f_legge_acc_cost} \`e la legge di moto pi\`u ``semplice''
che si possa adottare.
In tale figura, come nelle precedenti 
 \ref{fig:f_confronto_accelerazioni}--\ref{fig:f_confronto_velocita},
e nelle seguenti, riguardanti le leggi di moto, le ascisse riportano
sia il tempo $t$, sia l'angolo $\alpha$. Pertanto, queste leggi
possono essere qui interpretate come eseguite in un tempo unitario oppure
in un angolo unitario, ponendo cos\`i $\omega=1\; {\rm rad}/{\rm s}$. Le stesse leggi producono tutte una alzata unitaria.
\begin{figure}[hbt]
\centering
\begin{minipage}[b]{0.49\textwidth}
\centering
\includegraphics[width=0.9\textwidth]{part2/camme/FIG/generic_law/legge_acc_cost.pdf}
\begin{picture}(0,0)(150,-10)
\scriptsize{
\put(-17,75){$\ddot y$}
\put(-17,65){$\dot y$}
\put(-17,55){$y$}
\put(15,65){${{}\ddot{y}}_{\rm max}=4$}
\put(64,59){${{}\dot{y}}_{\rm max}=2$}
\put(140,-9){$\alpha$}
\put(134,-9){$t,$}
}
\end{picture}
\caption{\em Legge di moto con accelerazione costante.}
 \label{fig:f_legge_acc_cost}
\end{minipage}\hfill
\begin{minipage}[b]{0.49\textwidth}
\centering
\includegraphics[width=0.9\textwidth]{part2/camme/FIG/generic_law/legge_acc_cost_tagliata.pdf}
\begin{picture}(0,0)(150,-10)
\scriptsize{
}
\put(-17,75){$\ddot y$}
\put(-17,65){$\dot y$}
\put(-17,55){$y$}
\put(39,70){${{}\ddot{y}}_{\rm max}=4.76$}
\put(48,55){${{}\dot{y}}_{\rm max}=1.43$}
\put(140,-9){$\alpha$}
\put(134,-9){$t,$}
\end{picture}
	\caption{\em Legge di moto con accelerazione costante tagliata.}
     \label{fig:f_acc_cost_tagliata}
\end{minipage}
\end{figure}
La scelta di rappresentare leggi di moto che producono
alzata unitaria in tempi o angoli unitari, leggi che chiameremo {\em unitarie}\index{leggi di moto!unitarie},
agevola il confronto
tra accelerazioni ``di forme diverse'', in quanto tale confronto
pu\`o avvenire semplicemente tra
alcuni dei valori notevoli che le grandezze cinematiche, di tali leggi
unitarie, assumono. Per passare da questi valori, cio\`e quelli
relativi alle grandezze cinematiche delle leggi unitarie e che,
limitatamente a questo capoverso, indicheremo
con pedice $u$, alle grandezze cinematiche effettive (pedice $e$) di una legge
 di uguale forma,
la quale per\`o produce un'alzata pari a $h$ in un tempo $t_s$ o in un angolo
$\alpha_s$, con riferimento agli integrali
\ref{eq:ha} e \ref{eq:hv}, si ha
\begin{equation}
\ddot y_e = \ddot y_u \frac{h}{t_s^2}\,,\;\;\;\;
{y''_e} = {y''_u} \frac{h}{\alpha_s^2}\,,\;\;\;\;
\dot y_e = \dot y_u \frac{h}{t_s}\,,\;\;\;\;
{y'_e} = {y'_u} \frac{h}{\alpha_s}\,.
\end{equation}
\noindent Ad esempio, si pu\`o notare che per la legge
ad accelerazione costante che si compie in un tempo unitario e fornisce
una alzata $h$ unitaria, si ha accelerazione massima ${{}\ddot {y}}_{\rm max}=4$,
come riportato nella figura \ref{fig:f_legge_acc_cost}.
Chiamando questa grandezza, cio\`e la massima accelerazione raggiunta
da una legge unitaria, {\em coefficiente di accelerazione}\index{coefficiente!di accelerazione} $c_a$, avremo per l'effettiva accelerazione
fornita da una legge di moto qualsivoglia
\begin{equation}
{{}\ddot{y}}_{\rm max}= c_a {h\over t_s^2}\;\;\; {\rm e}\;\;\;\;  
{{y''}}_{\rm max}= c_a {h\over \alpha_s^2}\,.  
\label{e_ca}
\end{equation}
\noindent \`E facile osservare che per una legge di moto di forma qualsiasi,
che fornisca
al suo termine $\dot {y}=0$ e che sia simmetrica rispetto alla met\`a dell'angolo sul quale si svolge, risulta $c_a \geq 4$, essendo il valore quattro
riservato proprio al caso di accelerazione costante simmetrica.
Sempre richiamando la figura \ref{fig:f_legge_acc_cost} notiamo che la velocit\`a
massima raggiunta dal cedente \`e $\dot {y}_{\rm max}=2$. Questo valore
\`e il {\em coefficiente di velocit\`a}\index{coefficiente!di velocit\`a}
 $c_v$ di questa legge di moto. 
Analogamente a quanto esposto per le accelerazioni si ha
\begin{equation}
{{}\dot{y}}_{\rm max}= c_v {h\over t_s}\;\;\; {\rm e}\;\;\;\;  
{{y'}}_{\rm max}= c_v {h\over \alpha_s}\,.  
\label{e_cv}
\end{equation}
\noindent Il coefficiente di velocit\`a, che qualifica la legge di moto rispetto
alla velocit\`a massima raggiunta, \`e altrettanto importante nella sintesi
delle camme in quanto, come vedremo tra poco, influenza direttamente
una grandezza di primaria importanza nella trasmissione del movimento.
In figura \ref{fig:f_acc_cost_tagliata} sono riportati i grafici di accelerazione
velocit\`a e spostamento per una legge cosiddetta con {\em accelerazione
costante tagliata}\index{legge!accelerazione costante} dove si nota
che, rispetto alla legge ad accelerazione costante, il coefficiente di
accelerazione cresce, mentre diminuisce quello di velocit\`a.
\begin{figure}[hbt]
\centering
\begin{minipage}[b]{0.49\textwidth}
\centering
\includegraphics[width=0.9\textwidth]{part2/camme/FIG/generic_law/legge_basso_cv.pdf}
\begin{picture}(0,0)(150,-10)
\scriptsize{
}
\put(-17,75){$\ddot y$}
\put(-17,65){$\dot y$}
\put(-17,55){$y$}
\put(2,86){${{}\ddot{y}}_{\rm max}=44.9$}
\put(40,56){${{}\dot{y}}_{\rm max}=1.03$}
\put(140,-9){$\alpha$}
\put(134,-9){$t,$}
\end{picture}
      \caption{\em Legge di moto a basso coefficiente di velocit\`a.}
 \label{fig:f_legge_basso_cv}
\end{minipage}\hfill
\begin{minipage}[b]{0.49\textwidth}
\centering
\includegraphics[width=0.9\textwidth]{part2/camme/FIG/generic_law/legge_acc_sin_tagliata.pdf}
\begin{picture}(0,0)(150,-10)
\scriptsize{
\put(-17,75){$\ddot y$}
\put(-17,65){$\dot y$}
\put(-17,55){$y$}
\put(35,74){${{}\ddot{y}}_{\rm max}=6.54$}
\put(47,56){${{}\dot{y}}_{\rm max}=1.66$}
\put(140,-9){$\alpha$}
\put(134,-9){$t,$}
}
\end{picture}
	\caption{\em Legge di moto con accelerazione sinusoidale tagliata.}
     \label{fig:f_acc_sin_tagliata}
\end{minipage}
\end{figure}
\noindent In figura \ref{fig:f_legge_basso_cv} \`e riportata una legge che presenta 
un bassissimo coefficiente di velocit\`a $c_v = 1.03$ (il limite inferiore teorico \`e uno),  
a discapito dei valori dell'accelerazione che risultano essere elevati, con
coefficiente di accelerazione $c_a=44.9$.
La legge con {\em accelerazione sinusoidale tagliata}\index{legge!accelerazione
sinusoidale} di figura \ref{fig:f_acc_sin_tagliata} soddisfa le esigenze
del progettista che \`e preoccupato delle vibrazioni che possono insorgere
durante il funzionamento della camma e che sono influenzate da cambiamenti repentini
di accelerazione: \cite{ruggieri}, pag. 45.
\begin{wrapfigure}{r}{0.5\textwidth}
     \begin{center}
     \includegraphics[width=0.42\textwidth]{part2/camme/FIG/generic_law/legge_trapezoidale.pdf}
     \end{center}
\begin{picture}(0,0)(-29,-35)
\scriptsize{
}
\put(-17,75){$\ddot y$}
\put(-17,65){$\dot y$}
\put(-17,55){$y$}
\put(-5,81){${{}\ddot{y}}_{\rm max}=7.67$}
\put(40,52){${{}\dot{y}}_{\rm max}=1.53$}
\put(129,-8){$\alpha$}
\put(123,-8){$t,$}
\end{picture}
\vskip -6mm
        \caption{\em Legge trapezoidale tagliata (asimmetrica).}
     \label{fig:f_legge_trapezoidale}
\end{wrapfigure}
\noindent La derivata terza dello spostamento,
grandezza che peraltro compare assai raramente nella meccanica, \`e in certo
modo responsabile delle vibrazioni che insorgono nella catena cinematica
del cedente durante il funzionamento delle camme veloci. La legge sinusoidale
citata poc'anzi presenta bassi valori di {\em jerk}\index{jerk} (cos\`i viene
talvolta chiamata $\dddot{y}(t)$). Purtroppo tale legge presenta, a parit\`a di
altre circostanze, valori di $c_v$ pi\`u elevati della legge ad
accelerazione costante.
\noindent In un'ottica di compromesso si potrebbe modificare la legge ad accelerazione costante rendendo le sue due ``gobbe'' di forma trapezoidale, come in figura 
\ref{fig:f_legge_trapezoidale},  passando
cos\`i da valori molto alti di {\em jerk} (teoricamente infiniti) a valori finiti
e controllabili.
Anche qui per\`o non \`e tutto rose e fiori: le rampe inclinate di questa legge di moto
si traducono in una partenza della salita poco decisa e, quel che \`e peggio,
in una difficolt\`a materiale di realizzazione in officina di queste rampe con
precisione tale da garantire l'esattezza della legge di moto
dato che, per un certo tratto, si discostano troppo poco dalla circonferenza.
\begin{figure}[t]
\centering
\begin{minipage}[b]{0.49\textwidth}
\centering
\includegraphics[width=0.9\textwidth]{part2/camme/FIG/generic_law/legge_7tratti.pdf}
\begin{picture}(0,0)(150,-10)
\scriptsize{
\put(-17,75){$\ddot y$}
\put(-17,65){$\dot y$}
\put(-17,55){$y$}
\put(43,72){${{}\ddot{y}}_{\rm max}=5.08$}
\put(48,57){${{}\dot{y}}_{\rm max}=1.66$}
\put(140,-9){$\alpha$}
\put(134,-9){$t,$}
}
\end{picture}
      \caption{\em Legge di moto con accelerazione trapezoidale modificata.}
 \label{fig:f_legge_7tratti}
\end{minipage}\hfill
\begin{minipage}[b]{0.49\textwidth}
\centering
\includegraphics[width=0.9\textwidth]{part2/camme/FIG/generic_law/legge_7tratti_non_simm.pdf}
\begin{picture}(0,0)(150,-10)
\scriptsize{
\put(-17,75){$\ddot y$}
\put(-17,65){$\dot y$}
\put(-17,55){$y$}
\put(42,72){${{}\ddot{y}}_{\rm max}=4.86$}
\put(48,60){${{}\dot{y}}_{\rm max}=1.59$}
\put(38,4){${{}\ddot{y}}_{\rm min}=-7.6$}
\put(140,-9){$\alpha$}
\put(134,-9){$t,$}
}
\end{picture}
      \caption{\em Legge di moto con accelerazione asimmetrica.}
     \label{fig:f_legge_7tratti_non_sim}
\end{minipage}
\vskip -3mm
\end{figure}
\noindent Una legge di moto raffinata e spesso adottata per la sintesi delle camme, \`e senza dubbio la cosiddetta {\em legge
di moto trapezoidale modificata}\index{legge!trapezoidale modificata}, figura \ref{fig:f_legge_7tratti}.
Si compone di sette tratti: sinusoide, costante, sinusoide, nulla, sinusoide,
costante, sinusoide. Ciascuno di questi intervalli pu\`o essere lungo a
piacere permettendo in tal modo di dare origine eventualmente a leggi non simmetriche, come quella di figura \ref{fig:f_legge_7tratti_non_sim}. 

\definecolor{dark-green}{rgb}{0,0.5,0}
\begin{wrapfigure}{r}{0.6\textwidth}
\begin{center}
\includegraphics[width=0.55\textwidth]{part2/camme/FIG/generic_law/legge_acc_cost_e_7tratti_confronto.pdf}
\end{center}
\begin{picture}(0,0)(0,0)
\put(5,135){\color{red}$\ddot y$}
\put(5,120){\color{dark-green}$\dot y$}
\put(5,105){\color{blue}$y$}
\put(190,26){$\alpha$}
\put(184,26){$t,$}
\end{picture}
\vskip -6.5mm
      \caption{\em Confronto tra legge ad accelerazione costante e legge a sette tratti con lo stesso tempo di taglio.}
 \label{fig:f_confronto_cost_7}
\end{wrapfigure}
\noindent In figura \ref{fig:f_confronto_cost_7} si riporta,
a parit\`a di alzata e di lunghezza del ``taglio'', il confronto tra
una legge ad accelerazione costante e una legge trapezoidale modificata.
Sebbene una tale analisi
sia gi\`a stata illustrata in figura \ref{fig:f_confronto_accelerazioni},
si desidera qui esplicitarla di nuovo tramite l'impiego di due leggi molto
pi\`u usate nella pratica, in quanto la presenza del taglio assicura $c_v$
meno elevati: caratteristica, questa, molto apprezzabile, come vedremo.
\noindent Rimarchiamo ancora una volta la sostanziale equivalenza delle due leggi a livello
del diagramma delle alzate, praticamente indistinguibili nella figura.
L'inconveniente presentato dalla legge trapezoidale modificata
di avere un $c_a$ leggermente
pi\`u elevato \`e ben assorbito dalla maggiore ``dolcezza di movimento'' che essa 
garantisce.
\noindent Infine riportiamo nelle figure
\ref{fig:f_legge_completa_7_a} e
\ref{fig:f_legge_completa_7_h} rispettivamente il diagramma delle accelerazioni completo per progettare la nostra camma e il corrispondente diagramma delle alzate.
Tale diagramma, gi\`a illustrato in figura
\ref{fig:f_diagramma_alzate}, qui ottenuto per\`o
partendo da leggi di accelerazione trapezoidali modificate tagliate \`e
il candidato idoneo a essere ``avvolto'' sul tamburo della camma.
\begin{figure}[hbt]
\centering
\begin{minipage}[b]{0.49\textwidth}
\centering
\includegraphics[width=0.9\textwidth]{part2/camme/FIG/generic_law/legge_completa_7tratti_a.pdf}
\hfill
\begin{picture}(0,0)(130,0)
\scriptsize{
\put(105,2){$\alpha$}
\put(89,97){$T$}
\put(105,89){$t$}
\put(-46,86){$y''$}
}
\end{picture}
      \caption{\em Legge completa con tratti di  accelerazione trapezoidali modificate.}
     \label{fig:f_legge_completa_7_a}
\end{minipage}
\hfill
\begin{minipage}[b]{0.49\textwidth}
\centering
\includegraphics[width=0.9\textwidth]{part2/camme/FIG/generic_law/legge_completa_7tratti_h.pdf}
\begin{picture}(0,0)(116,0)
\scriptsize{
\put(105,2){$\alpha$}
\put(89,97){$T$}
\put(105,89){$t$}
\put(-58,86){$y(\alpha)$}
}
\end{picture}
      \caption{\em Legge completa delle alzate ottenuta da tratti di  accelerazione trapezoidali modificate.}
     \label{fig:f_legge_completa_7_h}
\end{minipage}
\end{figure}
\noindent Un'ultima considerazione circa la progettazione delle leggi di moto
dovrebbe mettere in luce la delicatezza di questo processo e l'atteggiamento
di compromesso necessario. Diamo uno sguardo alla seconda legge di accelerazione
presente nella figura \ref{fig:f_legge_completa_7_a}.
Tale legge produce in un angolo di $45^{\circ}$ un'alzata di $30\, {\rm mm}$ e,
delle tre, \`e quella con le caratteristiche pi\`u esasperate. Essa si presenta
pi\`u
``squadrata'' rispetto alle altre in modo da contenere i valori di $c_{a^+}$
(coefficiente di accelerazione considerando $\ddot y_{\rm max}$)
e di $c_{a^-}$
(coefficiente di accelerazione considerando $\ddot y_{\rm min}$). Tali valori sarebbero ancora inferiori se si eliminasse
il taglio ma questa scelta peggiorerebbe il valore dell'angolo di pressione 
$\theta$ aumentandolo e compromettendo, come vedremo a breve, il buon
funzionamento del meccanismo. 

\noindent Da un diverso punto di vista,
 contenere il valore di $c_{a^-}$ significa aumentare il valore del
minimo raggio di curvatura convessa $\rho_{\rm min}$.
Questo risulta essere un beneficio, come cercheremo di spiegare pi\`u avanti;
tuttavia,
eccedere in tal senso (aumentando il tempo su cui si estende la ``gobba''
negativa dell'accelerazione) produrrebbe il pessimo risultato di
aumentare $c_{a^+}$, parametro che governa
le forze d'inerzia durante l'alzata.

\vskip 1mm
\null
\section{Profilatura delle Camme}\index{camme}

\noindent Come accennato all'inizio di questo capitolo, la legge di spostamento
$y(\alpha)$ 
viene (salvo casi particolari) avvolta su di un tamburo in modo da generare
una camma piana.
Nel processo di sintesi della camma le grandezze  in gioco sono molte e tra
queste il raggio del tamburo test\'e citato, che si chiama {\em raggio di base}\index{raggio
di base} e che verr\`a indicato con $R_b$, gioca il  ruolo pi\`u importante.
All'aumentare del raggio di base scompaiono infatti dalla
camma le problematiche principali di profilatura,
cui accenneremo fra poco.  Di contro,  col crescere di $R_b$,
aumentano le dimensioni della camma stessa rendendola 
scomoda da utilizzare nel progetto organico di una macchina\footnote{Ci preme
precisare 
che il raggio di base pu\`o essere riferito a entit\`a geometriche diverse
dal profilo fisico della camma. In particolare, vedremo che le camme tastate
da una rotella presentano un importantissimo luogo geometrico ``virtuale'', il
{\em profilo del
centro rotella}, che coincide con il profilo di una camma tastata da una punta che esegue le stesse leggi di moto.
In questo caso $R_b$ sar\`a il raggio della circonferenza minima
tangente al tracciato del centro rotella, in quanto difficilmente
risulter\`a utile tirare in ballo il raggio di base del profilo fisico.
Pertanto, nonostante
molteplici trattazioni pi\`u autorevoli della nostra introducano per
il raggio di base del profilo del centro rotella  il simbolo $R_{b0}$, siamo convinti
di non creare equivoci accontentandoci di usare soltanto $R_b$.
}.
Si intravede quindi un percorso di progetto 
che ha come obiettivo quello di trovare una soluzione
che, evitando eventuali criticit\`a 
derivanti da dimensioni geometriche mal scelte (quasi sempre
valore di $R_b$ insufficiente),
soddisfino anche requisiti di compattezza, limitazione
dell'ingombro
 e funzionalit\`a.
La figura \ref{fig:f_ang_press} ci aiuta a riconoscere una delle grandezze
geometriche 
fondamentali di un meccanismo a camma: l'{\em angolo di pressione} $\theta(
\alpha)$, funzione i cui valori, in generale,
variano nell'arco dell'angolo giro.
Notiamo subito che $\theta$ \`e anche l'angolo compreso tra la retta tangente al
profilo nel punto di contatto col cedente e una retta generica
ortogonale al cedente stesso. 
\begin{figure}[bht]
\centering
\includegraphics[width=0.8\textwidth]{part2/camme/FIG/ang_press.pdf}
\begin{picture}(70,0)(130,0)
\scriptsize{
\put(222,202){$\theta$}
\put(208,183.5){${\rm d}y$}
\put(168,167){${\rm d}\alpha (R_b+y)$}
\put(168,111){$\theta$}
\put(55,99){\rotatebox{90}{{$R_b + y$}}}
\put(150,82){$\omega$}
\put(100,33){$R_b$}
}
\end{picture}
      \caption{\em Angolo di pressione $\theta$.}
 \label{fig:f_ang_press}
\end{figure}
La forza $F$  che la camma deve trasmettere al cedente,
soggetto a una spinta assiale $S$, risulta
essere $F=S/\cos(\theta)$, relazione dalla quale si evince l'importanza
di mantenere l'angolo $\theta$ entro stretti limiti.
Angoli $\theta > 30^{\circ}-35^{\circ}$ non sono
in genere accettabili nei tratti di salita, dove cio\`e la camma spinge la
punteria, riservando il valore limite pi\`u elevato a soluzioni
che prevedono guide di scorrimento per la punteria con attriti particolarmente
ridotti, \cite{molian}, pagg. 131-132. \`E frequente, nell'ambito delle macchine
automatiche, l'utilizzo di punterie costituite da colonne 
a sezione circolare, rettificate, che scorrono in manicotti a ricircolo di
sfere con attrito molto limitato.
Pertanto poniamo il massimo angolo di
pressione tollerato al valore $\theta_{\rm lim}=35^{\circ}$, osservando che angoli
di pressione maggiori genererebbero una forza ortogonale al cedente troppo
elevata col pericolo di impuntamento di quest'ultimo, oppure di una trasmissione
del movimento poco efficiente.
Poniamoci dal punto di vista relativo a un osservatore solidale con la camma.
In tale riferimento la punta percorrer\`a, durante una rotazione infinitesima
${\rm d}\alpha$ dell'albero, un tratto infinitesimo della tangente nel punto di contatto.
Questo tratto infinitesimo si scompone facilmente sui cateti 
del triangolo rettangolo
${\rm d}y$ e ${\rm d}\alpha (R_b +y)$, dalla quale scomposizione risulta
\begin{equation}
\tan(\theta)={{\rm d}y\over{(R_b+y)}{\rm d}\alpha}= {y'\over{R_b+y}}\,.
\label{e_ang_press}
\end{equation}
\noindent Affinch\'e il meccanismo funzioni correttamente il valore massimo di questa espressione dovr\`a essere contenuto al di
sotto del valore prescritto. Purtroppo \`e alquanto probabile che il massimo valore di $y'(\alpha)$ e il minimo di $y(\alpha)$ non si verifichino per lo
stesso angolo $\alpha$, quindi 
occorrerebbe cercare il massimo della funzione \ref{e_ang_press}. 
\noindent Ricordiamo che $y'_{\rm max}=c_v h/\alpha_s$, cio\`e il massimo
valore della velocit\`a geometrica \`e dato dal coefficiente di velocit\`a
della legge di moto moltiplicato per l'alzata e diviso per l'angolo di salita.
In via cautelativa si pu\`o asserire che affinch\'e la trasmissione del moto
dalla camma al cedente non sia critica dovr\`a essere 
\begin{equation}
\tan(\theta_{\rm lim})> {c_v h\over{\alpha_s(R_b+y)}}\,.
\label{e_ang_press_cv}
\end{equation}
\noindent Cautelandoci ulteriormente si pu\`o porre  $y(\alpha)=0$ e scrivere
\begin{equation}
\tan(\theta_{\rm lim})> {c_v h\over{\alpha_s R_b}}\,,
\label{e_ang_press_max}
\end{equation}
\noindent ottenendo in questo modo un'indicazione per la scelta del raggio di base:
\begin{equation}
R_b > {c_v h\over{\alpha_s \tan(\theta_{\rm lim})}}\,.
\label{e_R_b}
\end{equation}
\noindent Dalla \ref{e_R_b} emerge appieno l'importanza di utilizzare
leggi di moto a basso valore di coefficiente di velocit\`a: da tale
grandezza dipende infatti linearmente la dimensione e quindi l'ingombro della camma.
Si pu\`o anche osservare come alzate $h$ importanti
basate su angoli di salita $\alpha_s$ modesti comportino
elevati valori del raggio di base. A questo proposito chiariamo
che nell'ambito delle  macchine automatiche \`e tutt'altro che infrequente
lavorare con salite che si svolgono in $20^{\circ}$ o $30^{\circ}$. 
\noindent La figura \ref{fig:f_camma_punteria} riporta una camma con {\em cedente a
punteria}\index{cedente!a punteria}. La punta (o coltello visto che le camme hanno
spessore diverso da zero) tasta, in questo caso, direttamente il profilo della camma e segue
la legge di figura \ref{fig:f_legge_completa_7_h}.
Diciamo per inciso che, tanto la camma di figura \ref{fig:f_camma_punteria} quanto
tutte le altre che seguiranno nelle varie illustrazioni, sono state
ottenute da questa medesima legge di moto che trae origine dal progetto di una 
pressa reale, tuttora funzionante in una fabbrica cartotecnica.
\noindent Molto raramente il profilo della camma viene per\`o tastato 
da una punta come nelle figure \ref{fig:f_ang_press} e \ref{fig:f_camma_punteria}.
Per evidenti questioni tribologiche si preferisce affidare lo scorrimento
del cedente al rotolamento di una rotella sopra un secondo profilo
tracciato in modo che 
il centro di tale rotella segua 
il profilo generato dalla legge delle alzate.
Normalmente, le rotelle sono  cuscinetti a sfere o a rulli appositamente
costruiti per questo scopo.
\vskip 2mm
\begin{figure}[hbt]
\centering
\begin{minipage}[b]{0.49\textwidth}
\centering
{\includegraphics[width=0.9\textwidth]{part2/camme/FIG/camma/camma_punteria.pdf}}
\hfill
%\hspace{\fill}
\begin{picture}(0,0)(150,0)
\scriptsize{
\put(0,150){${\rho_0}_{\rm min}=46\; {\rm mm,\;}\;\alpha=126^{\circ}$}
\put(9,144){\rotatebox{-54}{$\longrightarrow$}}
\put(39,145){\rotatebox{-77}{$\longrightarrow$}}
\put(43,140){$\theta_{\rm max}=35^{\circ},\; \alpha=103^{\circ}$}
}
\end{picture}
      \caption{\em Camma tastata da una punta, $R_b=80\;{\rm mm}$.}
 \label{fig:f_camma_punteria}
\end{minipage}
\hfill
\begin{minipage}[b]{0.46\textwidth}
\centering
\includegraphics[width=0.9\textwidth]{part2/camme/FIG/camma/inviluppo.pdf}
\begin{picture}(0,0)(148,-2)
\scriptsize{
\put(5,135){${\rho}_{\rm min}=26\; {\rm mm,\;}\;\alpha=126^{\circ}$}
\put(49,99){\rotatebox{-234}{$\longrightarrow$}}
\put(41.5,85.2){\rotatebox{-112}{\color{yellow}$\scriptscriptstyle{\rm circ}$}}
\put(40.9,74.9){\rotatebox{-94.5}{\color{yellow}$\scriptscriptstyle{\rm onf}$}}
\put(40.7,65){\rotatebox{-81}{$\color{yellow}\scriptscriptstyle{\rm ere}$}}
\put(42,56.5){\rotatebox{-70}{\color{yellow}$\scriptscriptstyle{\rm nza}$}}
\put(46,46){\rotatebox{-55}{\color{yellow}$\scriptscriptstyle{\rm di}$}}
\put(50.5,39.7){\rotatebox{-39}{\color{yellow}$\scriptscriptstyle{\rm  base}$}}
}
\end{picture}
      \caption{\em Profilo della camma inviluppato dal moto relativo della rotella di raggio  $r_r=20\; {\rm mm}$.}
     \label{fig:f_inviluppo}
\end{minipage}
\end{figure}
\noindent In figura \ref{fig:f_inviluppo} viene riportata la camma con il suo profilo
effettivo ottenuto dall'inviluppo
della rotella il cui centro segue il tracciato blu sul quale \`e posizionata,
cio\`e il {\em profilo del centro rotella}\index{profilo centro rotella}.
\noindent Le figure \ref{fig:f_camma_punteria} e \ref{fig:f_inviluppo} evidenziano un'altra
grandezza di elevato
rilievo nel progetto di una camma: 
la {\em curvatura}\index{curvatura!del profilo} del suo profilo.
Nelle due figure appena citate sono infatti riportati i valori minimi
di $\rho_0(\alpha)$ e $\rho(\alpha)$, che sono rispettivamente il raggio di curvatura
del profilo seguito dal centro della rotella e il raggio di curvatura del profilo fisico
della camma, valori variabili ovviamente da punto a punto.
Come nel caso dell'angolo di pressione, anche il controllo del raggio di curvatura \`e
di importanza fondamentale durante la fase di progetto.
Il profilo inviluppato dalla rotella, cio\`e l'effettivo profilo
della camma, presenter\`a raggi di curvatura  che saranno dati da $\rho=\rho_0-r_r$,
dove $r_r$ rappresenta il raggio della rotella stessa. \`E chiaro quindi che  
il raggio di curvatura del profilo effettivo pu\`o diventare
nullo o persino negativo nel caso in cui il raggio della rotella sia maggiore 
del minimo raggio di curvatura delle zone convesse 
del profilo del centro rotella.
Perci\`o, valori di $\rho_0$ eccessivamente bassi, nei tratti convessi di
tale profilo, possono causare le situazioni di {\em sottotaglio}\index{sottotaglio}
rappresentate in figura \ref{fig:f_sottotaglio},
e contrassegnate con le lettere $A$ e $B$. In tali zone si verifica che l'inviluppo
generato dalla rotella porta a un profilo intrecciato a coda di rondine,
quindi impossibile da seguire con un normale cedente.
\begin{figure}[t]
\hbox{
\vspace{-2cm} \hspace{.6cm}
\includegraphics[width=0.3\textwidth]{part2/camme/FIG/camma/sottotaglio_dettaglio.pdf}
}
%\rule[-4cm]{5pt}{6cm}
\hspace{-6cm}
\includegraphics[width=1.4\textwidth]{part2/camme/FIG/camma/sottotaglio.pdf}

\begin{picture}(0,0)(-100,0)
\scriptsize{
\put(-31,245){A}
\put(64.5,175){A}
\put(137,154){B}
}
\end{picture}
      \caption{\em Camma con evidente sottotaglio, raggio rotella $r_r=58\;{\rm mm}$.}
     \label{fig:f_sottotaglio}
\end{figure}
\noindent Va per\`o detto che occorre tenere il valore di ${\rho}_{\rm min}$
ben lontano anche dallo zero, cio\`e dallo spigolo vivo che si verificherebbe
se il centro della rotella seguisse  una curva di raggio uguale a quello della rotella stessa.
La trasmissione delle forze, a volte ingenti, tra camma e punteria a rotella si basa
sul contatto tra una generatrice del ``cilindroide'' della camma e una del
cilindro della rotella stessa. Le pressioni di contatto che si generano in questi casi
si possono ricavare dalla {\em teoria di Hertz} per i contatti tra solidi elastici.
In particolare, nel nostro caso il contatto avviene tra due cilindri aventi raggio
$\rho$ per la camma e $r_r$ per la rotella, quindi possiamo affermare che la pressione di contatto
$p_c$ sar\`a \cite{timoshenko}, pagg. 418-419\footnote{Questo testo ha rappresentato per  
molto tempo un utile riferimento per tutti i problemi inerenti
il comportamento elastico dei solidi.},
\begin{equation}
p_c \propto \sqrt {{\rho + r_r}\over {\rho r_r}}\,,
\label{eq_hertz}
\end{equation}
\noindent che esplicita la proporzionalit\`a inversa tra tale pressione e la radice
quadrata del prodotto dei due raggi. Risulta quindi chiaro che il desiderio
di non far nascere pressioni troppo elevate, le quali potrebbero compromettere
la forma della
camma deformandola plasticamente, ci porta a desiderare un raggio minimo per il
profilo percorso dalla rotella ben lontano dallo zero
(lo spigolo). Raramente si progettano camme in cui si trovi nelle convessit\`a
del profilo un raggio minore alla met\`a di quello della rotella che lo tasta.
A questo proposito, mostriamo in figura \ref{fig:f_camma_rotella} una camma
correttamente profilata, la quale rappresenta con pi\`u
dettagli la stessa di figura \ref{fig:f_inviluppo}. Si nota che il valore dell'angolo
di pressione massimo non cambia rispetto alla situazione di cedente a punta,
\cite{molian}, pag. 61,
e il raggio minimo di curvatura del profilo risulta
accettabile.
Anche qui, come abbiamo fatto per l'angolo di pressione, ci si potrebbe incamminare alla ricerca
del legame che intercorre tra le leggi di moto, il raggio di base della camma, il raggio della rotella e  
il minimo raggio di curvatura (convessa) che si otterr\`a sul profilo.
Il volume \cite{ruggieri} riporta, a mio avviso, una delle pi\`u ampie e utili
trattazioni della sintesi di meccanismi a camma. Su questo libro si trova,
a pagina 59, una formula che lega il raggio del percorso del centro della rotella
con i valori del raggio di base, della salita $y(\alpha)$ e delle sue derivate, prima e seconda.
Si ha per il raggio del percorso del  ``centro rotella''\index{curvatura!formula}
\begin{equation}
\rho_0={{[{y'}^2+(R_b + y)^2]}^{3/2}\over{(R_b + y)^2-(R_b +y)y'' +2{y'}^2}}\,.
\label{r_cur}
\end{equation}
\noindent Pertanto elevati valori di accelerazione negativa generano raggi di curvatura modesti, come anche 
eccessivi valori per la velocit\`a $y'(\alpha)$, mentre, ancora una volta, a
un aumento del raggio di base $R_b$ corrisponde un aumento del raggio di curvatura
del profilo, ma come sempre anche un poco desiderabile aumento dell'ingombro. 
\begin{figure}[h]
\centering
\includegraphics[width=0.9\textwidth]{part2/camme/FIG/camma/camma_rotella.pdf}
\begin{picture}(0,0)(235,-72)
\scriptsize{
\put(0,150){${\rho}_{\rm min}=26\; {\rm mm,\;}\;\alpha=126^{\circ}$}
\put(24,133){\rotatebox{126}{$\longrightarrow$}}
\put(69,124){\rotatebox{103}{$\longrightarrow$}}
\put(58,112){$\theta_{\rm max}=35^{\circ}, \alpha=103^{\circ}$}
}
\end{picture}
      \caption{\em Camma correttamente profilata, $R_b=80\, {\rm mm}$, $r_r=20\, {\rm mm}$.}
 \label{fig:f_camma_rotella}
\end{figure}
\noindent Nonostante ricavare la \ref{r_cur} non presenti difficolt\`a insormontabili, abbiamo
preferito evitare la sua  dimostrazione che sarebbe risultata lunghetta anzich\'e no,
e avrebbe aggiunto poco alla comprensione del problema che stiamo trattando.
\noindent Oggigiorno la profilatura delle camme si esegue totalmente
con tecniche numeriche e l'individuazione dei punti cospicui, cio\`e quelli che presentano il valore pi\`u elevato dell'angolo
di pressione e il minore raggio di curvatura, risulta essere piuttosto semplice:
normalmente i programmi di calcolo li rendono facilmente rintracciabili
evidenziandoli.
Le figure \ref{fig:ap_camma_rotella} e \ref{fig:cur_camma_rotella} dovrebbero
dare
l'idea della variazione delle due grandezze fondamentali, $\theta$ e $\rho$, per
la camma di figura \ref{fig:f_camma_rotella}. Mentre il grafico di
$\theta$ \`e contenuto entro precisi limiti, quello di $\rho$ pu\`o assumere
valori assoluti molto grandi ($\rho$ \`e infinito nei punti di flesso
del profilo), pertanto una parte di tale grafico \`e escluso dalla figura.
\begin{figure}[b]
\centering
\begin{minipage}[b]{0.48\textwidth}
\centering
\includegraphics[width=0.9\textwidth]{part2/camme/FIG/camma/ck_ang_press_camma_punteria.pdf}
\begin{picture}(0,0)(130,0)
\scriptsize{
}
\put(127,2){$\alpha$}
\put(-31,86){$\theta$}
\end{picture}
      \caption{\em Andamento di $\theta$ per la camma di figura \ref{fig:f_camma_rotella}.}
 \label{fig:ap_camma_rotella}
\end{minipage}\hfill
\begin{minipage}[b]{0.48\textwidth}
\centering
\includegraphics[width=0.9\textwidth]{part2/camme/FIG/camma/ck_cur_camma_rotella.pdf}
\begin{picture}(0,0)(129,0)
\scriptsize{
\put(127,2){$\alpha$}
\put(-38,86){$|\rho|$}
}
\end{picture}
      \caption{\em Andamento di $|\rho|$ per la camma di figura \ref{fig:f_camma_rotella}.}
     \label{fig:cur_camma_rotella}
\end{minipage}
\end{figure}
\noindent Mediante opportune maggiorazioni e minorazioni dei termini della
\ref{r_cur} si pu\`o ottenere, anche in questo caso, 
una formula di progetto per $R_b$, come mostrato in \cite{ruggieri}, pag. 72, formula che non riportiamo.
Aggiungiamo, tuttavia, qualche considerazione
qualitativa circa l'individuazione del raggio di base, mettendolo maggiormente
in luce nella \ref{r_cur}, riscrivendola nel seguente modo
\begin{equation}
\rho_0={{{{[{{y'}^2\over{(R_b + y)^2}}+1]}^{3/2}(R_b+y)}\over{{1-{y''\over{(R_b + y)}}} +{2{y'}^2\over{(R_b + y)^2}}}}}\,.
\label{r_cur1}
\end{equation}
\noindent Notiamo che, a un aumento del raggio di base $R_b$, il valore assoluto del 
termine del denominatore che contiene $y''$
diminuisce, e con esso
cala la possibilit\`a che il denominatore (nel caso di valori negativi di
$y''$) diventi troppo grande e tale da compromettere il 
raggio di curvatura della camma.
Ma soprattutto osserviamo una squisita ``quasi proporzionalit\`a''
tra $\rho_0$ e $R_b$ essendo $\rho_0\approxprop (R_b +y)$, la quale proporzionalit\`a ci garantisce che all'aumentare del raggio di base spariscono
i problemi legati a raggi di curvatura troppo piccoli del profilo.
Perci\`o, aumentando il raggio di base ci togliamo
sia dall'impaccio di angoli di pressione troppo elevati
sia dal pericolo di curvature eccessive del profilo della camma.


\noindent Nonostante la presenza di una formula che pu\`o guidarci
durante il progetto verso un valore di $R_b$ rispettoso dei limiti da imporre
alla curvatura,
la sintesi delle camme, come si \`e detto, si esegue sempre pi\`u spesso 
per tentativi ripetuti, mediante codici dedicati
che rendono veloce ed efficace il confronto tra soluzioni diverse: a un angolo di
pressione massimo troppo grande o a un raggio di curvatura del profilo eccessivamente
ridotto segue un nuovo tentativo con $R_b$ aumentato, oppure anche con modifiche
(tollerabili) alle leggi di moto.

\noindent Anche se in questa sede abbiamo ritenuto 
di fornire soltanto un'infarinatura generale che
renda chiari i concetti e le difficolt\`a che stanno alla base della progettazione
delle camme, in un perimetro di studi pi\`u vasto si trova qualche
altro ``trucco'' che permette
di non aumentare, oltre il ragionevole, l'ingombro di questi eccentrici e allo stesso
tempo di non eccedere coi valori dell'angolo di pressione. 
Uno di tali {\em escamotage} \`e quello di costruire un dispositivo camma-cedente in cui
l'asse di quest'ultimo non passi per il centro della camma stessa.
Il disassamento del cedente \`e in qualche modo equivalente a inclinare il cedente di un dato angolo che 
si sottrarr\`a all'angolo di pressione durante la salita, sommandosi invece a
questo nei tratti di discesa. Una discreta percentuale delle camme
che si realizzano sono di questo tipo: {\em camme a punteria deviata}\index{punteria deviata} oppure, con terminologia anglosassone,
{\em camme offset}\index{camme!offset}.
Un breve cenno alla progettazione di questi dispositivi si trova nel
capitolo di approfondimento \ref{cam2}.

\begin{figure}[b]
\centering
\begin{minipage}[b]{0.68\textwidth}
\centering
\includegraphics[width=0.9\textwidth]{part2/camme/FIG/cambil.pdf}
\begin{picture}(0,0)(150,-20)
\scriptsize{
}
\put(35,143){$\beta(t)$}
\put(-29,60){$\omega$}
\end{picture}
        \caption{\em Camma con cedente a bilanciere.}
     \label{fig:f_cambil}
\end{minipage}\hfill
\begin{minipage}[b]{0.28\textwidth}
\centering
     \includegraphics[width=0.9\textwidth]{part2/camme/FIG/campiat.pdf}
\begin{picture}(0,0)(70,-20)
\scriptsize{
}
\put(7,214){$y(t)$}
\put(10,45){$\omega$}
\end{picture}
        \caption{\em Camma tastata da un piattello.}
     \label{fig:f_campiat}
\end{minipage}\hfill
\end{figure}

\noindent Le camme possono azionare cedenti che non traslano, come mostrato
in figura \ref{fig:f_cambil}.
Sono diffusissimi meccanismi in
cui la camma impone al
cedente a bilanciere un moto rotatorio.
Mutando opportunamente
ci\`o che nel ragionamento si deve mutare e in particolare sostituendo l'alzata $y(t)$ con l'angolo
che descrive il bilanciere nella sua rotazione, $\beta(t)$, tutto quanto esposto circa le precauzioni
nel progetto della camma rimane invariato: le leggi di moto con i loro
 coefficienti di velocit\`a e di accelerazione si scelgono con gli stessi criteri e
nella profilatura della camma si devono fare i soliti conti con curvatura e angolo di 
pressione. Anche qui, rimandiamo il lettore desideroso di approfondire un poco la
progettazione delle camme tastate da cedente a bilanciere al capitolo
\ref{cam2}, oppure a \cite{ruggieri}, pag. 74.


\noindent La trasmissione del movimento tra camma e cedente non avviene sempre tramite il rotolamento
di una rotella sul profilo della camma. Bench\'e tastare una camma con un {\em piattello}\index{piattello}, come in figura \ref{fig:f_campiat}, sia ormai raro nell'ambito delle macchine automatiche, in altri 
domini della meccanica la soluzione camma-piattello \`e molto usata.
Anzi, a dire il vero, vi \`e un vastissimo repertorio di camme tastate esclusivamente
da piattelli che, strisciando sul profilo, in questo caso
rigorosamente convesso di quest'ultime,
aziona cedenti a punteria o a bilanciere: le camme dei motori endotermici a quattro
tempi. Si tratta in genere di camme di dimensioni modeste alle quali calza naturalmente
tutta la teoria qui esposta. 

\noindent Una breve introduzione alla progettazione
delle camme tastate da piattelli si trova nel capitolo di
approfondimento \ref{cam2} e, per esteso, in \cite{ruggieri}, pag. 65.
 I problemi legati alla progettazione di queste camme
sono per\`o di natura cos\`i
diversa da quelli che si incontrano nelle macchine automatiche, la loro progettazione
segue dettami cos\`i intimamente legati alla legge delle alzate, a sua volta legata
al percorso dei gas nei condotti fluidodinamici che esse aprono e chiudono, che
ci sembra meglio lasciare questi piccoli e sofisticati eccentrici al loro
mondo, cio\`e quello motoristico, e tornare alle nostre camme per macchine
automatiche.

\begin{figure}[t]
\centering
\begin{minipage}[b]{0.48\textwidth}
\centering
\includegraphics[width=0.9\textwidth]{part2/camme/FIG/camma/camma_desmo.pdf}
\begin{picture}(0,0)(130,0)
\scriptsize{
}
\end{picture}
      \caption{\em Camma desmodromica.}
 \label{fig:f_camma_desmo}
\end{minipage}\hfill
\begin{minipage}[b]{0.48\textwidth}
\centering
\includegraphics[width=0.9\textwidth]{part2/camme/FIG/camma/desmo_costola.pdf}
\begin{picture}(0,0)(129,0)
\scriptsize{
}
\end{picture}
      \caption{\em Camma a costola.}
     \label{fig:f_desmo_costola}
\end{minipage}
\end{figure}

\section{Camme Coniugate}

\noindent L'accoppiamento, cio\`e il contatto tra camma e cedente, pu\`o essere
 garantito nelle fasi di discesa\footnote{
Ribadiamo che il termine discesa si riferisce al movimento contrario
a quello generato dalla camma mentre compie il suo lavoro funzionale.
Esso non si riferisce
quindi in alcun modo all'avvicinamento alla Terra.
} in due diversi modi:
imponendo che la punteria prema sulla camma durante tutto l'angolo giro, oppure
facendo in modo, mediante una camma accoppiata a quella primaria, che la punteria
sia costretta a seguire il profilo di questa anche nei tratti che presentano $\dot y$ 
negative.
Nel primo caso si parla di {\em accoppiamento di forza}\index{accoppiamento!di forza}
e molto spesso tale forza viene esercitata da una molla (quasi sempre
pre-caricata) che, deformata
durante le fasi di salita, garantisce mediante la sua elasticit\`a il contatto
anche durante la discesa. Questa soluzione \`e raramente impiegata nell'ambito
delle macchine automatiche, l'autore non ne ha mai vedute (ma il mondo
\`e grande...), mentre trova largo impiego nelle piccole camme ad uso motoristico.
Nel secondo caso si impone il contatto mediante un {\em accoppiamento di forma}\index{accoppiamento!di forma}, cio\`e progettando una camma ``negativa'' detta
{\em camma coniugata}\index{camme!coniugata} o {\em camma desmodromica}\index{camme!desmodromica}, riportata in colore rosa
nella figura \ref{fig:f_camma_desmo}.
La generazione di questa camma potrebbe in linea di principio seguire il percorso
per la progettazione della camma ``primaria'' avendo, in questo caso, l'avvertenza di invertire i
segni delle leggi di moto. Di fatto, una volta disegnata la camma primaria,
baster\`a
considerare il prolungamento del cedente fino al luogo ove si desidera tastare 
la camma desmodromica e riportare una seconda rotella.

\begin{figure}[hbt]
\centering
\includegraphics[width=0.85\textwidth]{part2/camme/FIG/camma/rotella_interna.pdf}
\begin{picture}(0,0)(130,0)
\scriptsize{
}
\end{picture}
      \caption{\em Camma tastata dall'anello interno di un cuscinetto.}
 \label{fig:f_rotella_interna}
\end{figure}


\noindent Tale 
rotella, durante il movimento generato dalla camma primaria (sottintendendo il contatto
costante tra cedente e profilo),  ``taglier\`a'' la camma desmodromica o {\em
camma coniugata}\index{camme!coniugata}, e questo modo di 
procedere \`e anche quello utilizzato dal nostro codice
per produrre le illustrazioni \ref{fig:f_camma_desmo} e 
\ref{fig:f_desmo_costola}. Delle due citate figure, la seconda rappresenta
una soluzione molto elegante di accoppiamento
desmodromico, dove le rotelle tastano una costola. Purtroppo, questa
soluzione non \`e
di facilissima realizzazione in officina. Normalmente le camme coniugate
sono pi\`u sottili delle relative primarie e le loro rotelle sono di
minori dimensioni.
Questo \`e dovuto alla funzione ``di servizio'' da loro svolta,
che semplicemente consiste nel mantenere il contatto tra la punteria e la
camma primaria
e nel riportare il cedente nella posizione di partenza: esse
svolgono molto raramente azioni di lavoro. Persino le cautele sull'angolo di
pressione massimo e sul minimo raggio di curvatura convessa,
ammesse per queste camme 
(parliamo sempre delle fasi attive), possono essere leggermente rilassate.

\noindent Anche la fantasia pu\`o partecipare con successo nella progettazione delle camme.
Riportiamo, in figura \ref{fig:f_rotella_interna}, 
una soluzione poco comune di camma-punteria dove
il profilo della camma viene tastato dall'anello interno di un cuscinetto
(a rulli). In tale figura, il meccanismo viene rappresentato nel momento in
cui la camma ha svolto $180^{\circ}$ della sua rotazione e, stanti le leggi
proposte per questa camma, la punteria si trova alla massima estensione 
dell'alzata.
Le due curvature, quella della camma e quella della rotella,
sono, in questo caso, concordi. Pertanto,
nella \ref{eq_hertz} uno dei due
raggi cambia segno e la pressione andrebbe addirittura
ad annullarsi nel caso (da non contemplarsi) di raggi di curvatura
uguali tra loro.
\begin{figure}[hbt]
\centering
\begin{minipage}[b]{0.48\textwidth}
\centering
\includegraphics[width=0.9\textwidth]{part2/camme/FIG/albero.pdf}
\begin{picture}(0,0)(130,0)
\scriptsize{
}
\end{picture}
      \caption{\em Albero a camme per pressa di grandi dimensioni, tastato dall'anello interno di un cuscinetto.}
 \label{fig:f_albero}
\end{minipage}\hfill
\begin{minipage}[b]{0.48\textwidth}
\centering
\includegraphics[width=0.9\textwidth]{part2/camme/FIG/rotella.pdf}
\begin{picture}(0,0)(129,0)
\scriptsize{
}
\end{picture}
      \caption{\em Rotella costituita da cuscinetto a rulli che tasta la camma mediante il suo anello interno.}
     \label{fig:f_rotella}
\end{minipage}
\end{figure}
\noindent La pressione di contatto risulta quindi
notevolmente ridotta in questo tipo di camme che possono essere usate quando le forze da trasmettere 
sono ingenti. Nelle figure \ref{fig:f_albero} e \ref{fig:f_rotella} riportiamo una realizzazione
della tipologia di camme appena descritte e di una delle quattro rotelle,
con il desiderio di mostrare al lettore quali ragguardevoli 
dimensioni possono essere raggiunte da un albero a camme e da una rotella.

\endinput


